%!TEX root = /Users/zolkko/Projects/zolkko-alarm/doc/main.tex
\subsection{Разработка принципиальной электрической схемы управляющего устройства в системе Eagle}
Для создания принципиальных эдектрических схем в системе Eagle используется программа
Eagle Layout Editor.

Чтобы открыть новое окно Eagle Layout Editor нужно вызвать команду ''Schematic''
главного меню ''New''.

\subsubsection{Подбор библиотек компонентов}
% Сильной стороной Eagle является -- обшираня коллекция библиотек поставляемых с системой.
% В случае же если какой-либо необходимый компонент отстуствует в коллекции,

\subsubsection{Ввод принципиальной электрической схемы}

\subsubsection{Принцип работы схемы управления датчиком температуры холодного спая}
Для получение температуры холодного спая в устройстве управления используется цифровой датчик
температуры общего применения DS18B20.

Данный датчик работает по протоколу 1-wire -- минималистичный интерфейс, использующий
всего одну линию для передачи данных. Обычно в качестве ведущего устройства
шины 1-wire используют один выводов микроконтроллера, а для формирования
тайм-слотов необходимой длительности -- применяют либо задержки сформированные
командами NOP, либо пустыми циклами.


Однако, для формирования тайм-слотов такими способами, на высоких рабочих
частотах, микроконтроллер использует слишком большое количество машинных циклов.


Для решения этой проблемы компания Maxim Integrated Products Inc.
предлагает использовать \cite{max2usart} свободный модуль UART микроконтроллера.


Сконфигурировав модуль UART по схеме 8 бит данных, без чётности, один стоп-бит и
манипулируя скоростью передачи, его можно использовать для формирования
импульсов нужной длинны.


В шине 1-wire ведомое устройство для ответа должно иметь возможность
опускать или оставлять не именным уровень логической единицы,
передаваемый ведущим.
Однако, вывод TX модуля UART, фактически замкнутый на его же вход RX у большинства
микроконтроллеров не является выводом с общим коллектором, а значит и ведомое
устройство не сможет опустить шину до уровня логической единицы.

Для решения этой проблемы, в цепь TX перед первым ведомым устройством, внедряется
буферная схема с общим коллектором.

В схеме устройства управления, в качестве такого не инвертирующего буферного
элемента c общим стоком мной используются два полевых NPN транзистора с изолированным
затвором -- 2N7002.

Номиналы подтягивающих резисторов в цепи тока и цепи затвора транзисторов брались из расчёта:
\begin{itemize}
	\item максимальный ток DS18B20 -- 5 мА;
	\item максимальный ток 2N7002 -- 115 мА;
	\item максимальный ток вывода ATXMega -- 150 мА;
	\item минимальный уровень логической единицы при напряжении
		питания микроконтроллера 3.3 В -- 0.8 В.
\end{itemize}
Этим условиям удовлетворяют сопротивления номиналом 2.7 KОм.


\subsubsection{Расчёт параметров элементов схемы усиления сигнала термопары}
Сигналы от термопары принимают значения от микровольт до милливольт, поэтому необходимо принимать
дополнительные меры по снижению уровня шумов и наводок.
Обычно это экранирование и сокращение длины соединительных проводов.

Кроме того, учитывая, что термоэдс термопары в процессе её работы изменяется
с низкой частотой, можно подавлять помехи с помощью
фильтра нижних частот. В устройстве управления такие фильтры устанавливаются на выходах
каждого усилителя сигнала термопар.

Для упрощения схемы, устройства в ней используется фильтр нижих частот первого порядка, с
частотой среза 2 Гц и желаемой граничной частотой полосы задержания 4 Гц на 50 бД.

\begin{equation}
	F = \frac{1} {2 \times{} \pi{} RC}
\end{equation}

Соответственно при сопротивлении  $R = 1500$
$C = \frac{1}{2\pi{}R} = \frac{1}{1500 \times{} 6.28} = 0.0001$ Ф.


Результатом проделанной работы стала схема электрическая принципиальная (приложение Б).
