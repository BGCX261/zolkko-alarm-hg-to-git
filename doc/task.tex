%%
%% Something is gioing wrong. Dispite sections was defined as
%% \MakeUppercase they still remains as normal text.
%% However, \MakeUppercase works for english characters.
%%
%% So I'm gonna use this sections like this:
%%
%% \section[Постановка задачи]{ПОСТАНОВКА ЗАДАЧИ}
\section{ПОСТАНОВКА ЗАДАЧИ}
\begin{par}
Спроектировать и разработать устройство позволяющее в заданные моменты времени выдавать сигнал
звукового оповещения, производить индикацию текущего времени, а так же прозводить его синхронизацию
с удалённым сервером используя вычислительную сеть стандарта 802.3 Ethernet.
\end{par}

\begin{par}
Центральный вычислительный модуль устройства должен быть выполнен на микроконтроллере AVR компании
Atmel семейства XMega.
\end{par}

\begin{par}
Не смотря на то, что реализация подобного устройства на ПЛИС имеет ряд преимуществ по
сравнению с реализацией на основе микроконтроллеров, а именно более высокая скорость
работы и более низкое потребление энергии, новое семейство 8-битных микроконтроллеров
AVR переносит микроконтроллерные устройства на новый уровень системных характеристик,
которые в сочетации с более низкой стоимостью самих микроконтроллеров и иструментария
разработчика, делает их адекватным выбором при проектировании систем. Микроконтроллеры
AVR семейства XMega интегрируют в себе устройства ввода-вывода с улучшенными характеристиками,
более низкую потребляемую мощьность за счёт применения в них новой технологии picoPower.
Новая система обработки событий Event System, обеспечивает независимую от ЦПУ
быстродействующую передачу данных между внутренними переферийными устройствами 4-канального
контроллера ПДП, улучьшающего характеристики микроконтроллера. Микроконтроллеры имеют
быстродействующие 12-битнее модули АЦП и ЦАП. И ускоритель криптографических алгоритмов AES и DES.
Каждое перечисленное нововведение в семействе XMega микроконтроллеров AVR, позволит
в более сжатые сроки, а следовательно и с меньшими финансовыми затратами реализовать
целевое устройство, при этом отставание в производительности и потребляемой мощьности
становиться не значительным для данного класса устройств.
\end{par}

\begin{par}
Необходимо в качестве устройств ввода и вывода использовать один модуль сенсорного экрана
для уменьшения рабочей поверхности изделия.
\end{par}

\begin{par}
В качестве устройства взаимодействия с удалённым сетевым сервисом времени необходимо использовать
контроллер фирмы Micro Chip enc28j60. Применени этого контроллера позволит устройству производить
синхронизацию текущего времени по протоколу SNTP.
\end{par}

\begin{par}
Необходимо споектировать и разработать:
    \begin{itemize}
        \item{}Принципиальную электрическую схему;
        \item{}печатную плату.
    \end{itemize}
\end{par}

\begin{par}
Необходимо разработать на языке высокого уровня С++  программу управления устройва:
\begin{itemize}
    \item{} Модуль учёта текущего времени;
    \item{} модуль индикации текущего времени, выводящий информацию о текущем времени на ЖК-дисплее;
    \item{} модуль пользовательского ввода с сенсорного экрана, обеспечивающий первоначальную
            конфигурацию устройства, и установку времени срабатывания звукового сигнала;
    \item{} модуль воспроизведения звукового сигнала с внешней энергонезависимой памяти;
    \item{} модуль синхронизации текущего времени с удалённым SNTP сервисом.
\end{itemize}
\end{par}

\subsection{Актуальность задачи}
\begin{par}
Не смотря на то, что микроконтроллеры AVR серии XMega, первые версии которых были выпущены в 2009 году,
предлагают разработчикам в значительной степени более совершенный инструмент для
реализации своих идей, на фоне не значительного повышения в цене устройства, количество статей в журналах,
обзоров в сети Интернет, демонстрационных проектов и книг на русском языке с примерами их применения
ничтожно мало.
\end{par}

\begin{par}
Пример реализации устройства, выполненяемый с применением исключительно свободного и бесплатного
программного обеспечения и доступный любому зантересованному лицу по адресу http://zolkko-alarm.google.com
по свободной лицензии, может прояснить нюансы использования микроконтроллеров AVR серии
XMega в реальных устройства и применения языка высокого уровня C++ для написания программного
обеспечения 8-битных микроконтроллеров.
\end{par}

\begin{par}
Таким образом эта тема может представлять интерес как для специалистов в области микроэлектронники,
так и для студентов, руководителей детских кружков радио-любителей.
\end{par}

\newpage{}
