\section*{Заключение}
\addcontentsline{toc}{section}{Заключение}

В данном дипломном проекте была произведена автоматизированная разработка программно-аппаратного
комплекса на базе микроконтроллера AVR семейства XMega -- <<Универсальная система терморегулирования
на базе микроконтроллера AVR семейства XMega>>. В качестве системы проектирования принципиальной электрической
и топологической схемы использовалась программа Eagle компании CadSoft. Для реализации микропрограммы
микроконтроллера использовался язык программирования C++ входящий в состав коллекции компиляторов AVR GCC 4.3.3.
Для реализации контролирующего сервиса системы применялась система ErLang/OTP. Все программные коды
реализованные в ходе выполнения дипломного проекта содержат необходимые пояснения и комментарии.

Для анализа пригодности выбранных инструментальных средств, 
были проведены все необходимые этапы автоматизированного проектирования и разработки
программно-аппаратных комплексов. В ходе выполнения дипломного проектирования, так же был
проведён анализ пригодности микроконтроллеров AVR семейства XMega для автоматизации
технологических процессов, а так же возможности выполнения на 8 разрядных микроконтроллерах
зач, обычно решаемых на микроконтроллерах, микропроцессорах большей разрядности.

Разработанный программно-аппаратный комплекс <<Универсальная система терморегулирования на базе микроконтроллера
AVR семейства XMega>> является примером реализации, внедрение его в промышленную эксплуатацию
требует уточнений отдельных компонентов входящих в его состав.
