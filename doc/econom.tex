\section{ОРГАНИЗАЦИОННО-ЭКОНОМИЧЕСКАЯ ЧАСТЬ}

\subsection{Обоснование необходимости и актуальности разработки проекта}
\begin{par}
В существующем на данный момент варианте контроля и оперативного
вмешательства в ход соблюдения технологического процесса обработки выпускаемой продукции (колбас
МПП <<Волжский бекон>>) предусмотрен лишь контроль со стороны работника термического
участка-оператора термопечей. Данный вариант значительно повышает степень риска выпуска некачественной
продукции в следствии несоблюдения и выдержки требуемых параметров обработки. Ответственность, в
данном случае, полностью ложится на конкретного исполнителя и зависит от его внимательности,
добросовестности и личной заинтересованности в результатах своей работы.
\end{par}

\begin{par}
В современных условиях вариант совершенно не приемлемый, так как результат  работы всего
коллектива не может зависеть от желания или нежелания отдельного работника.
В этих целях, для устранения подобных проблем, широко применяется 
автоматизация технологических процессов. Автоматизация технологических процессов это
совокупность методови средств, предназначенных для реализации системы или систем, позволяющих
осуществлять управление самим технологическим процессом без непосредственного участия человека,
либо оставления за ним права принятия наиболее ответственного решения.
Как правило, в результате автоматизации технологического процесса создается АСУТП.
Основными целями автоматизации технологического процесса являются: повышение эффективности,
безопасности, экономичности и экологичности производственного процесса.
\end{par}

\subsection{Определение трудоемкости разработки системы контроля технологического процесса}
Для определения трудоемкости разработки программного продукта необходимо воспользоваться <<Укрупненными
нормами времени на изготовление и сопровождение программных средств вычислительной техник>>.
Стадиями разработки программных средств являются:
\begin{itemize}
    \item{} техническое задание;
    \item{} эскизный проект;
    \item{} технический проект;
    \item{} рабочий проект;
    \item{} внедрение.
\end{itemize}

Так как программный продукт является развитием определенного параметрического ряда ПП на прежнем типе ЭВМ/ОС,
то Кн=0.4 --- поправочный коэффициент, учитывающий степень новизны ПП и использования при разработке
программного продукта новых типов ЭВМ и ОС. Степень  охвата реализуемых функций стандартными ПП составляет 60\%, 
поэтому принимаем Кт=0.8 --- поправочный коэффициент, учитывающий степень 
использования в разработке типовых ПП.

Для разработки программного продукта используем языки JavaScript, Erlang, C++.
В качестве среды разработки используется персональная ЭВМ с установленной Apple Mac OS X 10.6.
Поправочный коэффициент, учитывающий характер среды разработки и средств разработки ПП, Кур=0.36.
Программный продукт имеет вторую группу сложности и выполняет следующие функции:

\begin{itemize}
    \item{} управление работой компонентов ПС (V1);
    \item{} обработка ошибочных данных (V2);
    \item{} вывод данных на экран (V3);
    \item{} система настройки ПС на условия применения (V4).
\end{itemize}

Общая трудоемкость разработки ПП расчитывается по формуле

\begin{equation}
    Т_{общ.} = \sum_{i=0}^{n}Т_i
\end{equation}
\begin{ESKDexplanation}
    \item[где ]{} $Т_i$ --- трудоемкость i-той стадии разработки ПП (чел.-дни);
    \item{} n --- количество стадий разработки.
\end{ESKDexplanation}

В соответствии с п.11 <<Укрупненных норм времени на изготовление и сопровождение программных
средств вычислительной техники» в том случае, когда на стадии технико-экономического
обоснования проекта невозможно расчитать точный объем ПП, то данный объем может быть 
получен на основании прогнозной оценки или путем применения нормативов по каталогу функций.
Общая трудоемкость разработки ПП определяется по формуле

\begin{equation}
    То = Тур \times{} Ксл
\end{equation}
\begin{ESKDexplanation}
    \item[где ]{} Тур --- трудоемкость разработки ПП с учетом конкретных условий (чел.-дн.)
    \item{} Ксл --- коэффициент сложности 
\end{ESKDexplanation}


\begin{equation}
    Т_{ур} = ТБ \times{} К_{ур}
\end{equation}

\begin{equation}
    К_{сл} = 1 + \sum{}_{i=0}{n}К_i
\end{equation}
\begin{ESKDexplanation}
    \item[где ]{} ТБ --- базова трудоемкость разработки ПП (чел.-дн);
    \item{} Кур --- поправочный коэффициент, учитывающий характер среды и средств разработки ПП;
    \item{} Кi --- коэффициент повышения сложности ПП.
\end{ESKDexplanation}

\begin{par}
С учетом вышесказанного получаем следующие значения расчетных величин:
$Кур = 0.34$ \\
$Кi  =  0.08$,  тогда  $Кур = 1.08$ \\
$Т_{баз} = 2291$ (чел.-дн.) \\
$V_0 = 6000$ (строк) \\
$Тур = ТБ \times{} Кур = 2291 \times{} 0.34 = 778.9$ \\
$То =  Тур \times{} Ксл = 778.94 \times{} 1.08 = 841.25 (чел.-дн.) $ \\
\end{par}

\begin{par}
Трудоемкость i-той стадии разработки программного проекта определяется
по формулам: 
\begin{equation}
	Ti = Li \times{} Kн \times{} To
\end{equation}
для i = 1, 2, 3 , 5.
\begin{equation}
	T_i = Li \times{}  KT \times{} Kн \times{} T0
\end{equation}
для i = 4.
\end{par}

\begin{ESKDexplanation}
	\item[где ]{} Кн --- поправочный коэффициент, учитывающий степень новизны ПП и
		использования при разработке новых типов ЭВМ и ОС;
	\item{} Кт --- поправочный коэффициент,учитывающий степень использования в разработках типовых ПП;
	\item{} Li --- удельный вес трудоемкости i-той стадии разработки ПП.
\end{ESKDexplanation}

\begin{par}
Определяем значения вышеуказанных коэффициентов: \\
$Кн = 0.4$ \\
$Кт = 0.8$ \\
$L_1 = 0.06$, тогда: $Т_1 = 0.06 \times{} 0.4 \times{} 841.25 = 20.19$ (чел.-дн.) \\
$L_2 = 0.2$; $Т_2 = 0.2 \times{} 0.4 \times{} 841.25 = 67.3$ (чел.-дн.) \\
$L_3 = 0.3$; $Т_3 = 0.3 \times{} 0.4 \times{} 841.25 = 100.95$ (чел.-дн.) \\
$L_4 = 0.34$; $Т_4 = 0.34 \times{} 0.4 \times{} 0.8 \times{} 841.25 = 91.53$ (чел.-дн.) \\
$L_5 = 0.1$; $Т_5 = 0.1 \times{} 0.4 \times{} 841.25 = 33.648$ (чел.-дн.)
\end{par}

\begin{par}
\begin{table}
\caption{Трудоемкость стадий разработки ПП}
\begin{tabular}{|l|p{7cm}|p{4cm}|r|}
\hline{}
№ & Стадия разработки & Удельный вес & Трудоемкость \\
\hline{}
1 & Техническое задание & 0.06 &  20.19 \\
\hline{}
2 & Эскизный проект & 0.2 & 67.30 \\
\hline{}
3 & Технический проект & 0.3 & 100.95 \\
\hline{}
4 & Рабочий проект & 0.34 & 91.53 \\
\hline{}
5 & Внедрение & 0.1 & 33.65 \\
\hline{}
& Итого: & & 314.00 (чел.- дн.) \\
\hline
\end{tabular}
\label{table:trdSoftDev}
\end{table}
Следовательно, общая трудоемкость разработки программного продукта
составляет: $To = \sum{}_{i=1}^{n=5} = T1 + T2 + T3 + T4 + T5 = 314.0$ (человеко-дней).
\end{par}


\subsection{Определение состава исполнителей}
Среднее число исполнителей, участвующих в разработке программного проекта расчитывается по формуле:
\begin{equation}
    Chi=\frac{To}{Fm\times{}D}
\end{equation}

\begin{ESKDexplanation}
    \item[где ]{} Fм --- фонд времени одного работающего в месяц (дней);
    \item{} Д --- директивный срок выполнения разработки (месяцев).
\end{ESKDexplanation}

Фонд рабочего времени одного работающего в месяц определяется по формуле:
\begin{equation}
    F = \frac{Dk - Db - Dn}{12}
\end{equation}

\begin{ESKDexplanation}
    \item[где ]{} Dk --- общее число дне в году, равно 365 дней;
    \item{} Dв --- число выходных дне в году;
    \item{} Dп --- число праздничных дне в году.
\end{ESKDexplanation}

В 2010 году при пятидневной рабочей неделе с двумя выходными, количество рабочих дней составило 249,
в том числе 5 сокращенных на один час рабочих дней. Количество выходных составило 116 с учетом 5
дополнительных дней отдыха. Следовательно фонд рабочего времени одного работающего в месяц всоответствии
с вышесказанным составит:

$F_m = \frac{249}{12} = 20.75 (дней)$


Среднее число исполнителей,участвующих в разработке проекта определяется по формуле:
\begin{equation}
	Чи = \frac{To} {Fм \times{} Д}
\end{equation}

\begin{ESKDexplanation}
	\item[где ] Д --- директивный срок исполнения проекта , принимаем 5 месяцев.
\end{ESKDexplanation}


Тогда расчетное число исполнителей будет равно : $Чи = \frac{314}{20.75 \times{} 5} = 3.06$ (человек).

\begin{par}
Исходя из полученного результата принимаем число исполнителей равным 3.
А именно: два инженера-программиста 3 квалификационного уровня с окладом 6500 рублей (повышающий коэффициент = 0.25);
один инженер-программист 4 квалификационного уровня (ведущий) исполняющий обязанности руководителя
проекта с окладом 7500р (повышающий коэффициент=0.44). Базовый оклад, без учета повышающего коэффициента,
составляет 5200 рублей.
\end{par}

\begin{par}
\begin{table}
\caption{Состав исполнителей исполнителей программного проекта}
\begin{tabular}{|l|c|c|c|c|r|}
\hline{}
№ & Профессия исполнителя & Количество человек & Оклад базовый & Повышающий  коэфф-т & Заработная плата с учетом повышающего коэф-та \\
\hline{}
1 & Инженер-программист & 2 & 5200 & 0.25 & 6500 \\
\hline{}
2 & Руководитель & 1 & 5200 & 0.44 & 7500 \\
\hline{}
& Всего &  & 3 & 20500(руб.) \\
\hline
\end{tabular}
\label{table:econSostavIspol}
\end{table}
\end{par}

\begin{par}
В таблице \ref{table:econSostavIspol} указана заработная плата исполнителей без учета премии, которая будет в
дальнейшем учтена, при расчете стоимости разработки программного продукта.
\end{par}

\subsection{Расчет  стоимости  разработки программного продукта}
\begin{par}
Цена на научно-техническую продукцию устанавливается на этапе технического задания
до начала проведения исследований.
При  этом она должна соответствовать ряду требований: возмещать  издержки разработчику,
регулировать спрос и предложение такого вида продукции,заинтересовывать
разработчика и заказчика в проведении эффективных разработок. В основе
договорной цены программного продута заложена  сметная  стоимость 
разработки , определяемая в калькуляционном разрезе.
\end{par}

Материалы и покупные изделия.

\begin{par}
\begin{table}
\caption{Расчет затрат на материалы и покупные изделия}
\begin{tabular}{|l|p{7cm}|p{4cm}|c|r|}
\hline{}
№ & Наименование & Цена за единицу& Норма расхоа(месяц) & Стоимость(руб) \\
\hline{}
1 & Бумага офисная & 135.0 & 1 & 135.0 \\
\hline{}
2 & Услуги интернета & 360.0 & --- & 350.0 \\
\hline{}
3 & Компакт диски & 20.0 & 1 & 20.0 \\
\hline{}
& Итого: & &  505.0 \\
Транспортно-заготовительные расходы & 10\% & 50.5 \\
\hline{}
& Всего: & &  555.5 \\
\hline
\end{tabular}
\label{table:reschZatrMat}
\end{table}
\end{par}

\begin{par}
Таким образом затраты на материалы, услуги и покупные изделия составляют:
\begin{itemize}
    \item{} в месяц --- 555.5 рублей
    \item{} в год --- 6666 рублей
    \item{} за время директивного сока --- 2777.5 рублей.
\end{itemize}
\end{par}

\subsubsection{Основная заработная плата исполнителей, занятых в разработке проекта}
Основная заработная плата определяется по формуле:
\begin{equation}
	Зп = З_{дол.ок.} + З_{дол.ок.} \times{} \sum{}_{i=1}^n
\end{equation}

\begin{ESKDexplanation}
	\item[где ]{} ЗП --- заработная плата;
	\item{} ЗДОЛ.ОК. --- должностной оклад работника.
\end{ESKDexplanation}

Должностной оклад работника определяется по формуле:
\begin{equation}
	Здол.окл. = Зокл \times{} КПОВ^{min}
\end{equation}

\begin{ESKDexplanation}
	\item[где ]{} Ki --- повышающий коэффициент (в данном случае это премия 15\%);
	\item{} Кпов --- повышающий коэффициент квалификационного уровня для конкретного работника (min);
	\item{} Зокл. --- базовый оклад по ПКГ (руб);
\end{ESKDexplanation}

\begin{par}
Подставляем исходные значения в формулы (1.9) и получаем:
\begin{itemize}
	\item{} Заработная плата инженера-программиста равна --- 7475 рублей;
	\item{} Заработная плата руководителя — 8625 рублей.
\end{itemize}
Таким образом общая основная заработная плата исполнителей, занятых в работе над проектом оставит:
\begin{itemize}
	\item{} за месяц --- 23575 рублей;
	\item{} за год --- 282900 рублей;
	\item{} за 5 месяцев --- 117875 рублей.
\end{itemize}
\end{par}
%
%
%\subsubsection{Дополнительная заработная плата}
%Дополнительная заработная плата определяется по формуле:
%\begin{equation}
%	Рдоп = \frac{Росн \times{} ННДоп} {100\%}
%\end{equation}
%
%\begin{ESKDexplanation}
%	\item[где ]{} Росн --- основная заработная плата;
%	\item{} Ндоп --- норматив дополнительной заработной платы (15-20\%);
%\end{ESKDexplanation}
%
%В наших расчетах принимаем процент премии равный 15.
%Производим вычисления и получаем следующие значения дополнительной
%заработной платы:
%\begin{itemize}
%	\item{} в месяц Рдоп = 3536.25 (руб.);
%	\item{} за год Рдоп = 42435 (руб.);
%	\item{} за 5 месяцев = 17681.25 (руб.).
%\end{par}
%
%
%\subsubsection{Единый социальный налог}
%\begin{par}
%Единый социальный налог определяется по формуле:
%\begin{equation}
%    Ротч = \frac{(Розп + Рдзп) \times{} Нотч} {100\%}
%\end{equation}
%
%Нотч. --- норматив отчислений на социальные  нужды, составляет 34 \%.
%Таким образом месячный размер отчислений составит-8015.5 рублей
%Отчисления в единый социальный налог за год составят -96186 рублей
%Отчисления за директивный период будут равны — 40077.5 рублей
%
%\subsubsection{Амортизационные  отчисления  с  оборудования}
%\begin{par}
%Определяются по формуле:
%\begin{equation}
%    А = ЦОБ \times{} НА
%\end{equation}
%\begin{ESKDexplanation}
%	\item[где ]{} ЦОБ --— балансовая стоимость одной ПЭВМ с периферией.
%\end{ESKDexplanation}
%В нашем случае используется три одинаковых электронно-вычислительные машины.
%Балансовая стоимость составляет 10000 рублей.
%\end{par}
%
%
%\begin{par}
%НА --- норма амортизации на компьютерную технику, определяется по формуле:
%\begin{equation}
%	НА = \frac{1}{Тп} \times{} 100\%
%\end{equation}
%\begin{ESKDexplanation}
%	\item[где ] ТП --- срок полезного использования объекта в месяцах.
%\end{ESKDexplanation}
%Для компьютерной техники срок полезного использования установлен 2--3 года.
%Тогда $НА= (\frac{1}{3} \times{} 12) \times{} 100 = 2.7$.
%\end{par}
%
%Подставляем полученное значение в формулу для вычисления амортизационных отчислений
%и получаем: $А = ЦОБ \times{} НА = 3 \times{} 10000 \times{} \frac{2.7}{100} = 810 (руб.)$.
%Таким образом ежемесячные отчисления на амортизацию оборудования составляют
%810 рублей. Отчисления за год составят: $810 \times{} 12 = 9720 руб.$. Отчисления за
%директивный период: $810 \times{} 5 = 4050 руб.$.
%
%\subsubsection{Затраты на силовую электроэнергию и освещение}
%Затраты на силовую электроэнергию определяются по формуле:
%\begin{equation}Зэн^{с} = Fэф \times{} Цэ \times{} Рэвм\end{equation}
%\begin{equation}Fэф = Fном \times{} Ксм \times{} (1 - \frac{a}{100})\end{equation}
%\begin{equation}Fном. = tсм \times{} Dр - tп \times{}Dп\end{equation}
%
%\begin{ESKDexplanation}
%	\item[где ]{} Fэф --- эффективный фонд рабочего времени оборудования;
%	\item{} Fном --- номинальный  фонд времени работы оборудования;
%	\item{} tсм --- длительность рабочей смены в часах;
%	\item{} Dр --- количество рабочих дней в плановом периоде;
%	\item{} tп = 1 --- продолжительность нерабочего времени в предпраздничные дни;
%	\item{} Dп --- количество предпраздничных дней;
%	\item{} α --- процент плановых потерь рабочего времени.
%\end{ESKDexplanation}
%
%\begin{par}
%Количество рабочих дней составляет 249 дней, в том числе 5 дней предпраздничных, сокращенных на один час.
%В этом случае номинальное количество рабочих часов составит: $Fном =  8 \times{} 249 - 1 \times{} 5 = 1987 (часов)$.
%\end{par}
%
%\begin{par}
%Эффективный фонд  рабочего времени определяется о формуле:
%\begin{equation}
%	Fэф = Fном \times{} Kсм \times{} (1 - \frac{a}{100})
%\end{equation}
%\end{par}
%
%\begin{ESKDexplanation}
%	\item[где ]{} Ксм --- количество смен;
%	\item{} α --- коэффициент, учитывающий процент плановых потерь рабочего времени.
%\end{ESKDexplanation}
%
%
%Подставляем  все значения в формулу и получаем величину эффективного рабочего времени:
%$Fэф = 1987 \times{} 1 \times{} (1 - \frac{20}{100}) = 1589.6 (ч.)$.
%
%\begin{par}
%Затраты на силовую электроэнергию вычисляются по следующей формуле:
%\begin{equation}
%	Зэн^с = Fэф \times{} Цэ \times{} Рэвм
%\end{equation}
%\begin{ESKDexplanation}
%	\item[где ]{} Цэ --- стоимость электроэнергии (в нашем случае = 4.94 руб/кВт);
%	\item{} Рэвм --- суммарная мощность ПЭВМ с  периферией (0.7 --- 1.2 кВт ).
%\end{ESKDexplanation}
%$Зэн^c = 1589.6 \times{} 4.94 \times{} 0.7 = 5496.84$ (руб.)
%\end{par}
%
%\subsubsection{Затраты электроэнергии на освещение}
%\begin{equation}
%	Зэн = Fэф \times{} Цэ \times{} Рос.
%\end{equation}
%\begin{ESKDexplanation}
%	\item[где ]{} Рос --- суммарная мощность, идущая на освещение (в нашем случае = 0.2кВт).
%\end{ESKDexplanation}
%
%Следовательно Зэн = 1589.6 * 4.94 * 0.2 = 1570.53 (руб.).
%Затраты за пять месяцев составят: 654.39 (руб.).
%Общие затраты на электроэнергию: $Зэн = Зэн^с + Зэн^{ос}$.
%
%
%Годовые  затраты на электроэнергию  составляют 7067.37  (руб.).
%Затраты за пять месяцев равны 2944,74 (руб.).
%
%
%\subsubsection{Расходы на профилактику оборудования}
%Расходы на профилактику оборудования вычисляются по формуле:
%\begin{equation}
%	Рпроф = \frac{ЦОБ \times{} Нпроф}{100\%}
%\end{equation}
%
%\begin{ESKDexplanation}
%	\item[где ]{} ЦОБ --- балансовая стоимость одной ПЭВМ с периферией;
%	\item{} Нпроф --- процент расходов на профилактику (2-4\%).
%\end{ESKDexplanation}
%
%$ Рпроф = (10000 * 2) / 100  = 200 (руб.) $
%
%Так как в работе над проектом задействованы три ПВЭМ, то суммарные расходы на профилактику составят 600 рублей.
%
%\subsubsection{Прочие производственные расходы}
%\begin{equation}
%Рпр.=\frac{Росн\times{}Нпр}{100\%}
%\end{equation}
%
%\begin{ESKDexplanation}
%	\item[где ]{} Росн --- основная заработная плата работников, обеспечивающих функционирование ПВЭМ;
%	\item{} Нпр --- процент прочих производственных расходов (30\%).
%\end{ESKDexplanation}
%
%\begin{par}
%$Рпр. = \frac{(23575 \times{} 30)}{100} = 7072.5$ (руб.). \\
%Производственные расходы за год =  84870 (руб.). \\
%Производственные расходы за 5 месяцев = 35362.5 (руб.).
%\end{par}
%
%\subsubsection{Накладные расходы}
%\begin{equation}
%	Рнак = \frac{Розп \times{} Ннакл}{100}
%\end{equation}
%\begin{ESKDexplanation}
%	\item[где ]{} Розп --- основная заработная плата;
%	\item{} Ннакл. --- норматив накладных расходов (120-150\%);
%\end{ESKDexplanation}
%
%$Рнак = \frac{23575 \times{} 120}{100} = 28290$ (руб.).
%
%Годовая величина накладных расходов равна = 28290 * 12 = 339480 (руб.).
%Накладные расходы за 5 месяцев равны = 28290 * 5 = 141450 (руб.).
%
%\subsubsection{Расчет расходов на содержание и эксплуатацию ПЭВМ}
%$Рсэ = Росн + Рдоп + Ротч + А + Зэн + Рпроф + Рпр$. \\
%$Рсэ = 282900 + 42435 + 96186 + 9720 + 7067.37 + 600 + 84870 = 517417.37$ (руб).
%
%Стоимость одного машино-часа работы ПВЭМ:
%\begin{equation}                                                   
%	Смч = \frac{Рсэ}{Fэф}
%\end{equation}
%\begin{ESKDexplanation}
%	\item{} Рсэ --- годовые расходы на содержание и эксплуатацию одной ПВЭМ. В нашем случае задействованы три ПВЭМ;
%	\item{} Fэф --- эффективный годовой фонд времени работы ПВЭМ.
%\end{ESKDexplanation}
%
%$Смч = \frac{172472.4}{1589.6} = 108.5$ (руб.)
%
%
%\begin{par}
%\begin{table}
%\caption{Расчет затрат на эксплуатацию оборудования (ПВЭМ)}
%\begin{tabular}{|l|p{7cm}|r|}
%\hline{}
%№ & Показатель & Значение \\
%\hline{}
%1 & Основная заработная плата & 282900.0 \\
%\hline{}
%2 & Дополнительная заработная плата & 42435.0 \\
%\hline{}
%3 & Отчисления единого социального налога & 96186.0 \\
%\hline{}
%4 & Амортизационные отчисления & 9720.0 \\
%\hline{}
%5 & Затраты на электроэнергию & 7067.37 \\
%\hline{}
%6 & Расходы на профилактику оборудования & 600.0 \\
%\hline{}
%7 & Прочие производственные расходы & 84870.0 \\
%\hline{}
%8 & Годовые расходы на содержание и эксплуатацию & 517417.37 \\
%\hline{}
%9 & Стоимость одного машино-часа работы ПВЭМ & 108.5 \\
%\hline{}
%10 & Расходы на содержание и эксплуатацию ПЭВМ, относящиеся к данному ПП & 225489.04 \\
%\hline
%\end{tabular}
%\label{table:reschZatrMat}
%\end{table}
%\end{par}
%
%
%
%\subsubsection{Расчет расходов на содержание и эксплуатацию ПЭВМ, относящихся к данному программному продукту}
%\begin{ESKDexplanation}
%Рсэ^{пс} = Cмч \times{} Тпп
%\end{ESKDexplanation}
%\begin{ESKDexplanation}
%	\item[где ]{} Смч --- стоимость одного машино-часа работы ПЭВМ;
%	\item{} Тпп --- суммарное время этапов , требующих использования ПЭВМ;
%\end{ESKDexplanation}
%
%$Тпп =  67.3 + 100.95 + 91.53 =259 .78$ (чел.-дн) (или 2078.24 час).
%$Рсэ =  108.5 * 2078.24 = 225489.04$ (руб.).
%
%
%\begin{par}
%\begin{table}
%\caption{Расчет сметной стоимости программного продукта}
%\begin{tabular}{|l|p{7cm}|r|}
%№ & Наименование затрат & Величина затрат \\
%\hline{}
%1 & Материалы и покупные изделия & 2777.5 \\
%\hline{}
%2 & Расходы на содержание эксплуатацию оборудования & 225302.0 \\
%\hline{}
%3 & Основная заработная плата исполнителей & 117875.0 \\
%\hline{}
%4 & Дополнительная заработная плата исполнителей & 17681.25 \\
%\hline{}
%5 & Единый социальный налог & 40077.5 \\
%\hline{}
%6 & Накладные расходы & 141450.0 \\
%\hline{}
%7 & Сметная стоимость разработки & 545163.25 \\
%\hline{}
%8 & Прибыль & 81774.49 \\
%\hline{}
%9 & Договорная цена & 5995.99 \\
%\hline{}
%10 & Затраты на эксплуатацию & 225489.04 \\
%\hline
%\end{tabular}
%\label{table:softwareCost}
%\end{table}
%\end{par}
%
%
%\subsection{Расчет трудоемкости сопровождения системы}
%Параметры влияющие на расчет трудоемкости сопровождения:
%
%\begin{par}
%\begin{table}
%\caption{Параметры влияющие на расчет трудоемкости сопровождения}
%\begin{tabular}{|l|p{7cm}|r|}
%№ & Название параметра & Принимаемые значения \\
%\hline{}
%1 & Характер поставки & Локальная поставка нестандартного комплекта ПП \\
%\hline{}
%2 & Характеристика средств разработки & Процедурные алгоритмические языки \\
%\hline{}
%3 & Степень участия службы сопровождения в разработке ПП & Не участвовала \\
%\hline{}
%4 & Характер внедрения & Локальное внедрение ПП \\
%\hline{}
%5 & Функции ПП & Управление работой компонентов ПС. Отработка ошибочных ситуаций. Вывод данных на экран система настройки ПС на условия применения \\
%\hline{}
%6 & Объем документации & 6000 строк \\
%\hline{}
%7 & Разработка дополнительных функций & Порядка 1000 условных машинных команд. \\
%\hline
%\end{tabular}
%\label{table:supportLaboriousness}
%\end{table}
%\end{par}
%
%
%Сопровождение подразумевает под собой поддержание технически исправного состояния.
%Обновление документов. Общая трудоемкость сопровождения ПП расчитывается по формуле:
%\begin{equation}
%	Тсопр. = Тос + Тпф + Тип + Топ
%\end{equation}
%\begin{ESKDexplanation}
%	\item[где ]{} Тос --- трудоемкость приемки и освоения ПП (чел.-дн.);
%	\item{} Тпф --- трудоемкость проверки функционирования поставленных ПП на контрольных задачах пользователя;
%	\item{} Тап --- трудоемкость изготовления и контроля комплекта поставки программной части на магнитном носителе;
%	\item{} Топ --- трудоемкость оказания технической помощи пользователю.
%\end{ESKDexplanation}
%
%\begin{par}
%\begin{equation}Тос=Ксл\times{}Кан\times{}Куч\times{}Нвр.ос.\end{equation}
%\begin{ESKDexplanation}
%	\item{} Ксл --- коэффициент, учитывающий сложность ПП;
%	\item{} Кан --- коэффициент, учитывающий наличие в фонде аналогов данного ПП;
%	\item{} Куч --- коэффициент, характеризующий степень участия службы сопровождения в разработке ПП;
%	\item{} Нвр.ос --- норма времени на приемку и освоение опытного образца ПП.
%\end{ESKDexplanation}
%\end{par}
%
%\begin{par}
%\begin{equation}Ксл =1 + sum{}_{i=1}^{n}Ксл_i\end{equation}
%\begin{ESKDexplanation}
%	\item[где ]{} Ксл._i --- коэффициент,  учитывающий уровень повышения сложности ПП.
%\end{ESKDexplanation}
%\end{par}
%
%\begin{par}
%Находим требуемые коэффициенты из таблиц: \\
%Ксл. = 0.21 \\
%Нвр.ос. = 23 (чел.-дн.) \\
%Куч. = 1.1 \\
%Кан. = 0.12
%\end{par}
%
%\begin{par}
%Определяем кэффициент сложности : \\
%Ксл. = 1 + 0.21 = 1.21. \\
%Тос. =  1.21 \times{} 0.12 \times{} 1.1 \times{} 23 = 3.67 (чел.-дн.) 
%\end{par}
%
%\begin{par}
%\begin{equation}
%	Тпф = Кхв \times{} Нвр.пф.
%\end{equation}
%\begin{ESKDexplanation}
%	\item[где ] Кхв. --- коэффициент, учитывающий характер внедрения;
%	\item{} Нвр.пф. --- норма времени на проверку функционирования поставляемых ПП на контрольных задачах пользователя.
%\end{ESKDexplanation}
%\end{par}
%
%\begin{par}
%Тогда :
%Тпф. = 1 * 4 = 4 ( чел.-дн.)
%Тип. = 1 (чел.-дн.)
%Топ. = Кхв * Ксл. * Нвр.оп.
%Нвр.оп. - норма времени на оказание технической помощи 
%Топ. = 1.0 * 1.21 * 2.5 =   3.025 (чел.-дн.)
%В итоге получаем суммарное время сопровождения :
%Тсопр. = Тос + Тпф +Тип + Топ    или
%Тсопр = 3.67 + 4 + 1 + 3.025 =  11.6 (чел.-дн.)
%\end{par}
%
%
%\subsection{Определение стоимости сопровождения программного продукта}
%\begin{equation}
%	Ссопр = Тсопр \times{} ЗПдн
%\end{equation}
%\begin{ESKDexplanation}
%	\item[где ]{} Зпдн --- дневная заработная плата исполнителя.
%\end{ESKDexplanation}
%
%Дневная заработная плата исполнителя, осуществляющего сопровождение программного продукта вычисляется по формуле:
%\begin{equation}
%	ЗПдн = \frac{ЗП}{Fм}
%\end{equation}
%\begin{ESKDexplanation}
%	\item[где ]{} ЗП --- месячная заработная плата (7475руб.);
%	\item{} Fм --- месячный фонд рабочего времени (20.75 дней).
%\end{ESKDexplanation}
%$Зпдн = \frac{7475}{20.75} = 360.24$ ( руб.).
%
%Стоимость сопровождения ПП: $Ссопр. = Тсопр. \times{} ЗПдн = 11.6 \times{} 369.24 = 4178.78$ (руб.).
%
%\subsection{Планирование цены и прогнозирование прибыли}
%\subsubsection{Сравнительный анализ существующих средств автоматизированного контроля технологических параметров и предлагаемого в данном проекте}
%Существующий рынок средств промышленной автоматики в основном представлен  продукцией фирмы <<ОВЕН>>.
%Предприятие выпускает широкую гамму контрольно-измерительных приборов.
%В том числе датчики измерения температуры (табл. \ref{table:temperatureSensors}), веса и других физических параметров.
%
%\begin{par}
%\begin{table}
%\caption{Датчики измерения температуры}
%\begin{tabular}{|l|p{7cm}|r|}
%№ & Наименование & Цена \\
%\hline{}
%1 & Универсальный измеритель-регулятор температуры ОВЕН ТРМ138 & 9440.0 \\
%\hline{}
%2 & Универсальный регулятор ОВЕН ТРМ148 & 10325.0 \\
%\hline{}
%3 & Устройство контроля температуры ОВЕН УКТ38-13 & 8791.0 \\
%\hline{}
%4 & Универсальный регулятор ОВЕН ТРМ138В & 14168.0 \\
%\hline
%\end{tabular}
%\label{table:temperatureSensors}
%\end{table}
%\end{par}
%
%\begin{par}                                                                 
%В отличии от представленной в данном проекте системы контроля и управления 
%технологическим процессом контрольные приборы фирмы <<ОВЕН>> не предусматривают
%возможность дистанционного управления и контроля за ходом выполнения установленных
%режимов обработки продукции. Все операции --- установка и контроль температуры, времени выдержки 
%производятся в ручном режиме оператором термопечей. На основе данных  о затратах на разработку и сопровождение, результатах прогнозирования объема продаж, определяем стоимость одного комплектапрограммного обеспечения. Исходя из потребности на рынке подобных продуктов
%можно предположить, что количество проданных копий составитоколо 300 штук. Стоимость выставляемого на рынок ПП определяется частью стоимости разработки ПП, затрат на сопровождение и прибыли
%организации-разработчика. Стоимость сопровождения остается постоянной для каждой установки
%ПП, а частичная стоимость разработки, приходящаяся на каждый комплект ПП,
%определяется исходя из данных о планируемом объеме установок.
%\end{par}
%
%
%Стоимость программного продукта можно рассчитать, используя соотношение:
%\begin{equation}
%	Цпп = (\delta{}С + Ссопр.) \times{} (1 + Dприб)
%\end{equation}
%
%\begin{equation}
%	\delta{}С = \frac{Спп}{N} \times{} (1 + Нст)
%\end{equation}
%
%\begin{ESKDexplanation}                       
%	\item[где ] $\delta{}С$ --- часть стоимости разработки приходящаяся на одну копию ПП;
%	\item{} Ссопр. --- стоимость сопровождения ПП;
%	\item{} Dприб --- процент прибыли, закладываемый в стоимость;
%	\item{} Нст -  Ставка банковского процента (при условии кредита).
%\end{ESKDexplanation}
%
%$ \delat{}C = \frac{545163.25}{300} = 1817.21$ (руб.).
%Стоимость одного комплекта ПП: $Цпп = (1817.21 +4178.78)*1.15  =  6895.39$ руб.
%
%
%Сумма прибыли составит:
%\begin{equation}
%	Сприб = Спп \times{} Dприб \times{} (1 - Нндс)
%\end{equation}
%подставим все значения и получим: $Сприб = 545163.25 \times{} 0.15 =  81774.49 (руб.)$.
%
%
%
%\subsection{Анализ конкурентноспособности и качества   разрабатываемой системы}
%В рыночной экономике решающим фактором коммерческого успеха товара является конкурентноспособность.
%Это многоаспектное понятие, означающее соответствие товара условиям рынка, конкретным
%требованиям потребителей не только по своим качественным, техническим.
%Экономическим, эстетическим характеристикам, но и по коммерческим условиям его реализации.
%Данная система контроля технологического процесса конкурентноспособна, как и другие аналоги
%на рынке программной продукции.
%
%\subsubsection{Анализ технической  прогрессивности , разрабатываемого ПП}
%Техническая прогрессивность  измеряемых показателей  характеризуется
%коэффициентом технической прогрессивности. Расчет этого коэффициента 
%осуществляется путем сравнения  технического уровня  товара-конкурента
%и разрабатываемого по отношению к эталонному уровню  ПП  данного направления.
%Расчет коэффициента технической прогрессивности производим по формуле:
%
%\begin{equation}
%	Ктп. = \frac{Ктн.}{КтБ}
%\end{equation}
%
%\begin{ESKDexplanation}
%	\item[где ]{} Ктн --- коэффициент технического уровня нового ПП;
%	\item{} КтБ --- коэффициент технического уровня базового ПП.
%\end{ESKDexplanation}
%
%
%\begin{equation}
%	Кт(БН) = \sum{}_{i=1}^{I}\beta{}\frac{П_i(Б.Н.)}{Пэi}
%\end{equation}
%
%(при прямой зависимости)
%
%
%\begin{equation}
%	Кт(бн) = sum{}_{i=1}{I}\beta{}\frac{Пэ_i}{П_i(бн)}
%\end{equation}
%(при обратной зависимости)
%
%\begin{ESKDexplanation}
%	\item{} \beta{}i --- коэффициент весомости. Устанавливается экспертным путем;
%	\item{} Пi(Б.Н) --- численное значение параметра соответственно базового и нового ПП.
%\end{ESKDexplanation}
%
%
%\begin{par}
%Анализируемый ПП технически прогрессивен, если Ктп больше 1. Результаты расчета сводим в таблицу (таб. \ref{table:calcKoefTechProgress}).
%\begin{table}
%\caption{Расчет коэффициента технической прогрессивности разрабатываемого программного продукта}
%\begin{tabular}{|l|c|c|c|c|c|c|c|r|}
%Наимение параметра & \beta{} & Пэ & Пб & Пн & $\frac{Пн}{Пэ}$ & $\frac{Пб}{Пэ}$ & $\beta{}\frac{Пб}{Пэ}$ & $\beta{}\frac{Пн}{Пэ}$ \\
%\hline{}
%Flash-память & 0.4 & 256 & 128 & 256 & 1 & 0.5 & 0.2 & 0.4 \\
%\hline{}
%Рабочее напряжение & 0.4 & 12В & 25В & 3.6В & 3& 0.5 & 0.2 & 132 \\
%\hline{}
%Количество линий ввода-вывода & 0.2 & 64 & 20 & 64 & 1 & 0.31 & 0.06 & 0.2 \\
%\hline{}
%Итого: & & & & & & 1 & 0.46 & 192 \\
%\hline
%\end{tabular}
%\label{table:calcKoefTechProgress}
%\end{table}               
%Находим коэффициент технической прогрессивности: $Ктп = \frac{1.92}{0.46} = 4.1$.
%Полученный коэффициент больше единицы, следовательно ПП технически прогрессивен.
%\end{par}
%
%
%\subsubsection{Анализ изменений функциональных возможностей разрабатываемого программного продукта}
%\begin{par}
%В этом разделе анализируются  эстетические, эргономические,
%экологические параметры , характеризующие функциональные возможности 
%разрабатываемого программного продукта, не имеющие количественного 
%выражения, трудно поддающиеся непосредственной количественной оценке.
%Коэффициент изменения функциональных возможностей рассчитывается
%по  формуле:
%\end{par}
%
%
%\begin{equation}
%	Кфв = \frac{Ан}{АБ}
%\end{equation}
%\begin{ESKDexplanation}
%	\item[где ]{} Ан и АБ --- суммарная бальная оценка не измеряемых параметров соответственно разрабатываемого и базового ПП.
%\end{ESKDexplanation}
%Функциональные возможности нового ПП лучше чем у базового, если КфВ  больше 1.
%
%
%Результаты расчетов сводим в таблицу.
%
%\begin{par}
%\begin{table}
%\caption{Расчет коэффициента изменения функциональных возможностей разрабатываемого ПП}
%\begin{tabular}{|l|p{7cm}|r|}
%Не измеряемые параметры & Характеристика параметра & Характеристика параметра & Бальная оценка & Бальная оценка \\
%базовый
%новый
%базовый
%новый
%\hline{}
%1-wire протокол & Да & Да & 2 & 2 \\
%\hline{}
%Защита доступа & Нет & Да & 1 & 2 \\
%\hline{}
%Учет статистики & Нет & Да & 1 & 2 \\
%\hline{}
%Отражение графических обьектов & Нет & Да & 1 & 2 \\
%\hline{}
%Итого & 5 & & & 8 \\
%\hline
%\end{tabular}
%\label{table:calcKoefTechProgress}
%\end{table}               
%
%Таким образом: $Кфв = 1.6$  больше единицы, функциональные возможности нового ПП лучше чем базового.
%
%
%
%\subsubsection{Анализ соответствия разрабатываемого ПП нормативным показателям}
%Нормативные или, так называемые, регламентируемые параметры характеризуют соответствие разрабатываемого
%ПП международным и национальным стандартам, нормативам, законодательным актам. Для оценки этого
%показателя применяется единичный или групповой показатель- Кнорм. Единичный показатель может принимать
%только одно значение: 1-соответствие и 0- несоответствие. Для разрабатываемого программного продукта Кнорм=1.
%
%
%\subsubsection{Анализ экономических параметров программного продукта}
%На данном этапе осуществляется анализ стоимостных параметров ПП, характеризующих его основные экономические свойства.
%В ходе анализа рассчитывается коэффициент цены потребления по  формуле:
%\begin{equation}
%	Кцп = \frac{Цп^{нов}}{Цп^{баз}}
%\end{equation}
%
%\begin{ESKDexplanation}
%	\item[где ] Цп^{нов}, Цп^{баз} --- цена потребления разрабатываемого и базового программного продукта.
%\end{ESKDexplanation}
%
%Цена потребления представляет собой затраты покупателя на приобретение, доработку.
%А также эксплуатацию анализируемого программного продукта на протяжении периода эксплуатации.
%
%\begin{equation}
%	Цп = Цпр. + Рдор. + Иэкс. \times{} Тн
%\end{equation}
%
%\begin{ESKDexplanation}
%	\item[где ]{} Цп --- цена потребления (руб.);
%	\item{} Цпр. --- цена приобретения копии ПП покупателем (руб.);
%	\item{} Иэкс. --- годовые эксплуатационные издержки потребителя (руб.);
%	\item{} Тн --- нормативный срок эксплуатации (лет);
%	\item{} Рдор. --- затраты па доработку;
%\end{ESKDexplanation}
%
%$Рдор.=Тдор.\times{}Здн$.
%
%\begin{equation}
%Тдор. = Крз \times{} Ксл. \times{} Куч. \times{} Нвр.др.
%\end{equation}
%
%\begin{ESKDexplanation}
%	\item[где ]{} Здн. --- дневная заработная плата исполнителя (Здн.баз = 800 руб.; Здн.нов. = 360.24 руб);
%	\item{} Тдор. --- время доработки. (Тдор.баз = 9.6 дн;   Тдор.нов. = 8 дн.);
%	\item{} Крз --- коэффициент, учитывающий язык программирования (Крз.баз. = 1.0;  Крз.нов. = 1.0);
%	\item{} Ксл --- коэффициент сложности (Ксл. Баз =1.1;  Ксл.нов. = 1.21);
%	\item{} Куч. --- коэффициент участия (Куч.баз =1.1 ;Куч.нов. =1.1 );
%	\item{} Нвр.дор. --- норма времени на доработку (Нвр.дор.баз. = 9.6 дней; Нвр.дор.нов. = 8 дней);
%\end{ESKDexplanation}
%
%Эксплуатационные издержки за весь период эксплуатации:
%\begin{equation}
%	Иэкс =  Рперс +  Рнакл +  Рпр
%\end{equation}
%
%\begin{equation}
%	Рперс = sum{} (n_i \times{} Огод \times{} (1 + \frac{Нотч}{100}) \times{} (1 + \frac{П}{100}))
%\end{equation}
%
%\begin{equation}
%	Рперс = \sum{}(n_i \times{} Огод \times (1 + \frac{Нотч}{100}) \times{} (1 + \frac{П}{100}))
%\end{equation}
%
%\begin{ESKDexplanation}
%	\item[где ]{} n --- численность персонала;
%	\item{} Огод --- среднегодовая заработная плата одного работника;
%	\item{} Нотч. --- процент отчисления на ЕСН;
%	\item{} П --- процент премии;
%	\item{} Ннакл. --- накладные расходы;
%	\item{} Рпр. --- прочие расходы ( 1\% - 3\%  от суммы всех эксплуатационных расходов);
%\end{ESKDexplanation}
%
%\begin{par}
%Все расчеты сводим в таблицы.
%\end{par}
%
%\begin{par}
%\begin{table}
%\caption{Расчет годовых эксплуатационных издержек потребителя ПП}
%\begin{tabular}{|l|p{7cm}|r|}
%Наименование расходов & Базовый & Новый \\
%Расходы на содержание персонала & 567241.0 & 435948.9\\
%Накладные расходы & 587241.0 & 523138.68\\
%Прочие расходы & 11744.82 & 9590.86 \\
%Итого & 1186196.82 & 968678.44 \\
%\end{tabular}
%\label{table:yearEkspIzd}
%\end{table}
%\end{par}
%
%
%\begin{par}
%\begin{table}
%\caption{Расчет цены потребления программного продукта}
%\begin{tabular}{|l|p{7cm}|r|}
%Наименование расходов &  Базовый ПП & Новый ПП \\
%\hline{}
%Продажная цена ПП &  8791.0 & 6895.39 руб. \\
%\hline{}
%Расходы на доработку & 9292.8 & 3835.84 \\
%\hline{}
%Эксплуатационные издержки потребителя за весь период эксплуатации & 9489574.56 & 7749427.52 \\
%\hline{}
%Цена потребления & 9597658.36 & 7759259.35\\
%\hline
%\end{tabular}
%\label{table:zhenaPotreblenija}
%\end{table}
%\end{par}
%
%Исходя из полученных результатов определяем коэффициент цены потребления: $Кцп.=0.816$ меньше единицы,
%следовательно экономические параметры разрабатываемого ПП лучше базовых.
%
%\subsection{Оценка конкурентноспособности программного продукта}
%В целом конкурентноспособность нового ПП по отношению к базовому можно оценить с помощью интегрального
%коэффициента конкурентноспособности, учитывающего все ранее рассчитанные параметры.
%
%\begin{equation}
%Ки = Ктп \times{} Кфв \times{} Кнорм \times{} \frac{1}{Кцп}
%\end{equation}
%
%Подставляем рассчитанные значения коэффициентов: $Ки = 4.1 \times{} 1.6 \times{} 1.0 \times{} \frac{1}{0.816} = 8.01$.
%Полученное значение больше единицы, значит ПП конкурентноспособен.
%
%\subsubsection{Анализ технико - экономических показателей разработки и эксплуатации ПП}
%Ранее рассчитанные показатели сводим в таблицу.
%
%\begin{par}
%\begin{table}
%\caption{Технико-экономические показатели разработки и эксплуатации  ПП}
%\begin{tabular}{|l|r|}
%Показатели & Значение (руб.) \\
%\hline{}
%Затраты на разработку & 545163.25 \\
%\hline{}
%Затраты на сопровождение & 4178.78 \\
%\hline{}
%Продажная цена & 6895.39 \\
%\hline{}
%Годовые эксплуатационные издержки потребителя & 968678.0 \\
%\hline{}
%Цена потребления & 775259.35 \\
%\hline{}
%Интегральный коэффициент   конкурентноспособности ПП & 8.01 \\
%\hline{}
%Коэффициент технической прогрессивности ПП & 4.1 \\
%\hline{}
%Коэффициент функциональных возможностей ПП & 1.6 \\
%\hline
%\end{tabular}
%\label{table:econSostavIspol}
%\end{table}
%\end{par}
%
%\subsection{Выводы}
%Произведенные расчеты показали, что рассматриваемый проект по сравнению с базовым имеет ряд преимуществ,
%которые могут позволить ему занять достойное место на рынке контрольно-измерительных систем.
%\newpage{}
