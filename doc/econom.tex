\section{ОРГАНИЗАЦИОННО-ЭКОНОМИЧЕСКАЯ ЧАСТЬ}

\subsection{Обоснование необходимости и актуальности разработки проекта}
\begin{par}
В существующем на данный момент варианте контроля и оперативного
вмешательства в ход соблюдения технологического процесса обработки выпускаемой продукции (колбас
МПП <<Волжский бекон>>) предусмотрен лишь контроль со стороны работника термического
участка-оператора термопечей. Данный вариант значительно повышает степень риска выпуска некачественной
продукции в следствии несоблюдения и выдержки требуемых параметров обработки. Ответственность, в
данном случае, полностью ложится на конкретного исполнителя и зависит от его внимательности,
добросовестности и личной заинтересованности в результатах своей работы.
\end{par}

\begin{par}
В современных условиях вариант совершенно не приемлемый, так как результат  работы всего
коллектива не может зависеть от желания или нежелания отдельного работника.
В этих целях, для устранения подобных проблем, широко применяется 
автоматизация технологических процессов. Автоматизация технологических процессов это
совокупность методови средств, предназначенных для реализации системы или систем, позволяющих
осуществлять управление самим технологическим процессом без непосредственного участия человека,
либо оставления за ним права принятия наиболее ответственного решения.
Как правило, в результате автоматизации технологического процесса создается АСУТП.
Основными целями автоматизации технологического процесса являются: повышение эффективности,
безопасности, экономичности и экологичности производственного процесса.
\end{par}

\subsection{Определение трудоемкости разработки системы контроля технологического процесса}
Для определения трудоемкости разработки программного продукта необходимо воспользоваться <<Укрупненными
нормами времени на изготовление и сопровождение программных средств вычислительной техник>>.
Стадиями разработки программных средств являются:
\begin{itemize}
    \item{} техническое задание;
    \item{} эскизный проект;
    \item{} технический проект;
    \item{} рабочий проект;
    \item{} внедрение.
\end{itemize}

Так как программный продукт является развитием определенного параметрического ряда ПП на прежнем типе ЭВМ/ОС,
то Кн=0.4 --- поправочный коэффициент, учитывающий степень новизны ПП и использования при разработке
программного продукта новых типов ЭВМ и ОС. Степень  охвата реализуемых функций стандартными ПП составляет 60\%, 
поэтому принимаем Кт=0.8 --- поправочный коэффициент, учитывающий степень 
использования в разработке типовых ПП.

Для разработки программного продукта используем языки JavaScript, Erlang, C++.
В качестве среды разработки используется персональная ЭВМ с установленной Apple Mac OS X 10.6.
Поправочный коэффициент, учитывающий характер среды разработки и средств разработки ПП, Кур=0.36.
Программный продукт имеет вторую группу сложности и выполняет следующие функции:

\begin{itemize}
    \item{} управление работой компонентов ПС (V1);
    \item{} обработка ошибочных данных (V2);
    \item{} вывод данных на экран (V3);
    \item{} система настройки ПС на условия применения (V4).
\end{itemize}

Общая трудоемкость разработки ПП расчитывается по формуле
\begin{equation}
\textrm{Т}_{\textrm{общ}} = \sum_{i=0}^{n}\textrm{Т}_i
\end{equation}

\begin{ESKDexplanation}
    \item[где ]{} $\textrm{Т}_i$ --- трудоемкость i-той стадии разработки ПП (чел.-дни);
    \item{} n --- количество стадий разработки.
\end{ESKDexplanation}

В соответствии с п.11 <<Укрупненных норм времени на изготовление и сопровождение программных
средств вычислительной техники» в том случае, когда на стадии технико-экономического
обоснования проекта невозможно расчитать точный объем ПП, то данный объем может быть 
получен на основании прогнозной оценки или путем применения нормативов по каталогу функций.
Общая трудоемкость разработки ПП определяется по формуле
\begin{equation}
	\textrm{Т}_{\textrm{о}} = \textrm{Т}_{\textrm{ур}} \times{} \textrm{К}_{\textrm{сл}}
\end{equation}
\begin{ESKDexplanation}
    \item[где ]{} $\textrm{Т}_{\textrm{ур}}$ --- трудоемкость разработки ПП с учетом конкретных условий (чел.-дн.);
    \item{} $\textrm{К}_{\textrm{сл}}$ --- коэффициент сложности.
\end{ESKDexplanation}

\begin{equation}
    \textrm{Т}_{\textrm{ур}} = \textrm{Т}_{\textrm{б}} \times{} \textrm{К}_{\textrm{ур}}
\end{equation}

\begin{equation}
    \textrm{К}_{\textrm{сл}} = 1 + \sum_{i=0}^{n}\textrm{К}_i
\end{equation}
\begin{ESKDexplanation}
    \item[где ]{} $\textrm{Т}_{\textrm{б}}$ --- базова трудоемкость разработки ПП (чел.-дн);
    \item{} $\textrm{К}_{\textrm{ур}}$ --- поправочный коэффициент, учитывающий характер среды и средств разработки ПП;
    \item{} $\textrm{К}_i$ --- коэффициент повышения сложности ПП.
\end{ESKDexplanation}
С учетом вышесказанного получаем следующие значения расчетных величин: \\
$\textrm{К}_{\textrm{ур}} = 0.34$ \\
$\textrm{К}_i  =  0.08$,  тогда  $\textrm{К}_{\textrm{ур}} = 1.08$ \\
$\textrm{Т}_{\textrm{баз}} = 2291$ (чел.-дн.) \\
$V_0 = 6000$ (строк) \\
$\textrm{Т}_{\textrm{ур}} = \textrm{ТБ} \times{} \textrm{К}_{\textrm{ур}} = 2291 \times{} 0.34 = 778.9$ \\
$\textrm{Т}_{\textrm{о}} =  \textrm{Т}_{\textrm{ур}} \times{} \textrm{К}_{\textrm{сл}} = 778.94 \times{} 1.08 = 841.25$ (чел.-дн.) \\

\begin{par}
Трудоемкость i-той стадии разработки программного проекта определяется по формулам:
\begin{equation}
	T_i = L_i \times{} \textrm{K}_{\textrm{н}} \times{} \textrm{T}_{\textrm{o}}
\end{equation}
\begin{ESKDexplanation}
	\item[для ]{} i = 1, 2, 3 , 5.
\end{ESKDexplanation}

\begin{equation}T_i = L_i \times{} \textrm{KT} \times{} \textrm{K}_{\textrm{н}} \times{} \textrm{T}_0\end{equation}
\begin{ESKDexplanation}\item[для ]{} i = 4.\end{ESKDexplanation}

\begin{ESKDexplanation}
	\item[где ]{} $\textrm{К}_{\textrm{н}}$ --- поправочный коэффициент, учитывающий степень новизны ПП и
		использования при разработке новых типов ЭВМ и ОС;
	\item{} $\textrm{К}_{\textrm{т}}$ --- поправочный коэффициент,учитывающий степень использования в разработках типовых ПП;
	\item{} $L_i$ --- удельный вес трудоемкости i-той стадии разработки ПП.
\end{ESKDexplanation}
Определяем значения вышеуказанных коэффициентов: \\
$\textrm{К}_{\textrm{н}} = 0.4$ \\
$\textrm{К}_{\textrm{т}} = 0.8$ \\
$L_1 = 0.06$, тогда: $\textrm{Т}_1 = 0.06 \times{} 0.4 \times{} 841.25 = 20.19$ (чел.-дн.) \\
$L_2 = 0.2$; $\textrm{Т}_2 = 0.2 \times{} 0.4 \times{} 841.25 = 67.3$ (чел.-дн.) \\
$L_3 = 0.3$; $\textrm{Т}_3 = 0.3 \times{} 0.4 \times{} 841.25 = 100.95$ (чел.-дн.) \\
$L_4 = 0.34$; $\textrm{Т}_4 = 0.34 \times{} 0.4 \times{} 0.8 \times{} 841.25 = 91.53$ (чел.-дн.) \\
$L_5 = 0.1$; $\textrm{Т}_5 = 0.1 \times{} 0.4 \times{} 841.25 = 33.648$ (чел.-дн.)
\end{par}

\begin{par}
\begin{table}
\caption{Трудоемкость стадий разработки ПП}
\begin{tabular}{|l|p{7cm}|p{4cm}|p{4cm}|}
\hline{}
№ & Стадия разработки & Удельный вес & Трудоемкость (чел.- дн.) \\
\hline{}
1 & Техническое задание & 0.06 &  20.19 \\
\hline{}
2 & Эскизный проект & 0.2 & 67.30 \\
\hline{}
3 & Технический проект & 0.3 & 100.95 \\
\hline{}
4 & Рабочий проект & 0.34 & 91.53 \\
\hline{}
5 & Внедрение & 0.1 & 33.65 \\
\hline{}
& Итого: & & 314.00  \\
\hline
\end{tabular}
\label{table:trdSoftDev}
\end{table}
Следовательно, общая трудоемкость разработки программного продукта
составляет: $\textrm{T}_{\textrm{o}} = \sum_{i=1}^{n=5} = T_1 + T_2 + T_3 + T_4 + T_5 = 314.0$ человеко-дней.
\end{par}


\subsection{Определение состава исполнителей}
Среднее число исполнителей, участвующих в разработке программного проекта расчитывается по формуле:
\begin{equation}
    \textrm{Ч}_{\textrm{и}} =\frac{\textrm{T}_{\textrm{o}}}{F_{\textrm{м}}\times{}\textrm{Д}}
\end{equation}
\begin{ESKDexplanation}
    \item[где ]{} $F_{\textrm{м}}$ --- фонд времени одного работающего в месяц (дней);
    \item{} $\textrm{Д}$ --- директивный срок выполнения разработки (месяцев).
\end{ESKDexplanation}
Фонд рабочего времени одного работающего в месяц определяется по формуле:
\begin{equation}
    F = \frac{ \textrm{Д}_{\textrm{к}} - \textrm{Д}_{\textrm{в}} - \textrm{Д}_{\textrm{п}}   }{12}
\end{equation}
\begin{ESKDexplanation}
    \item[где ]{} $\textrm{Д}_{\textrm{к}}$ --- общее число дне в году, равно 365 дней;
    \item{} $\textrm{Д}_{\textrm{в}}$ --- число выходных дне в году;
    \item{} $\textrm{Д}_{\textrm{п}}$ --- число праздничных дне в году.
\end{ESKDexplanation}



В 2010 году при пятидневной рабочей неделе с двумя выходными, количество рабочих дней составило 249,
в том числе 5 сокращенных на один час рабочих дней. Количество выходных составило 116 с учетом 5
дополнительных дней отдыха. Следовательно фонд рабочего времени одного работающего в месяц всоответствии
с вышесказанным составит: $F_{\textrm{м}} = \frac{249}{12} = 20.75$ дней.


Среднее число исполнителей, участвующих в разработке проекта определяется по формуле:
\begin{equation}
	\textrm{Ч}_{\textrm{и}} = \frac{\textrm{T}_{\textrm{o}}} {F_{\textrm{м}} \times{} \textrm{Д}}
\end{equation}
\begin{ESKDexplanation}
	\item[где ] Д --- директивный срок исполнения проекта, принимаем 5 месяцев.
\end{ESKDexplanation}
Тогда расчетное число исполнителей будет равно : $\textrm{Ч}_{\textrm{и}} = \frac{314}{20.75 \times{} 5} = 3.06$ человек.

\begin{par}
Исходя из полученного результата принимаем число исполнителей равным 3.
А именно: два инженера-программиста 3 квалификационного уровня с окладом 6500 рублей (повышающий коэффициент = 0.25);
один инженер-программист 4 квалификационного уровня (ведущий) исполняющий обязанности руководителя
проекта с окладом 7500р (повышающий коэффициент=0.44). Базовый оклад, без учета повышающего коэффициента,
составляет 5200 рублей.
\end{par}

\begin{table}
\caption{Состав исполнителей исполнителей программного проекта}
\begin{tabular}{|l|p{3cm}|p{2.5cm}|p{2.5cm}|p{3cm}|p{2.5cm}|}
\hline{}
№ & Профессия исполнителя & Количество человек & Оклад базовый & Повышающий  коэфф-т & Заработная плата с учетом повышающего коэф-та (руб.) \\
\hline{}
1 & Инженер-программист & 2 & 5200 & 0.25 & 6500 \\
\hline{}
2 & Руководитель & 1 & 5200 & 0.44 & 7500 \\
\hline{}
& Всего & 3  & & & 20500 \\
\hline
\end{tabular}
\label{table:employeeList}
\end{table}


\begin{par}
В таблице \ref{table:employeeList} указана заработная плата исполнителей без учета премии, которая будет в
дальнейшем учтена, при расчете стоимости разработки программного продукта.
\end{par}

\subsection{Расчет  стоимости  разработки программного продукта}
Цена на научно-техническую продукцию устанавливается на этапе технического задания
до начала проведения исследований.
При  этом она должна соответствовать ряду требований: возмещать  издержки разработчику,
регулировать спрос и предложение такого вида продукции,заинтересовывать
разработчика и заказчика в проведении эффективных разработок. В основе
договорной цены программного продута заложена  сметная  стоимость 
разработки, определяемая в калькуляционном разрезе.

\subsubsection{Материалы и покупные изделия}
\begin{table}
\caption{Расчет затрат на материалы и покупные изделия}
\begin{tabular}{|p{1cm}|p{4cm}|p{3cm}|p{3cm}|p{4cm}|}
\hline{}
№ & Наименование & Цена за единицу& Норма расхода (месяц) & Стоимость (руб.) \\
\hline{}
1 & Бумага офисная & 135.0 & 1 & 135.0 \\
\hline{}
2 & Услуги интернета & 360.0 & --- & 350.0 \\
\hline{}
3 & Компакт диски & 20.0 & 1 & 20.0 \\
\hline{}
   & Итого:              &          &    & 505.0 \\
\hline{}
4 & Транспортно-заготовительные расходы & & 10\%  & 50.5 \\
\hline{}
  & Всего: & & &  555.5 \\
\hline
\end{tabular}
\label{table:meterials}
\end{table}
Из таблицы \ref{table:meterials} видно, что затраты на материалы, услуги и покупные изделия составляют:
\begin{itemize}
    \item{} в месяц --- 555.5 руб.;
    \item{} в год --- 6666 руб.;
    \item{} за время директивного сока --- 2777.5 руб.
\end{itemize}


\subsubsection{Основная заработная плата исполнителей, занятых в разработке проекта}
Основная заработная плата определяется по формуле:
\begin{equation}
	\textrm{З}_{\textrm{п}} = \textrm{З}_{\textrm{дол.ок.}} + \textrm{З}_{\textrm{дол.ок.}} \times{} \sum_{i=1}^n
\end{equation}

\begin{ESKDexplanation}
	\item[где ]{} $\textrm{З}_{\textrm{п}}$ --- заработная плата;
	\item{} $\textrm{З}_{\textrm{дол.ок.}}$ --- должностной оклад работника.
\end{ESKDexplanation}
Должностной оклад работника определяется по формуле:
\begin{equation}
	\textrm{З}_{\textrm{дол.ок.}} = \textrm{З}_{\textrm{ок}} \times{} \textrm{К}_{\textrm{пов}}^{min}
\end{equation}
\begin{ESKDexplanation}
	\item[где ]{} $K_i$ --- повышающий коэффициент (в данном случае это премия 15\%);
	\item{} $\textrm{К}_{\textrm{пов}}^min$ --- повышающий коэффициент квалификационного уровня для конкретного работника;
	\item{} $\textrm{З}_{\textrm{окл}}$ --- базовый оклад по ПКГ (руб.).
\end{ESKDexplanation}
Подставляем исходные значения в формулы (1.9) и получаем: \\
Заработная плата инженера-программиста равна 7475 рублей. \\
Заработная плата руководителя --- 8625 рублей. \\
Таким образом общая основная заработная плата исполнителей, занятых в работе над проектом оставит: \\
За месяц --- 23575 рублей. \\
За год --- 282900 рублей. \\
За 5 месяцев --- 117875 рублей.


\subsubsection{Дополнительная заработная плата}
Дополнительная заработная плата определяется по формуле:
\begin{equation}
	\textrm{Р}_{\textrm{доп}} = \frac{\textrm{Р}_{\textrm{осн}} \times{} \textrm{Н}_{\textrm{доп}}} {100\%}
\end{equation}

\begin{ESKDexplanation}
	\item[где ]{} $\textrm{Р}_{\textrm{осн}}$ --- основная заработная плата;
	\item{} $\textrm{Н}_{\textrm{доп}}$ --- норматив дополнительной заработной платы (15-20\%);
\end{ESKDexplanation}
В наших расчетах принимаем процент премии равный 15.
Производим вычисления и получаем следующие значения дополнительной
заработной платы: \\
В месяц $\textrm{Р}_{\textrm{доп}} = 3536.25$ (руб.). \\
За год $\textrm{Р}_{\textrm{доп}} = 42435$ (руб.). \\
За 5 месяцев $\textrm{Р}_{\textrm{доп}} = 17681.25$ (руб.).


\subsubsection{Единый социальный налог}
Единый социальный налог определяется по формуле:
\begin{equation}
    \textrm{Р}_\textrm{отч} = \frac{(\textrm{Р}_\textrm{озп} + \textrm{Р}_\textrm{дзп}) \times{} \textrm{Н}_\textrm{отч}} {100\%}
\end{equation}
$\textrm{Н}_\textrm{отч}$ --- норматив отчислений на социальные  нужды, составляет 34\%.
Таким образом месячный размер отчислений составит 8015.5 рублей.
Отчисления в единый социальный налог за год составят 96186 рублей.
Отчисления за директивный период будут равны 40077.5 рублей.

\subsubsection{Амортизационные  отчисления  с  оборудования}
Амортизационные отчисления с оборудования определяются по формуле:
\begin{equation}
    \textrm{А} = \textrm{ЦОБ} \times{} \textrm{Н}_\textrm{а}
\end{equation}
\begin{ESKDexplanation}
	\item[где ]{} $\textrm{ЦОБ}$ --— балансовая стоимость одной ПЭВМ с периферией;
	\item{} $\textrm{Н}_\textrm{а}$ --- норма амортизации на компьютерную технику.
\end{ESKDexplanation}
В нашем случае используется три одинаковых электронно-вычислительные машины.
Балансовая стоимость составляет 10000 рублей.
Норма амортизации, определяется по формуле:
\begin{equation}
	\textrm{Н}_\textrm{а} = \frac{1}{\textrm{Т}_\textrm{п}} \times{} 100\%
\end{equation}
\begin{ESKDexplanation}
	\item[где ] ТП --- срок полезного использования объекта в месяцах.
\end{ESKDexplanation}
Для компьютерной техники срок полезного использования установлен 2--3 года.
Тогда $\textrm{Н}_\textrm{а}= (\frac{1}{3} \times{} 12) \times{} 100 = 2.7$.
Подставляем полученное значение в формулу для вычисления амортизационных отчислений
и получаем: $\textrm{А} = \textrm{ЦОБ} \times{} \textrm{Н}_\textrm{а} = 3 \times{} 10000 \times{} \frac{2.7}{100} = 810$ (руб.).
Таким образом ежемесячные отчисления на амортизацию оборудования составляют
810 рублей. Отчисления за год составят: $810 \times{} 12 = 9720$ (руб). Отчисления за
директивный период: $810 \times{} 5 = 4050$ (руб).

\subsubsection{Затраты на силовую электроэнергию и освещение}
Затраты на силовую электроэнергию определяются по формуле:
\begin{equation}\textrm{З}_\textrm{эн}^{с} = F_\textrm{эф} \times{} \textrm{Ц}_\textrm{э} \times{} \textrm{Р}_\textrm{эвм}\end{equation}
\begin{equation}F_\textrm{эф} = F_\textrm{ном} \times{} \textrm{К}_\textrm{см} \times{} (1 - \frac{\alpha}{100})\end{equation}
\begin{equation}F_\textrm{ном} = t_\textrm{см} \times{} \textrm{Д}_\textrm{р} - t_\textrm{п} \times{} \textrm{Д}_\textrm{п}\end{equation}

\begin{ESKDexplanation}
	\item[где ]{} $F_\textrm{эф}$ --- эффективный фонд рабочего времени оборудования;
	\item{} $F_\textrm{ном}$ --- номинальный  фонд времени работы оборудования;
	\item{} $t_\textrm{см}$ --- длительность рабочей смены в часах;
	\item{} $\textrm{Д}_\textrm{р}$ --- количество рабочих дней в плановом периоде;
	\item{} $\textrm{t}_\textrm{п} = 1$ --- продолжительность нерабочего времени в предпраздничные дни;
	\item{} $\textrm{Д}_\textrm{п}$ --- количество предпраздничных дней;
	\item{} $\alpha$ --- процент плановых потерь рабочего времени.
\end{ESKDexplanation}
Количество рабочих дней составляет 249 дней, в том числе 5 дней предпраздничных, сокращенных на один час.
В этом случае номинальное количество рабочих часов составит: $Fном =  8 \times{} 249 - 1 \times{} 5 = 1987$ (ч.).
Эффективный фонд  рабочего времени определяется о формуле:
\begin{equation}
	F_\textrm{эф} = F_\textrm{ном} \times{} \textrm{K}_\textrm{см} \times{} (1 - \frac{\alpha}{100})
\end{equation}
\begin{ESKDexplanation}
	\item[где ]{} $\textrm{К}_\textrm{см}$ --- количество смен;
	\item{} $\alpha$ --- коэффициент, учитывающий процент плановых потерь рабочего времени.
\end{ESKDexplanation}
Подставляем  все значения в формулу и получаем величину эффективного рабочего времени:
$F_\textrm{эф} = 1987 \times{} 1 \times{} (1 - \frac{20}{100}) = 1589.6$ (ч.).

\begin{par}
Затраты на силовую электроэнергию вычисляются по следующей формуле:
\begin{equation}
	\textrm{З}_\textrm{эн}^{с} = F_\textrm{эф} \times{} \textrm{Ц}_\textrm{э} \times{} \textrm{Р}_\textrm{эвм}
\end{equation}
\begin{ESKDexplanation}
	\item[где ]{} $\textrm{Ц}_\textrm{э}$ --- стоимость электроэнергии (в нашем случае = 4.94 руб/кВт);
	\item{} $\textrm{Р}_\textrm{эвм}$ --- суммарная мощность ПЭВМ с  периферией (0.7 --- 1.2 кВт ).
\end{ESKDexplanation}
$\textrm{З}_\textrm{эн}^c = 1589.6 \times{} 4.94 \times{} 0.7 = 5496.84$ (руб.).
\end{par}

\subsubsection{Затраты электроэнергии на освещение}
\begin{equation}
	\textrm{З}_\textrm{эн} = F_\textrm{эф} \times{} \textrm{Ц}_\textrm{э} \times{} \textrm{Р}_\textrm{ос}.
\end{equation}
\begin{ESKDexplanation}
	\item[где ]{} $\textrm{Р}_\textrm{ос}$ --- суммарная мощность, идущая на освещение (в нашем случае = 0.2кВт).
\end{ESKDexplanation}
Следовательно $\textrm{З}_\textrm{эн} = 1589.6 \times{} 4.94 \times{} 0.2 = 1570.53$ (руб.).
Затраты за пять месяцев составят: 654.39 (руб.).
Общие затраты на электроэнергию: $\textrm{З}_\textrm{эн} = \textrm{З}_\textrm{эн}^{с} + \textrm{З}_\textrm{эн}^{ос}$.
Годовые  затраты на электроэнергию  составляют 7067.37  (руб.). Затраты за пять месяцев равны 2944,74 (руб.).


\subsubsection{Расходы на профилактику оборудования}
Расходы на профилактику оборудования вычисляются по формуле:
\begin{equation}
	\textrm{Р}_\textrm{проф} = \frac{\textrm{ЦОБ} \times{} \textrm{Н}_\textrm{проф}}{100\%}
\end{equation}
\begin{ESKDexplanation}
	\item[где ]{} $\textrm{ЦОБ}$ --- балансовая стоимость одной ПЭВМ с периферией;
	\item{} $\textrm{Н}_\textrm{проф}$ --- процент расходов на профилактику (2-4\%).
\end{ESKDexplanation}
$\textrm{Р}_\textrm{проф} = (10000 * 2) / 100  = 200$ (руб.).
Так как в работе над проектом задействованы три ПВЭМ, то суммарные расходы на профилактику составят 600 рублей.

\subsubsection{Прочие производственные расходы}
\begin{equation}
\textrm{Р}_\textrm{пр}=\frac{\textrm{Р}_\textrm{осн}\times{}\textrm{Н}_\textrm{пр}}{100\%}
\end{equation}
\begin{ESKDexplanation}
	\item[где ]{} $\textrm{Р}_\textrm{осн}$ --- основная заработная плата работников, обеспечивающих функционирование ПВЭМ;
	\item{} $\textrm{Н}_\textrm{пр}$ --- процент прочих производственных расходов (30\%).
\end{ESKDexplanation}
$\textrm{Р}_\textrm{пр} = \frac{(23575 \times{} 30)}{100} = 7072.5$ (руб.). \\
Производственные расходы за год =  84870 (руб.). \\
Производственные расходы за 5 месяцев = 35362.5 (руб.).

\subsubsection{Накладные расходы}
\begin{equation}
	\textrm{Р}_\textrm{нак} = \frac{\textrm{Р}_\textrm{озп} \times{} \textrm{Н}_\textrm{накл}}{100}
\end{equation}
\begin{ESKDexplanation}
	\item[где ]{} $\textrm{Р}_\textrm{озп}$ --- основная заработная плата;
	\item{} $\textrm{Н}_\textrm{накл}$ --- норматив накладных расходов (120-150\%);
\end{ESKDexplanation}
$\textrm{Р}_\textrm{нак} = \frac{23575 \times{} 120}{100} = 28290$ (руб.).
Годовая величина накладных расходов равна $= 28290 \times{} 12 = 339480$ (руб.).
Накладные расходы за 5 месяцев равны $= 28290 \times{} 5 = 141450$ (руб.).

\subsubsection{Расчет расходов на содержание и эксплуатацию ПЭВМ}
$\textrm{Р}_\textrm{сэ} = \textrm{Р}_\textrm{осн} + \textrm{Р}_\textrm{доп} + \textrm{Р}_\textrm{отч} + \textrm{А} + \textrm{З}_\textrm{эн} + \textrm{Р}_\textrm{проф} + \textrm{Р}_\textrm{пр}$. \\
$\textrm{Р}_\textrm{сэ} = 282900 + 42435 + 96186 + 9720 + 7067.37 + 600 + 84870 = 517417.37$ (руб).

Стоимость одного машино-часа работы ПВЭМ:
\begin{equation}                                                   
	\textrm{С}_\textrm{мч} = \frac{\textrm{Р}_\textrm{сэ}}{F_\textrm{эф}}
\end{equation}
\begin{ESKDexplanation}
	\item{} $\textrm{Р}_\textrm{сэ}$ --- годовые расходы на содержание и эксплуатацию одной ПВЭМ. В нашем случае задействованы три ПВЭМ;
	\item{} $F_\textrm{эф}$ --- эффективный годовой фонд времени работы ПВЭМ.
\end{ESKDexplanation}
$\textrm{С}_\textrm{мч} = \frac{172472.4}{1589.6} = 108.5$ (руб.)

\begin{table}
\caption{Расчет затрат на эксплуатацию оборудования (ПВЭМ)}
\begin{tabular}{|l|p{13cm}|r|}
\hline{}
№ & Показатель & Значение \\
\hline{}
1 & Основная заработная плата & 282900.0 \\
\hline{}
2 & Дополнительная заработная плата & 42435.0 \\
\hline{}
3 & Отчисления единого социального налога & 96186.0 \\
\hline{}
4 & Амортизационные отчисления & 9720.0 \\
\hline{}
5 & Затраты на электроэнергию & 7067.37 \\
\hline{}
6 & Расходы на профилактику оборудования & 600.0 \\
\hline{}
7 & Прочие производственные расходы & 84870.0 \\
\hline{}
8 & Годовые расходы на содержание и эксплуатацию & 517417.37 \\
\hline{}
9 & Стоимость одного машино-часа работы ПВЭМ & 108.5 \\
\hline{}
10 & Расходы на содержание и эксплуатацию ПЭВМ, относящиеся к данному ПП & 225489.04 \\
\hline
\end{tabular}
\label{table:staffExp}
\end{table}



\subsubsection{Расчет расходов на содержание и эксплуатацию ПЭВМ, относящихся к данному программному продукту}
\begin{equation}
\textrm{Р}_\textrm{сэ}^{\textrm{пс}} = \textrm{C}_\textrm{мч} \times{} \textrm{Т}_\textrm{пп}
\end{equation}
\begin{ESKDexplanation}
	\item[где ]{} $\textrm{С}_\textrm{мч}$ --- стоимость одного машино-часа работы ПЭВМ;
	\item{} $\textrm{Т}_\textrm{пп}$ --- суммарное время этапов , требующих использования ПЭВМ.
\end{ESKDexplanation}
$\textrm{Т}_\textrm{пп} =  67.3 + 100.95 + 91.53 =259 .78$ (чел.-дн) (или 2078.24 час).
$\textrm{Р}_\textrm{сэ} =  108.5 * 2078.24 = 225489.04$ (руб.).

\begin{table}
\caption{Расчет сметной стоимости программного продукта}
\begin{tabular}{|l|p{12cm}|r|}
\hline{}
№ & Наименование затрат & Величина затрат \\
\hline{}
1 & Материалы и покупные изделия & 2777.5 \\
\hline{}
2 & Расходы на содержание эксплуатацию оборудования & 225302.0 \\
\hline{}
3 & Основная заработная плата исполнителей & 117875.0 \\
\hline{}
4 & Дополнительная заработная плата исполнителей & 17681.25 \\
\hline{}
5 & Единый социальный налог & 40077.5 \\
\hline{}
6 & Накладные расходы & 141450.0 \\
\hline{}
7 & Сметная стоимость разработки & 545163.25 \\
\hline{}
8 & Прибыль & 81774.49 \\
\hline{}
9 & Договорная цена & 5995.99 \\
\hline{}
10 & Затраты на эксплуатацию & 225489.04 \\
\hline
\end{tabular}
\label{table:softwareCost}
\end{table}



\subsection{Расчет трудоемкости сопровождения системы}
Параметры влияющие на расчет трудоемкости сопровождения:
\begin{table}
\caption{Параметры влияющие на расчет трудоемкости сопровождения}
\begin{tabular}{|l|p{8cm}|p{8cm}|}
\hline{}
№ & Название параметра & Принимаемые значения \\
\hline{}
1 & Характер поставки & Локальная поставка нестандартного комплекта ПП \\
\hline{}
2 & Характеристика средств разработки & Процедурные алгоритмические языки \\
\hline{}
3 & Степень участия службы сопровождения в разработке ПП & Не участвовала \\
\hline{}
4 & Характер внедрения & Локальное внедрение ПП \\
\hline{}
5 & Функции ПП & Управление работой компонентов ПС. Отработка ошибочных ситуаций.
Вывод данных на экран система настройки ПС на условия применения \\
\hline{}
6 & Объем документации & 6000 строк \\
\hline{}
7 & Разработка дополнительных функций & Порядка 1000 условных машинных команд. \\
\hline
\end{tabular}
\label{table:supportFactors}
\end{table}


Сопровождение подразумевает под собой поддержание технически исправного состояния.
Обновление документов. Общая трудоемкость сопровождения ПП расчитывается по формуле:
\begin{equation}
	\textrm{Т}_\textrm{сопр} = \textrm{Т}_\textrm{ос} + \textrm{Т}_\textrm{пф} + \textrm{Т}_\textrm{ип} + \textrm{Т}_\textrm{оп}
\end{equation}
\begin{ESKDexplanation}
	\item[где ]{} $\textrm{Т}_\textrm{ос}$ --- трудоемкость приемки и освоения ПП (чел.-дн.);
	\item{} $\textrm{Т}_\textrm{пф}$ --- трудоемкость проверки функционирования поставленных ПП на контрольных задачах пользователя;
	\item{} $\textrm{Т}_\textrm{ап}$ --- трудоемкость изготовления и контроля комплекта поставки программной части на магнитном носителе;
	\item{} $\textrm{Т}_\textrm{оп}$ --- трудоемкость оказания технической помощи пользователю.
\end{ESKDexplanation}
\begin{equation}
	\textrm{Т}_\textrm{ос}=\textrm{К}_\textrm{сл}\times{}\textrm{К}_\textrm{ан}\times{}\textrm{К}_\textrm{уч}\times{}\textrm{Н}_\textrm{вр.ос}
\end{equation}
\begin{ESKDexplanation}
	\item[где ]{} $\textrm{К}_\textrm{сл}$ --- коэффициент, учитывающий сложность ПП;
	\item{} $\textrm{К}_\textrm{ан}$ --- коэффициент, учитывающий наличие в фонде аналогов данного ПП;
	\item{} $\textrm{К}_\textrm{уч}$ --- коэффициент, характеризующий степень участия службы сопровождения в разработке ПП;
	\item{} $\textrm{Н}_\textrm{вр.ос}$ --- норма времени на приемку и освоение опытного образца ПП.
\end{ESKDexplanation}
\begin{equation}
	\textrm{К}_\textrm{сл} =1 + \sum_{i=1}^{n}\textrm{К}_\textrm{сл i}
\end{equation}
\begin{ESKDexplanation}
	\item[где ]{} $\textrm{К}_\textrm{сл i}$ --- коэффициент,  учитывающий уровень повышения сложности ПП.
\end{ESKDexplanation}
Находим требуемые коэффициенты из таблиц: \\
$\textrm{К}_\textrm{сл} = 0.21$ \\
$\textrm{Н}_\textrm{вр.ос} = 23$ (чел.-дн.) \\
$\textrm{К}_\textrm{уч} = 1.1$ \\
$\textrm{К}_\textrm{ан} = 0.12$


\begin{par}
Определяем кэффициент сложности : \\
$\textrm{К}_\textrm{сл} = 1 + 0.21 = 1.21$. \\
$\textrm{Т}_\textrm{ос} =  1.21 \times{} 0.12 \times{} 1.1 \times{} 23 = 3.67$ (чел.-дн.) 
\end{par}

\begin{equation}
	\textrm{Т}_\textrm{пф} = \textrm{К}_\textrm{хв} \times{} \textrm{Н}_\textrm{вр.пф}
\end{equation}
\begin{ESKDexplanation}
	\item[где ] $\textrm{К}_\textrm{хв}$ --- коэффициент, учитывающий характер внедрения;
	\item{} $\textrm{Н}_\textrm{вр.пф}$ --- норма времени на проверку функционирования поставляемых ПП на контрольных задачах пользователя.
\end{ESKDexplanation}
Тогда : \\
$\textrm{Т}_\textrm{пф} = 1 * 4 = 4$ ( чел.-дн.) \\
$\textrm{Т}_\textrm{ип} = 1$ (чел.-дн.) \\
$\textrm{Т}_\textrm{оп} = \textrm{К}_\textrm{хв} \times{} \textrm{К}_\textrm{сл} \times{} \textrm{Н}_\textrm{вр.оп}$ \\
$\textrm{Н}_\textrm{вр.оп}$ - норма времени на оказание технической помощи. \\
$\textrm{Т}_\textrm{оп} = 1.0 * 1.21 * 2.5 = 3.025$ (чел.-дн.) \\
В итоге получаем суммарное время сопровождения:
$\textrm{Т}_\textrm{сопр} = \textrm{Т}_\textrm{ос} + \textrm{Т}_\textrm{пф} +\textrm{Т}_\textrm{ип} + \textrm{Т}\textrm{оп}$    или
$\textrm{Т}_\textrm{сопр} = 3.67 + 4 + 1 + 3.025 = 11.6$ (чел.-дн.)


\subsection{Определение стоимости сопровождения программного продукта}
\begin{equation}
	\textrm{С}_\textrm{сопр} = \textrm{Т}_\textrm{сопр} \times{} \textrm{З}_\textrm{пдн}
\end{equation}
\begin{ESKDexplanation}
	\item[где ]{} $\textrm{З}_\textrm{пдн}$ --- дневная заработная плата исполнителя.
\end{ESKDexplanation}
Дневная заработная плата исполнителя, осуществляющего сопровождение программного продукта вычисляется по формуле:
\begin{equation}
	\textrm{З}_\textrm{пдн} = \frac{\textrm{З}_\textrm{п}}{F_\textrm{м}}
\end{equation}
\begin{ESKDexplanation}
	\item[где ]{} $\textrm{З}_\textrm{п}$ --- месячная заработная плата (7475руб.);
	\item{} $F_\textrm{м}$ --- месячный фонд рабочего времени (20.75 дней).
\end{ESKDexplanation}
$\textrm{З}_\textrm{пдн} = \frac{7475}{20.75} = 360.24$ ( руб.).
Стоимость сопровождения ПП: $\textrm{С}_\textrm{сопр} = \textrm{Т}_\textrm{сопр} \times{} \textrm{З}_\textrm{пдн} = 11.6 \times{} 369.24 = 4178.78$ (руб.).


\subsection{Планирование цены и прогнозирование прибыли}
\subsubsection{Сравнительный анализ существующих средств автоматизированного контроля технологических параметров и предлагаемого в данном проекте}
Существующий рынок средств промышленной автоматики в основном представлен  продукцией фирмы <<ОВЕН>>.
Предприятие выпускает широкую гамму контрольно-измерительных приборов.
В том числе датчики измерения температуры (табл. \ref{table:temperatureSensors}), веса и других физических параметров.

\begin{table}
\caption{Датчики измерения температуры}
\begin{tabular}{|l|p{14cm}|r|}
\hline{}
№ & Наименование & Цена \\
\hline{}
1 & Универсальный измеритель-регулятор температуры ОВЕН ТРМ138 & 9440.0 \\
\hline{}
2 & Универсальный регулятор ОВЕН ТРМ148 & 10325.0 \\
\hline{}
3 & Устройство контроля температуры ОВЕН УКТ38-13 & 8791.0 \\
\hline{}
4 & Универсальный регулятор ОВЕН ТРМ138В & 14168.0 \\
\hline
\end{tabular}
\label{table:temperatureSensors}
\end{table}

\begin{par}                                                                 
В отличии от представленной в данном проекте системы контроля и управления 
технологическим процессом контрольные приборы фирмы <<ОВЕН>> не предусматривают
возможность дистанционного управления и контроля за ходом выполнения установленных
режимов обработки продукции. Все операции --- установка и контроль температуры, времени выдержки 
производятся в ручном режиме оператором термопечей. На основе данных  о затратах на разработку и
сопровождение, результатах прогнозирования объема продаж, определяем стоимость одного комплекта
программного обеспечения. Исходя из потребности на рынке подобных продуктов
можно предположить, что количество проданных копий составитоколо 300 штук. Стоимость выставляемого
на рынок ПП определяется частью стоимости разработки ПП, затрат на сопровождение и прибыли
организации-разработчика. Стоимость сопровождения остается постоянной для каждой установки
ПП, а частичная стоимость разработки, приходящаяся на каждый комплект ПП,
определяется исходя из данных о планируемом объеме установок.
\end{par}


Стоимость программного продукта можно рассчитать, используя соотношение:
\begin{equation}
	\textrm{Ц}_\textrm{пп} = (\Delta{}\textrm{С} + \textrm{С}_\textrm{сопр}) \times{} (1 + \textrm{Д}_\textrm{приб})
\end{equation}
\begin{equation}
	\Delta{}\textrm{С} = \frac{\textrm{С}_\textrm{пп}}{N} \times{} (1 + \textrm{Н}_\textrm{ст})
\end{equation}

\begin{ESKDexplanation}                       
	\item[где ] $\Delta{}\textrm{С}$ --- часть стоимости разработки приходящаяся на одну копию ПП;
	\item{} $\textrm{С}_\textrm{сопр}$ --- стоимость сопровождения ПП;
	\item{} $\textrm{Д}_\textrm{приб}$ --- процент прибыли, закладываемый в стоимость;
	\item{} $\textrm{Н}_\textrm{ст}$ --- Ставка банковского процента (при условии кредита).
\end{ESKDexplanation}

$ \Delta{} C = \frac{545163.25}{300} = 1817.21$ (руб.).
Стоимость одного комплекта ПП: $Цпп = (1817.21 +4178.78)*1.15  =  6895.39$ руб.
Сумма прибыли составит:
\begin{equation}
	\textrm{С}_\textrm{приб} = \textrm{С}_\textrm{пп} \times{} \textrm{Д}_\textrm{приб} \times{} (1 - \textrm{Н}_\textrm{ндс})
\end{equation}
подставим все значения и получим: $\textrm{С}_\textrm{приб} = 545163.25 \times{} 0.15 =  81774.49$ (руб.).



\subsection{Анализ конкурентноспособности и качества   разрабатываемой системы}
В рыночной экономике решающим фактором коммерческого успеха товара является конкурентноспособность.
Это многоаспектное понятие, означающее соответствие товара условиям рынка, конкретным
требованиям потребителей не только по своим качественным, техническим.
Экономическим, эстетическим характеристикам, но и по коммерческим условиям его реализации.
Данная система контроля технологического процесса конкурентноспособна, как и другие аналоги
на рынке программной продукции.

\subsubsection{Анализ технической  прогрессивности , разрабатываемого ПП}
Техническая прогрессивность  измеряемых показателей  характеризуется
коэффициентом технической прогрессивности. Расчет этого коэффициента 
осуществляется путем сравнения  технического уровня  товара-конкурента
и разрабатываемого по отношению к эталонному уровню  ПП  данного направления.
Расчет коэффициента технической прогрессивности производим по формуле:

\begin{equation}
	\textrm{К}_\textrm{тп} = \frac{\textrm{К}_\textrm{тн}}{\textrm{К}_\textrm{тб}}
\end{equation}
\begin{ESKDexplanation}
	\item[где ]{} $\textrm{К}_\textrm{тн}$ --- коэффициент технического уровня нового ПП;
	\item{} $\textrm{К}_\textrm{тб}$ --- коэффициент технического уровня базового ПП.
\end{ESKDexplanation}


\begin{equation}
	\textrm{К}_\textrm{т}\textrm{(бн)} = \sum_{i=1}^{I}\beta{}\frac{\textrm{П}_i\textrm{(бн)}}{\textrm{П}_\textrm{эi}}
\end{equation}
(при прямой зависимости)


\begin{equation}
	\textrm{К}_\textrm{т}\textrm{(бн)} = \sum_{i=1}^{I}\beta{}\frac{\textrm{П}_\textrm{э i}}{\textrm{П}_\textrm{i}\textrm{(бн)}}
\end{equation}
(при обратной зависимости)

\begin{ESKDexplanation}
	\item{} $\beta{}_i$ --- коэффициент весомости. Устанавливается экспертным путем;
	\item{} $\textrm{П}_i\textrm{(бн)}$ --- численное значение параметра соответственно базового и нового ПП.
\end{ESKDexplanation}


Анализируемый ПП технически прогрессивен, если $\textrm{К}_\textrm{тп}$ больше 1. Результаты
расчета сводим в таблицу (таб. \ref{table:calcKoefTechProgress}).
\begin{table}
\caption{Расчет коэффициента технической прогрессивности разрабатываемого программного продукта}
\begin{tabular}{|p{5cm}|p{1cm}|p{1cm}|p{1cm}|p{1cm}|p{1cm}|p{1cm}|p{1cm}|p{1cm}|}
\hline{}
Наимение параметра &
	$\beta $ &
	$\textrm{П}_\textrm{э}$ &
	$\textrm{П}_\textrm{б}$ &
	$\textrm{П}_\textrm{н}$ &
	$\frac{\textrm{П}_\textrm{н}}{\textrm{П}_\textrm{э}}$ &
	$\frac{\textrm{П}_\textrm{б}}{\textrm{П}_\textrm{э}}$ &
	$\beta \frac{\textrm{П}_\textrm{б}}{\textrm{П}_\textrm{э}}$ &
	$\beta \frac{\textrm{П}_\textrm{н}}{\textrm{П}_\textrm{э}}$ \\
\hline{}
Flash-память & 0.4 & 256 & 128 & 256 & 1 & 0.5 & 0.2 & 0.4 \\
\hline{}
Рабочее напряжение & 0.4 & 12В & 25В & 3.6В & 3& 0.5 & 0.2 & 132 \\
\hline{}
Количество линий ввода-вывода & 0.2 & 64 & 20 & 64 & 1 & 0.31 & 0.06 & 0.2 \\
\hline{}
Итого: & & & & & & 1 & 0.46 & 192 \\
\hline
\end{tabular}
\label{table:calcKoefTechProgress}
\end{table}               
Находим коэффициент технической прогрессивности: $\textrm{К}_\textrm{тп} = \frac{1.92}{0.46} = 4.1$.
Полученный коэффициент больше единицы, следовательно ПП технически прогрессивен.



\subsubsection{Анализ изменений функциональных возможностей разрабатываемого программного продукта}
\begin{par}
В этом разделе анализируются  эстетические, эргономические,
экологические параметры , характеризующие функциональные возможности 
разрабатываемого программного продукта, не имеющие количественного 
выражения, трудно поддающиеся непосредственной количественной оценке.
Коэффициент изменения функциональных возможностей рассчитывается
по  формуле:
\end{par}


\begin{equation}
	\textrm{К}_\textrm{фв} = \frac{\textrm{А}_\textrm{н}}{\textrm{А}_\textrm{б}}
\end{equation}
\begin{ESKDexplanation}
	\item[где ]{} $\textrm{А}_\textrm{н}$ и $\textrm{А}_\textrm{б}$ --- суммарная бальная оценка не измеряемых параметров соответственно разрабатываемого и базового ПП.
\end{ESKDexplanation}
Функциональные возможности нового ПП лучше чем у базового, если $\textrm{К}_\textrm{фВ}$  больше 1.
Результаты расчетов сводим в таблицу.

\begin{table}
\caption{Расчет коэффициента изменения функциональных возможностей разрабатываемого ПП}
\begin{tabular}{|p{3cm}|p{3cm}|p{3cm}|p{3cm}|p{3cm}|}
\hline{}
Не измеряемые параметры & Характе\-ристика параметра (базовый) & Характе\-ристика параметра (новый) & Бальная оценка (базовый) & Бальная оценка (новый) \\
\hline{}
1-wire протокол & Да & Да & 2 & 2 \\
\hline{}
Защита доступа & Нет & Да & 1 & 2 \\
\hline{}
Учет статистики & Нет & Да & 1 & 2 \\
\hline{}
Отражение графических обьектов & Нет & Да & 1 & 2 \\
\hline{}
Итого & & & 5 & 8 \\
\hline
\end{tabular}
\label{table:calcKoefTechProgress}
\end{table}

Таким образом: $\textrm{К}_\textrm{фв} = 1.6$  больше единицы, функциональные возможности нового ПП лучше чем базового.



\subsubsection{Анализ соответствия разрабатываемого ПП нормативным показателям}
Нормативные или, так называемые, регламентируемые параметры характеризуют соответствие разрабатываемого
ПП международным и национальным стандартам, нормативам, законодательным актам. Для оценки этого
показателя применяется единичный или групповой показатель- Кнорм. Единичный показатель может принимать
только одно значение: 1-соответствие и 0- несоответствие. Для разрабатываемого программного продукта Кнорм=1.


\subsubsection{Анализ экономических параметров программного продукта}
На данном этапе осуществляется анализ стоимостных параметров ПП, характеризующих его основные экономические свойства.
В ходе анализа рассчитывается коэффициент цены потребления по  формуле:
\begin{equation}Кцп = \frac{Цп^{нов}}{Цп^{баз}}\end{equation}

\begin{ESKDexplanation}
	\item[где ] $Цп^{нов}$, $Цп^{баз}$ --- цена потребления разрабатываемого и базового программного продукта.
\end{ESKDexplanation}

Цена потребления представляет собой затраты покупателя на приобретение, доработку.
А также эксплуатацию анализируемого программного продукта на протяжении периода эксплуатации.

\begin{equation}
	Цп = Цпр. + Рдор. + Иэкс. \times{} Тн
\end{equation}

\begin{ESKDexplanation}
	\item[где ]{} Цп --- цена потребления (руб.);
	\item{} Цпр. --- цена приобретения копии ПП покупателем (руб.);
	\item{} Иэкс. --- годовые эксплуатационные издержки потребителя (руб.);
	\item{} Тн --- нормативный срок эксплуатации (лет);
	\item{} Рдор. --- затраты па доработку;
\end{ESKDexplanation}

$Рдор.=Тдор.\times{}Здн$.

\begin{equation}
Тдор. = Крз \times{} Ксл. \times{} Куч. \times{} Нвр.др.
\end{equation}

\begin{ESKDexplanation}
	\item[где ]{} Здн. --- дневная заработная плата исполнителя (Здн.баз = 800 руб.; Здн.нов. = 360.24 руб);
	\item{} Тдор. --- время доработки. (Тдор.баз = 9.6 дн;   Тдор.нов. = 8 дн.);
	\item{} Крз --- коэффициент, учитывающий язык программирования (Крз.баз. = 1.0;  Крз.нов. = 1.0);
	\item{} Ксл --- коэффициент сложности (Ксл. Баз =1.1;  Ксл.нов. = 1.21);
	\item{} Куч. --- коэффициент участия (Куч.баз =1.1 ;Куч.нов. =1.1 );
	\item{} Нвр.дор. --- норма времени на доработку (Нвр.дор.баз. = 9.6 дней; Нвр.дор.нов. = 8 дней);
\end{ESKDexplanation}

Эксплуатационные издержки за весь период эксплуатации:
\begin{equation}
	Иэкс =  Рперс +  Рнакл +  Рпр
\end{equation}

\begin{equation}
	Рперс = sum{} (n_i \times{} Огод \times{} (1 + \frac{Нотч}{100}) \times{} (1 + \frac{П}{100}))
\end{equation}

\begin{equation}
	Рперс = \sum{}(n_i \times{} Огод \times (1 + \frac{Нотч}{100}) \times{} (1 + \frac{П}{100}))
\end{equation}

\begin{ESKDexplanation}
	\item[где ]{} n --- численность персонала;
	\item{} Огод --- среднегодовая заработная плата одного работника;
	\item{} Нотч. --- процент отчисления на ЕСН;
	\item{} П --- процент премии;
	\item{} Ннакл. --- накладные расходы;
	\item{} Рпр. --- прочие расходы ( 1\% - 3\%  от суммы всех эксплуатационных расходов);
\end{ESKDexplanation}

\begin{par}
Все расчеты сводим в таблицы.
\end{par}

\begin{par}
\begin{table}
\caption{Расчет годовых эксплуатационных издержек потребителя ПП}
\begin{tabular}{|l|p{5cm}|p{3cm}|}
\hline{}
Наименование расходов & Базовый & Новый \\
\hline{}
Расходы на содержание персонала & 567241.0 & 435948.9\\
\hline{}
Накладные расходы & 587241.0 & 523138.68\\
\hline{}
Прочие расходы & 11744.82 & 9590.86 \\
\hline{}
Итого & 1186196.82 & 968678.44 \\
\hline
\end{tabular}
\label{table:yearEkspIzd}
\end{table}
\end{par}


\begin{par}
\begin{table}
\caption{Расчет цены потребления программного продукта}
\begin{tabular}{|p{9cm}|p{4cm}|p{3cm}|}
\hline{}
Наименование расходов &  Базовый ПП & Новый ПП \\
\hline{}
Продажная цена ПП &  8791.0 & 6895.39 руб. \\
\hline{}
Расходы на доработку & 9292.8 & 3835.84 \\
\hline{}
Эксплуатационные издержки потребителя за весь период эксплуатации & 9489574.56 & 7749427.52 \\
\hline{}
Цена потребления & 9597658.36 & 7759259.35\\
\hline
\end{tabular}
\label{table:zhenaPotreblenija}
\end{table}
\end{par}

Исходя из полученных результатов определяем коэффициент цены потребления: $Кцп.=0.816$ меньше единицы,
следовательно экономические параметры разрабатываемого ПП лучше базовых.

\subsection{Оценка конкурентноспособности программного продукта}
В целом конкурентноспособность нового ПП по отношению к базовому можно оценить с помощью интегрального
коэффициента конкурентноспособности, учитывающего все ранее рассчитанные параметры.

\begin{equation}
Ки = Ктп \times{} Кфв \times{} Кнорм \times{} \frac{1}{Кцп}
\end{equation}

Подставляем рассчитанные значения коэффициентов: $Ки = 4.1 \times{} 1.6 \times{} 1.0 \times{} \frac{1}{0.816} = 8.01$.
Полученное значение больше единицы, значит ПП конкурентноспособен.

\subsubsection{Анализ технико - экономических показателей разработки и эксплуатации ПП}
\begin{par}
Ранее рассчитанные показатели сводим в таблицу.
\begin{table}
\caption{Технико-экономические показатели разработки и эксплуатации  ПП}
\begin{tabular}{|l|r|}
\hline{}
Показатели & Значение (руб.) \\
\hline{}
Затраты на разработку & 545163.25 \\
\hline{}
Затраты на сопровождение & 4178.78 \\
\hline{}
Продажная цена & 6895.39 \\
\hline{}
Годовые эксплуатационные издержки потребителя & 968678.0 \\
\hline{}
Цена потребления & 775259.35 \\
\hline{}
Интегральный коэффициент   конкурентноспособности ПП & 8.01 \\
\hline{}
Коэффициент технической прогрессивности ПП & 4.1 \\
\hline{}
Коэффициент функциональных возможностей ПП & 1.6 \\
\hline
\end{tabular}
\label{table:econSostavIspol}
\end{table}
\end{par}

\subsection{Выводы}
Произведенные расчеты показали, что рассматриваемый проект по сравнению с базовым имеет ряд преимуществ,
которые могут позволить ему занять достойное место на рынке контрольно-измерительных систем.
\newpage{}
