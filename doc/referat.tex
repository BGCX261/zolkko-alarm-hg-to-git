%!TEX root = /Users/zolkko/Projects/zolkko-alarm/doc/main.tex
\begin{center}\MakeUppercase{Реферат}\end{center}
\addcontentsline{toc}{section}{Реферат}

Пояснительная записка \ESKDtotal{page} с., \ESKDtotal{figure} рисунков, \ESKDtotal{table} таблиц,
\ESKDtotal{bibitem} источников.


Ключевые слова: AVR, XMega, автоматизированная разработка.


Объект разработки -- программно-аппаратный комплекс <<Универсальная система терморегулирования
 на базе микроконтроллера AVR семейства XMega>>.

Цель работы -- исследование методик и средств разработки про\-гра\-ммно\--аппа\-ра\-тных комплексов на базе
микроконтроллеров AVR семейства XMega компании Atmel.

Методы исследования и аппаратура -- для проектирования электрической принципиальной и
топологической схемы управляющего устройства применяется среда Eagle Lite. Разработка
микропрограммы управляющего устройства, программы управляющего сервиса и веб-сервиса осуществляетя
на языках программирования C++ и ErLang соответственно; в качестве среды разработки используется
редактор MVim 7.3, компилятор AVR-GCC 4.3.3 и ErLang 5.8.2.

Полученные результаты и их новизна -- спроектировано устройство терморегуляции; разработано
	программное обеспечение микроконтроллера устройства терморегуляции, использующее большую
	часть периферии микроконтроллера atxmega128a3; разработано
	программное обеспечение контролирующего сервиса; осуществлено сопряжение 8 разрядного
	микроконтроллера с внешним программно-аппаратным обеспечение по сети Ethernet 802.3.


Основные технико-эксплуатационные характеристики -- для управляющего устройства необходимо использовать
микроконтроллер не ниже A3 версии и операционные усилители AD8544 или AD8554.

Степень внедрения -- находиться в стадии разработки, модель терморегулирования требует уточнения.

Область применения -- результаты проведённой работы восполняют недостаточное количество информации
о применимости микроконтроллеров AVR семейства XMega к разработке программно-аппаратных микроконтроллерных
комплексов и могут представлять интерес для специалистов в
области микроэлектроники, студентов, руководителей детских кружков радио-любителей.

\newpage{}









