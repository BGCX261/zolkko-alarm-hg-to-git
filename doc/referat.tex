%!TEX root = /Users/zolkko/Projects/zolkko-alarm/doc/main.tex
\begin{center}\MakeUppercase{Реферат}\end{center}
\addcontentsline{toc}{section}{Реферат}

Пояснительная записка \ESKDtotal{page} с., \ESKDtotal{figure} рисунков, \ESKDtotal{table} таблиц,
\ESKDtotal{bibitem} источников.


Ключевые слова: AVR, XMega, автоматизированная разработка.


Объект разработки: программно-аппаратный комплекс <<Универсальная система терморегулирования
 на базе микроконтроллера AVR семейства XMega>>, на примере технологического процесса горячего копчения
колбасной продукции на малом предприятии.

Цель работы: исследование методик и средств разработки программно-аппаратных комплексов на базе
микроконтроллеров AVR семейства XMega компании Atmel. Созданная система должна:



Методы исследования и аппаратура: для проектирования электрической принципиальной и
топологической схемы управляющего устройства применяется среда Eagle Lite. Разработка
микропрограммы управляющего устройства, программы управляющего сервиса и веб-сервиса применяются
языки программирования C++ и ErLang соответственно; в качестве среды разработки используется
редактор MVim 7.3.


Основные технико-эксплуатационные характеристики: для управляющего устройства необходимо использовать
микроконтроллер не ниже A3 версии и операционные уселители AD8544 или AD8554.


\newpage{}









