%!TEX root = /Users/zolkko/Projects/zolkko-alarm/doc/main.tex
\begin{center}\MakeUppercase{Реферат}\end{center}
\addcontentsline{toc}{section}{Реферат}

Пояснительная записка \ESKDtotal{page} с., \ESKDtotal{figure} рисунков, \ESKDtotal{table} таблиц,
\ESKDtotal{bibitem} источников.


Ключевые слова: AVR, XMega, автоматизированная разработка.


Объект разработки: программно-аппаратный комплекс <<Универсальная система терморегулирования
 на базе микроконтроллера AVR семейства XMega>>.

Цель работы: исследование методик и средств разработки про\-гра\-ммно\--аппа\-ра\-тных комплексов на базе
микроконтроллеров AVR семейства XMega компании Atmel.

Созданная система должна состоять из управляющего устройства и его программного обеспечения,
контролирующего сервиса и веб-интерфейса пользователя. Управляющее устройство должно обеспечивать
вывод информации о текущей выполняемой задачи на жидкокристаллический дисплей с интерфейсом i80 на
базе микросхемы ili9320;
собирать показания термопар Т типа; производить компенсацию холодного спая термопар при помощи цифрового датчика
общего назначения DS18B20; управлять электронагревательным механизмом для вывода температуры
камеры на заданное значение, используя методику широтно-импульсной модуляции; оповещать контролирующий
сервис о текущих показаниях датчиков и текущем состоянии системы через сеть Ethernet 802.3, используя
микросхему enc28j60 компании Microchip. Контролирующая подсистема должна обеспечивать приём показаний
датчиков управляющего устройства и их надёжное хранение в документ-ориентированной распределённой базе
данных Mnesia. Интерфейс пользователя, предоставляемый веб-сервисом, должен выдавать график отражающий
показания датчиков, предоставлять возможность влиять на текущее состояние управляющего устройства.



Методы исследования и аппаратура: для проектирования электрической принципиальной и
топологической схемы управляющего устройства применяется среда Eagle Lite. Разработка
микропрограммы управляющего устройства, программы управляющего сервиса и веб-сервиса осуществляетя
на языках программирования C++ и ErLang соответственно; в качестве среды разработки используется
редактор MVim 7.3, компилятор AVR-GCC 4.3.3 и ErLang 5.8.2.


Основные технико-эксплуатационные характеристики: для управляющего устройства необходимо использовать
микроконтроллер не ниже A3 версии и операционные усилители AD8544 или AD8554.


\begin{par}
Значимость работы заключается в том, что она заполняет пробел в документации и примерах использования
микроконтроллеров AVR семейства XMega. Так как, не смотря на то, что эти микроконтроллеры, первые версии
которых были выпущены в 2009 году, предлагают разработчикам в значительной степени более совершенный
инструмент для реализации своих идей, на фоне не значительного повышения в цене устройства, количество
статей в журналах, обзоров в сети Интернет, демонстрационных проектов и книг на русском языке с
примерами их применения ничтожно мало.
\end{par}

\begin{par}
Пример реализации устройства, выполняемый с применением исключительно свободного и бесплатного
программного обеспечения и доступный любому заинтересованному лицу по адресу http://zolkko-alarm.google.com
по свободной лицензии, может прояснить нюансы использования микроконтроллеров AVR семейства
XMega в реальных устройства и применения языка высокого уровня C++ для написания программного
обеспечения 8 разрядных микроконтроллеров.
\end{par}

\begin{par}
Таким образом эта тема может представлять интерес как для специалистов в области микроэлектроники,
так и для студентов, руководителей детских кружков радио-любителей.
\end{par}

\begin{par}
Так же, разрабатываемое устройство, являться не только демонстрационным стендом возможностей микроконтроллера AVR,
но может использоваться в самостоятельно в качестве системы терморегуляции, коптильных камер, печей.
\end{par}
\newpage{}









