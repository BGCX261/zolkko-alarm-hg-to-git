%!TEX root = /Users/zolkko/Projects/zolkko-alarm/doc/main.tex
\begin{center}
\noindent{Министерство образования и науки Российской Федерации}
\end{center}

\textblockorigin{0mm}{0mm}
\setlength{\TPHorizModule}{1mm}
\setlength{\TPVertModule}{\TPHorizModule}

\begin{textblock}{210}(0, 60)
	\noindent{\makebox[210mm][c]{\scriptsize{(Факультет)}}}
\end{textblock}

\begin{center}
\MakeUppercase{Государственное образовательное учреждение} \\*
\MakeUppercase{Высшего профессионального образования} \\*	
<<\mbox{\MakeUppercase{Воронежский~государственный~технический~университет}}>>\\*
\underline{\makebox[170mm][c]{Факультет автоматики и электромеханики}} \\*
\begin{flushleft}
\noindent{Кафедра \underline{\makebox[154mm][l]{<<Автоматизированных и вычислительных систем>>}} \\*
Специальность \underline{\makebox[140mm][l]{230101 <<Вычислительные машины, комплексы, системы и сети>>}}}
\end{flushleft}
\end{center}

\vfill{}

\begin{center}
\MakeUppercase{Выпускная квалификационная работа}
\end{center}

\vfill{}

\begin{flushleft}
Тема дипломного проекта: \underline{<<Автоматизированная разработка программно-}
\underline{-аппаратного комплекса
на базе микроконтроллера AVR семейства XMega>>}
\end{flushleft}

\vfill{}

\begin{center}
Пояснительная записка
\end{center}

\begin{par}
	\vspace{100mm}
	\textblockorigin{0mm}{160mm}
	\setlength{\TPHorizModule}{1mm}
	\setlength{\TPVertModule}{\TPHorizModule}
	
	\begin{textblock}{210}(20, 0)
		\noindent{\makebox[65mm][l]{Разработал}\underline{\makebox[110mm][r]{А.Н. Анисимов}}}
	\end{textblock}
	\begin{textblock}{210}(85, 5)
		\noindent{\makebox[75mm][l]{\scriptsize{Дата, подпись}}\makebox[40mm][l]{\scriptsize{Инициалы, фамилия}}}
	\end{textblock}
	
	\begin{textblock}{210}(20, 10)
		\noindent{\makebox[65mm][l]{Зав. кафедрой}\underline{\makebox[110mm][r]{С.Л. Подвальный}}}
	\end{textblock}
	\begin{textblock}{210}(85, 16)
		\noindent{\makebox[75mm][l]{\scriptsize{Дата, подпись}}\makebox[40mm][l]{\scriptsize{Инициалы, фамилия}}}
	\end{textblock}
	
	\begin{textblock}{210}(20, 20)
		\noindent{\makebox[65mm][l]{Руководитель}\underline{\makebox[110mm][r]{В.Ф. Барабанов}}}
	\end{textblock}
	\begin{textblock}{210}(85, 26)
		\noindent{\makebox[75mm][l]{\scriptsize{Дата, подпись}}\makebox[40mm][l]{\scriptsize{Инициалы, фамилия}}}
	\end{textblock}
	
	\begin{textblock}{210}(20, 30)
		\noindent{\makebox[65mm][l]{Консультанты}\underline{\makebox[110mm][r]{Т.С. Наролина}}}
	\end{textblock}
	\begin{textblock}{210}(85, 36)
		\noindent{\makebox[75mm][l]{\scriptsize{Дата, подпись}}\makebox[40mm][l]{\scriptsize{Инициалы, фамилия}}}
	\end{textblock}
	
	\begin{textblock}{210}(20, 40)
		\noindent{\underline{\makebox[175mm][r]{В.П. Асташкин}}}
	\end{textblock}
	\begin{textblock}{210}(85, 45)
		\noindent{\makebox[75mm][l]{\scriptsize{Дата, подпись}}\makebox[40mm][l]{\scriptsize{Инициалы, фамилия}}}
	\end{textblock}
	
	\begin{textblock}{210}(20, 52)
		\noindent{\underline{\makebox[175mm][r]{~}}}
	\end{textblock}
	\begin{textblock}{210}(85, 54)
		\noindent{\makebox[75mm][l]{\scriptsize{Дата, подпись}}\makebox[40mm][l]{\scriptsize{Инициалы, фамилия}}}
	\end{textblock}
	
	\begin{textblock}{210}(20, 62)
		\noindent{\underline{\makebox[175mm][r]{~}}}
	\end{textblock}
	\begin{textblock}{210}(85, 64)
		\noindent{\makebox[75mm][l]{\scriptsize{Дата, подпись}}\makebox[40mm][l]{\scriptsize{Инициалы, фамилия}}}
	\end{textblock}
	
	\begin{textblock}{210}(20, 72)
		\noindent{\underline{\makebox[175mm][r]{~}}}
	\end{textblock}
	\begin{textblock}{210}(85, 74)
		\noindent{\makebox[75mm][l]{\scriptsize{Дата, подпись}}\makebox[40mm][l]{\scriptsize{Инициалы, фамилия}}}
	\end{textblock}
	
	\begin{textblock}{210}(20, 80)
		\noindent{\makebox[65mm][l]{Нормоконтроль проверил(а)}\underline{\makebox[110mm][r]{Т.И. Сергеева}}}
	\end{textblock}
	\begin{textblock}{210}(85, 86)
		\noindent{\makebox[75mm][l]{\scriptsize{Дата, подпись}}\makebox[40mm][l]{\scriptsize{Инициалы, фамилия}}}
	\end{textblock}
\end{par}

\vfill{}

\begin{center}Воронеж 2011\end{center}
\newpage{}

\textblockorigin{0mm}{0mm}
\setlength{\TPHorizModule}{1mm}
\setlength{\TPVertModule}{\TPHorizModule}

\begin{textblock}{210}(0, 50)
	\noindent{\makebox[210mm][c]{\scriptsize{(Факультет)}}}
\end{textblock}

\begin{textblock}{210}(0, 77)
	\noindent{\makebox[210mm][c]{\scriptsize{(индекс группы)}}}
\end{textblock}

\begin{textblock}{210}(0, 85)
	\noindent{\makebox[210mm][c]{\scriptsize{(Фамилия, имя, отчество)}}}
\end{textblock}

\noindent{\begin{center}
	\MakeUppercase{Государственное образовательное учреждение} \\
	\MakeUppercase{Высшего профессионального образования} \\
	<<\MakeUppercase{Воронежский~государственный~технический~университет}>> \\
	\underline{\makebox[170mm][c]{Факультет автоматики и электромеханики}} \\
	\begin{flushleft}
		\noindent{Кафедра \underline{\makebox[154mm][l]{<<Автоматизированных и вычислительных систем>>}} \\
		Специальность \underline{\makebox[140mm][l]{230101 <<Вычислительные машины, комплексы, системы и сети>>}}\\
		Студент группы \underline{\makebox[138mm][c]{ВМ-072}}}
		\underline{\makebox[175mm][c]{Анисимов Алексей Николаевич}}
	\end{flushleft}
	
\end{center}}

\vspace{1.5em}

\begin{center}
	\MakeUppercase{Задание}
\end{center}

\begin{flushleft}
1. Тема дипломной работы: \underline{<<Автоматизированная разработка программно-}\\
\underline{-аппаратного комплекса на базе микроконтроллера AVR семейства XMega>>} \\
утверждена распоряжением по факультету № \underline{\makebox[12mm]{}} от \underline{\makebox[50mm]{}} г.\\
\end{flushleft}

\begin{flushleft}
2. Технические условия: Для разработки проекта системы необходим IBM совместимый компьютер
, работающий под управлением Ubuntu 11.04, процессор Intel Core 2 Duo с тактовой
частотой 1.8 МГц, ОЗУ не менее 1024 МБ, не менее 20 Гб свободного дискового пространства,
устройства ввода и вывода информации.
\end{flushleft}

\noindent{\begin{flushleft}
% \noindent{\underline{\makebox[170mm][l]{    }}}
3. Содержание (разделы, графические работы, расчёты и проч.) \\
Разделы: 1)  Анализ современных систем автоматизированного проектирования программно-аппаратных
микроконтроллерных комплексов; 2) Выбор инструментальных средств разработки; 3) Реализация
программно-аппаратного универсальной системы терморегулирования на базе микроконтроллера AVR
семейства XMega; 4) Организационно-экономическая часть; 5) Безопасность и экологичность в
дипломном проекте.
Графичекие работы: 1) Структурная схема универсальной системы терморегулирования на базе
микроконтроллера AVR семейства XMega; 2) Принципиальная электрическая схема устройства управления;
3) Топологическая схема устройства управления; 4) Типовая динамика системы;
5) Архитектура контролирующего сервиса; 6) 7)Структура классов микропрограммы управляющего устройства 8) Технико-экономические показатели порграммно-аппаратного комплекса <<Универсальная система
терморегуляции на базе микроконтроллера AVR семейства XMega>>.
\end{flushleft}}

\vfill{}

\newpage{}

4. План выполнения дипломной работы
с << >> г. по << >>  г.

\begin{tabular}{ccccc}
\hline
Название элементов проектной работы & \% & Сроки & \% \\ выполн. & Подпись рук. консул. \\
\hline
r2c1 & r2c2 & r2c3 & r2c4 & r2c5\\
\hline
r3c1 & r3c2 & r3c3 & r3c4 & r3c5\\
\hline
r4c1 & r4c2 & r4c3 & r4c4 & r4c5\\
\hline
r5c1 & r5c2 & r5c3 & r5c4 & r5c5\\
\hline
r6c1 & r6c2 & r6c3 & r6c4 & r6c5\\
\hline
\end{tabular}

Руководитель дипломной работы
			Барабанов Владимир Федорович
(подпись) (фамилия, имя, отчество)

5. Дипломная работа закончена
<< >> г.
(подпись дипломника)
6. Пояснительная записка и все материалы просмотрены.
Консультанты \\

\begin{par}
	~\\
7. Допутить дипломника \rule{\linewidth}{0.5mm} \\
к защите дипломного проекта, дипломной работы в Государственной аттестационной
комисcии (протокол заседания кафедры №~~~~~ от << >> г.)
\end{par}




