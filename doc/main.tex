%%
%% Тема практики: <<Автоматизированная разработка программно-аппаратного комплекса
%% ''цифровые часы с будильником'' на базе микроконтроллера AVR семейства XMega>>
%%
\documentclass[russian,simple,utf8,pointsubsection]{eskdtext}
\usepackage[unicode]{hyperref}
\usepackage{multirow}
\usepackage{amstext}
\usepackage{amsmath}
\usepackage{listings}
\graphicspath{{/Users/zolkko/Projects/zolkko-alarm/doc/imgs/}}
\usepackage{longtable}

\ESKDdepartment{}
\ESKDcompany{}
\ESKDclassCode{31 1398}
\ESKDdocName{Пояснительная записка}
\ESKDsignature{Автоматизированная разработка программно-аппаратного комплекса <<цифровые часы с будильником>> на базе микроконтроллера AVR семейства XMega}
\ESKDtitle{Отчёт по преддипломной практике}
\ESKDauthor{Анисимов~А.~Н.}
\ESKDtitleApprovedBy{Руководитель}{Барабанов~В.~Ф.}
\ESKDtitleAgreedBy{Руководитель}{Барабанов~В.~Ф.}
\ESKDtitleDesignedBy{студент группы ВМ-072}{Анисимов А.Н.}
\ESKDdefaultTitleStyle{empty}
\ESKDdefaultFirstStyle{empty}
%% Уменьшаю размер шрифта колонки №1 основной надписи, так что бы эта
%% "умная" дура всё таки влезла в рамку
\renewcommand{\ESKDfontVIIsize}{\fontsize{10pt}{12pt}}
%% пол умалчанию перечисления используют арабские цифры
\renewcommand{\theenumi}{\arabic{enumi}}


\begin{document}
\begin{center}
Федеральное агентство по образованию \\*
ГОСУДАРСТВЕНОЕ ОБРАЗОВАТЕЛЬНОЕ УЧРЕЖДЕНИЕ \\*
ВЫСШЕГО ПРОФЕССИОНАЛЬНОГО ОБРАЗОВАНИЯ \\*
<<ВОРОНЕЖСКИЙ ГОСУДАРСТВЕННЫЙ ТЕХНИЧЕСКИЙ УНИВЕРСИТЕТ>> \\
\end{center}
\begin{center}
Факультет  автоматики и электромеханики \\
\end{center}
\begin{center}
Кафедра автоматизированных и вычислительных систем \\
\end{center}

\begin{center}
Специальность 230101 <<Вычислительные машины, комплексы, системы и сети>>
\end{center}

\vspace{1em}

\begin{center}
ОТЧЁТ \\*
ПО ПРЕДДИПЛОМНОЙ ПРАКТИКЕ \\*
Тема практики: <<Автоматизированная разработка программно-аппаратного комплекса
''цифровые часы с будильником'' на базе микроконтроллера AVR семейства XMega>>
\end{center}

\vspace{1em}

\begin{tabular}{p{12em}p{5em}@{}p{5em}@{}r@{}r}
Выполнил студент группы & \hrulefill{} ВМ-072 & \hrulefill{}  & \hrulefill{} А.Н. Анисимов \\
 &  \small{Группа}  & \small{Подпись} &\small{Инициалы, фамилия} \\
& & & \\
\multirow{2}{12em}{Руководитель практики\newline{} от кафедры АВС} & \hrulefill{} & \hrulefill{} & \hrulefill{} О. Я. Кравец \\
 & \small{Подпись}  & \small{Дата} & \small{Инициалы, фамилия} \\
 & & & \\
\multirow{2}{12em}{Руководитель дипломного  проекта} & \hrulefill{} & \hrulefill{} & \hrulefill{} А. В. Барабанов \\
 & \small{Подпись}  & \small{Дата} & \small{Инициалы, фамилия} \\
& & & \\
\multirow{2}{12em}{Консультант дипломного проекта} & \hrulefill{} & \hrulefill{} & \hrulefill{} В.Ф. Барабанов \\
 & \small{Подпись}  & \small{Дата} & \small{Инициалы, фамилия} \\
& & & \\
\multirow{2}{12em}{Руководитель по экономической части} & \hrulefill{} & \hrulefill{} & \hrulefill{} Т.С. Наролина \\
 & \small{Подпись}  & \small{Дата} & \small{Инициалы, фамилия} \\
& & & \\
Защищена \hrulefill{} &  & Оценка \hrulefill{} & \hrulefill{} 
\end{tabular}

\vspace{1.5em}

\begin{center}
г. Воронеж \\*
2011 г.
\end{center}

\newpage

\begin{center}ЗАМЕЧАНИЯ РУКОВОДИТЕЛЯ\end{center}
\newpage

\tableofcontents
\newpage

%% Введение
\section*{ВВЕДЕНИЕ}
\addcontentsline{toc}{section}{ВВЕДЕНИЕ}
\begin{par}
На сегодняшний день при проектировании систем промышленной автоматизации и устройств
бытового применения, перед проектировщиками и разработчиками встают вопросы не только технического
характера, но и вопросы экономической целесообразности применения тех или иных решений.
То есть при их решении необходимо учитывать не только системные характеристики применяемых для
реализации конечных устройств технологий, но и искать компромис с их стоимостью. При этом
на конечную стоимость изделия будут влиять цена применённых схемотехнических решений,
время затраченное на проектирование и реализацию устройства, цена применяемых средства автоматизации
и цена специалистов проектировщиков и разработчиков.
\end{par}

\begin{par}
Немаловажным при проектировании устройств является учёт стремления
современной европейской культуры не только к открытым, но и полностью свободным системам,
зачастую обладающим более качественными системными и
потребительскими характеристиками и способствующими общему\\*
научно-техническому прогрессу\cite{lessing}.
\end{par}
\newpage{}



%% Постановка задачи
%%
%% Something is gioing wrong. Dispite sections was defined as
%% \MakeUppercase they still remains as normal text.
%% However, \MakeUppercase works for english characters.
%%
%% So I'm gonna use this sections like this:
%%
%% \section[Постановка задачи]{ПОСТАНОВКА ЗАДАЧИ}
\section{ПОСТАНОВКА ЗАДАЧИ}
\begin{par}
Спроектировать и разработать устройство позволяющее в заданные моменты времени выдавать сигнал
звукового оповещения, производить индикацию текущего времени, а так же прозводить его синхронизацию
с удалённым сервером используя вычислительную сеть стандарта 802.3 Ethernet.
\end{par}

\begin{par}
Центральный вычислительный модуль устройства должен быть выполнен на микроконтроллере AVR компании
Atmel семейства XMega.
\end{par}

\begin{par}
Не смотря на то, что реализация подобного устройства на ПЛИС имеет ряд преимуществ по
сравнению с реализацией на основе микроконтроллеров, а именно более высокая скорость
работы и более низкое потребление энергии, новое семейство 8-битных микроконтроллеров
AVR переносит микроконтроллерные устройства на новый уровень системных характеристик,
которые в сочетации с более низкой стоимостью самих микроконтроллеров и иструментария
разработчика, делает их адекватным выбором при проектировании систем. Микроконтроллеры
AVR семейства XMega интегрируют в себе устройства ввода-вывода с улучшенными характеристиками,
более низкую потребляемую мощьность за счёт применения в них новой технологии picoPower.
Новая система обработки событий Event System, обеспечивает независимую от ЦПУ
быстродействующую передачу данных между внутренними переферийными устройствами 4-канального
контроллера ПДП, улучьшающего характеристики микроконтроллера. Микроконтроллеры имеют
быстродействующие 12-битнее модули АЦП и ЦАП. И ускоритель криптографических алгоритмов AES и DES.
Каждое перечисленное нововведение в семействе XMega микроконтроллеров AVR, позволит
в более сжатые сроки, а следовательно и с меньшими финансовыми затратами реализовать
целевое устройство, при этом отставание в производительности и потребляемой мощьности
становиться не значительным для данного класса устройств.
\end{par}

\begin{par}
Необходимо в качестве устройств ввода и вывода использовать один модуль сенсорного экрана
для уменьшения рабочей поверхности изделия.
\end{par}

\begin{par}
В качестве устройства взаимодействия с удалённым сетевым сервисом времени необходимо использовать
контроллер фирмы Micro Chip enc28j60. Применени этого контроллера позволит устройству производить
синхронизацию текущего времени по протоколу SNTP.
\end{par}

\begin{par}
Необходимо споектировать и разработать:
    \begin{itemize}
        \item{}Принципиальную электрическую схему;
        \item{}печатную плату.
    \end{itemize}
\end{par}

\begin{par}
Необходимо разработать на языке высокого уровня С++  программу управления устройва:
\begin{itemize}
    \item{} Модуль учёта текущего времени;
    \item{} модуль индикации текущего времени, выводящий информацию о текущем времени на ЖК-дисплее;
    \item{} модуль пользовательского ввода с сенсорного экрана, обеспечивающий первоначальную
            конфигурацию устройства, и установку времени срабатывания звукового сигнала;
    \item{} модуль воспроизведения звукового сигнала с внешней энергонезависимой памяти;
    \item{} модуль синхронизации текущего времени с удалённым SNTP сервисом.
\end{itemize}
\end{par}

\subsection{Актуальность задачи}
\begin{par}
Не смотря на то, что микроконтроллеры AVR серии XMega, первые версии которых были выпущены в 2009 году,
предлагают разработчикам в значительной степени более совершенный инструмент для
реализации своих идей, на фоне не значительного повышения в цене устройства, количество статей в журналах,
обзоров в сети Интернет, демонстрационных проектов и книг на русском языке с примерами их применения
ничтожно мало.
\end{par}

\begin{par}
Пример реализации устройства, выполненяемый с применением исключительно свободного и бесплатного
программного обеспечения и доступный любому зантересованному лицу по адресу http://zolkko-alarm.google.com
по свободной лицензии, может прояснить нюансы использования микроконтроллеров AVR серии
XMega в реальных устройства и применения языка высокого уровня C++ для написания программного
обеспечения 8-битных микроконтроллеров.
\end{par}

\begin{par}
Таким образом эта тема может представлять интерес как для специалистов в области микроэлектронники,
так и для студентов, руководителей детских кружков радио-любителей.
\end{par}

\subsubsection{Область пименения устройства отсчёта и }
Разрабатываемое устройство, может являться не только демонстрационным стендом возможностей микроконтроллера AVR,
но может использоваться в самостоятельно.
%% TODO: написать каким образом может быть использовано это устройство.

\newpage{}


%% Обзор САПР
\section{Обзор САПР}
\subsection{Общие сведения о САПР}
\begin{par}
САПР --- Система автоматизированного проектирования --- автоматизированная система, реализующая
информационную технологию выполнения функций проектирования, представляет собой
организационно-техническую систему, предназначенную для автоматизации процесса проектирования,
состоящую из персонала и комплекса технических, программных и других средств
автоматизации его деятельности.
\end{par}

\begin{par}
На некотором этапе своего развития системы проектирования претерпели качественное изменение.
Оно было связано с тем, что САПР из набора каким-то образом связанных между собой прикладных программ начали превращаться в мобильные и стройно организованные системы, способные к настройке на особенности предметной области и на требования конечного пользователя, допускающие расширение функциональных возможностей за счёт сравнительно несложного подключения новых прикладных программных модулей и обеспечивающих поддержку групповой разработки сложных схем коллективом проектировщиков.
\end{par}

\begin{par}
Особенное значение САПР приобрели в микроэлектронике, поскольку современная радиоэлектронная аппаратура базируется на применении сверхбольших интегральных схем, разработка которых без применения САПР невозможна или крайне затруднительна.
\end{par}

\begin{par}
Специализированные САПР для разработки электронных устройств и печатных плат получили название Electronic Design Automation - EDA, автоматизация проектирования электронных приборов. Комплексы такого типа зачастую позволяют создавать принципиальные электрические схемы схемы с помощью графического интерфейса, создавать и модифицировать  базу радиоэлектронных компонентов, проверять целостность сигналов на ней и проводить аналоговое и цифровое моделирование разрабатываемого устройства ещё на этапе проектирования.
\end{par}

\begin{par}
Ниже перечислены примеры EDA программ:
	\begin{itemize}
		\item{}P-CAD
		\item{}OrCAD
		\item{}Electric
		\item{}Proteus
		\item{}KiCad
		\item{}gEDA
		\item{}Eagle EDA
	\end{itemize}
\end{par}


\subsection{Обзор Eagle EDA}
\begin{par}
EAGLE это лёгкий в использовании, но достаточной мощный EDA пакет, разрабатываемый немецкой компанией CadSoft.
В состам системы входят:
\begin{enumerate}
	\item{}Layout Editor - пограмма для проектирования печатных плат. 
		\begin{itemize}
			\item{}Максимальная рабочая поверхность 1.6 x 1.6м;
			\item{}разрешение - 1/10,000мм (0.1 микрон);
			\item{}до шестнадцати сигнальных слоёв;
			\item{}большая библиотека компонентов;
			\item{}copper pouring - области на печатной плате заполненные  медью, что часто используется для создания областей <<земли>> и уменьшения необходимого травильного вещества при производстве;
			\item{}встроенная ДРЦ проверка.
		\end{itemize}
	
	\item{}Schematic Editor - программа для создания принципиальных электрических схем.
		\begin{itemize}
			\item{}до 999 листов на одну схему;
			\item{}проверка эллектичских правил;
			\item{}обмен вентелей и пинов;
			\item{}возможность создания печатной платы из схемы одной командой.
		\end{itemize}
	\item{}Autorouter - программа автоматической трассировки.
		\begin{itemize}
			\item{}ripup-and-retry трассировщик - на первом проходе выполняется соединение абсолютно всех проводников без обращения внимания на возможные конфликты, заключающиеся в пересечении проводников на одном слое и нарушении зазоров. На каждом последующем проходе автотрассировщик пытается уменьшить количество конфликтов, разрывая и вновь прокладывая связи;
			\item{}до шестнадцати сигнальных слоёв;
			\item{}стратегия трассировки может быть подстроена заданием весовых коэфициентов.
		\end{itemize}
\end{enumerate}
Все эти программы встроены в один пользовательский интерфейс, таким образом нет необходимости в предварительном конвертировании нет-листов.
\end{par}


%% Описываем выбор инструментария
%!TEX root = /Users/zolkko/Projects/zolkko-alarm/doc/main.tex
\section{Выбор инструментальных средств разработки}

\subsection{Выбор микроконтроллера}
Основными требованиями предъявляемыми мной к центральному вычислительному устройству
создаваемого устройства является:
\begin{itemize}
	\item{} невысокая стоимость;
	\item{} низкое энергопотребление;
	\item{} в число поддерживаемой переферии должен присутствовать ЦАП;
	\item{} аппаратная поддержка SPI;
	\item{} поддержка производителем этого рода устройств;
	\item{} доступность устройства;
	\item{} <<сильное>> сообщество разработчиков под эту архитектуру;
	\item{} наличие литературы и справочных материалов.
\end{itemize}


По всем этм пунктам идеально подходит микроконтрллер семейства XMega компании ATmel --- atxmega32a4.
Этот микроконтроллер полностью отвечает минимальным требованиям. В целях достижения максимальной
производительности и параллелизма у микроконтроллеров AVR используется
Гарвардская\\*
архитектура (рис. \ref{img:avr_arch}) с отдельными памятью и шинами программ и данных. Инструкции,
хранящиеся в памяти программ, выполняются на одноуровневом конвейере. Это означает, что
во время выполнения одной инструкции выполняется предварительная выборка из памяти программ
следующей инструкции. Данная концепция делает возможным выполнение по одной инструкции за
каждый цикл синхронизации.

\begin{figure}[ht]
	\center{\includegraphics[bb=0 0 590 460, clip, scale=0.6]{avr_arch.png}}
	\caption{Архитектура AVR}
	\label{img:avr_arch}
\end{figure}


Так же в случае необходимости может быть заменён на более мощьный микроконтроллер того же семейства.
Для обеспечения этого функционала компания Atmel предприняла ряд шагов:
\begin{itemize}
    \item{} Была реорганизована область памяти, таким образом, чтобы все связанные между
            собой регистры располагались в памяти строго последовательно;
    \item{} была переписана стандартная библиотека C/C++, так чтобы обеспечивалась
            максимальная переносимость кода написанного для одного микроконтроллера
            на другой микроконтроллер того же семейства.
\end{itemize}

Микроконтроллеры серии XMega обладают следующими характеристиками:
\begin{itemize}
	\item{} 8/16-битное высокопроизводительное RISC ЦПУ AVR;
	\item{} 138 инструкций;
	\item{} аппаратное умножающее устройство;
	\item{} 32 8-битных регистра, напрямую подключенные к АЛУ
	\item{} прямая адресация до 16 Мбайт памяти программ и 16 Мбайт памяти данных;
	\item{} полная поддержка 16/24-битного доступа к 16/24-битным регистрам ввода-вывода;
	\item{} эффективная поддержка 8-, 16- и 32-битных арифметических инструкций;
	\item{} защита от изменения настроек критических функций системы.
\end{itemize}

Помимо этого значительными изменениями нового семейства микроконтроллеров XMega по сравнению
с предыдущим семейством Mega стало\cite{avrref}:
\begin{itemize}
    \item{} Свехмалое энергопотребление обеспечиваемое технологией picoPower второго поколения.
    picoPower позволяет еще больше улучшить эффективность использования батарейного источника.
    То, что микроконтроллеры гарантируют нормальное функционирование при напряжении 1.6 В означает,
    что, например, в составе мобильных телефонов, они могут быть запитаны от стабилизированного
    источника напряжением 1.8В 10\%, тем самым, позволяя снизить себестоимость системы и увеличить
    длительность работы от батарейного источника. 
    \item{} Event System ---  позволяет организовать передачу данных между встроенными периферийными
    устройствами без вмешательства ЦПУ или использования ПДП.
    Этим гарантируется 100\% предсказуемость и малое время реагирования.
    До 8 одновременных событий или условий прерывания в периферийных устройствах могут автоматически
    инициировать действия в других периферийных устройствах \cite{avrxm}.
    \item{} 12 битные АЦП и ЦАП. Для обеспечения высокоточной обработки аналоговых сигналов в состав 
    микроконтроллеров XMEGA интегрированы 12-битные преобразователи аналоговых сигналов. АЦП
    микроконтроллеров XMEGA могут достигать частоты преобразования до 2 МГц, что делает их самыми
        быстродействующими и точными на фоне АЦП обычных микроконтроллеров.
        Поскольку микроконтроллеры XMEGA также интегрируют два 12-битных ЦАП на частоту преобразования
        до 1 МГц и четыре усовершенствованных аналоговых компаратора, это делает их лидерами по
        степени интеграции компонентов для аналоговой обработки.
    \item{} Расширена функциональноть портов ввода-вывода общего назначения. Каждый из потров
    ввода-вывода может быть сконфигурирован как источник внешнего прерывания, при этом событие внешнего прерывания
    может минуюя ЦПУ запускать обработку данных используя модуль ПДП. \\*
    Новыми доступными состояниями портов ввода-вывода стали totem-pull, bus-keeper, wired-or, wired-and.
\end{itemize}


\subsection{Язык программирования микроконтроллера}
\begin{par}
Микропрограммы для микроконтроллеров XMega можно писать на нескольких языках программирования.
Сама компания Atmel предоставляет для программирвания своих устройств два средства разработки:
	\begin{enumerate}
		\item{}AVR assembler --- язык низкого уровня, транслирует исходный код пользовательской
                программы в объектны и исполняемый код. Применение AVR assembler позволяет
                добиться от пограммы наибольшей производительности.
		\item{}AVR-GCC --- кодогенератор и набор дополнительных утилит для Gnu C Compiller от
                компании Atmel, активно поддерживаемый сообществом разработчиков, а так же
                входящий в комплект среды разработки AVR Studio, AVR32 Stdio и
                распростроняемый на правах свободного программного обеспечения.\\*
                В состав AVR-GCC входят компиляторы языка C и C++.
	\end{enumerate}
\end{par}

\begin{par}
В начале 2011 года компания предоставила новую версию своей интегрированной системы
разработки - AVR Studio 5. В отличии от предшественника, новая система позволяет разрабатывать
программное обеспечение для всех семейств микроконтроллеров AVR используюя единую среду разработки.
\end{par}

\begin{par}

\begin{figure}[ht]
	\center{\includegraphics[bb=0 0 800 700, clip, scale=0.5]{avrstudio5.png}}
	\caption{AVR Studio 5}
	\label{img:avr_studio}
\end{figure}


Новая среда разработки AVR Studio 5 (рис. \ref{img:avr_studio}) основана на базе лидирующего продука --- Microsoft Visual Studio.
И так же как и предшественники, новая версия AVR Studio 5  распространяется бесплатно.
В качестве компилятора и кодогенератора в AVR Studio 5 по прежнему используется AVR-GCC,
что позволяет продолжать использовать уже существующие наработки.
\end{par}

\begin{par}
Считается\cite{avrev}, что для разработки эффективного встраиваемого приложения для 8-битных микроконтрллеров
AVR наиболее целесообразно применять язык ассемблера или в случае, когда необходимо добиться
большей читаемости и поддерживаемости приложения --- язык С.
Однако, лидирующая группа японских производителей ЦПУ, возглавляемая NEC, Hitachi, Fujitsu и Toshiba,
разработала специализированных диалект языка C++ --- Embedded C++ (EC++), позволяющий применять
C++ для встраиваемых систем. Основная цель разработки --- перенос существующих методологий и шаблонов
программирования C++ в облать применения встраивамых систем.
При этом программы код генерируемых компиляторами EC++ может даже дать приемущества по сравлению
с языком C. Так как сама структура языка C++ позволяет минимизировать объём кода и одновременно
повышая его эффективность.
\end{par}

\begin{par}
Однако, при выборе в качестве языка программирования C++, из текущей реализации поставляемой с AVR-GCC,
необходимо учитывать некоторые ограничения:
\begin{itemize}
    \item{} Отсутствует стандартная библиотека C++.
    \item{} EC++ --- это только подмножество C++, по этому некоторые особенности языка были
        убраны из стандарта:
        \begin{itemize}
            \item{} Множественное наследование;
            \item{} базовые виртульаные классы;
            \item{} информация времени исполнения;
            \item{} приведение типов (static\_cast, dynamic\_cast, reinterpret\_cast и const\_cast);
            \item{} квалификаторы типов;
            \item{} пространства имён;
            \item{} исключения;
            \item{} шаблоны.
        \end{itemize}
\end{itemize}
\end{par}



\subsection{Выбор языка программирования сетевого сервиса}

При проектировании сетевого сервиса от инструментального средства ожидается, чтобы оно
отвечало следующим требованиям\cite{technob}:

\begin{enumerate}
	\item{} Инструментальное средство должно быть высокого уровня --- язык высокого уровня,
            освобождает разработчика от рутинной работы вроде ручного выделения и освобождения
            памяти, и позволяет ему сфокусироваться на оперировании абстракциями предметной области.
	\item{} Язык должен минимизировать количество ошибок которые может допустить программист в
             процессе разработки системы.
	\item{} Высокая степень параллелизма --- необходима возможность обслуживать тысячи клиентов
            одновременно.
	\item{} Отказоустойчивость --- телеком-системы слишком масштабны, чтобы самому разработчику
             имело смысл даже пытаться предусмотреть все возможные ошибки.
	\item{} Возможность обновления кода сервиса без останова выполнения программы.
	\item{} Наличие обширной системной библиотеки,  а так же предопределённых каркасов
            проектов --- это избавляет разработчика от необходимости реализовывать типовые
            решения самостоятельно, при этом тратя время на тестирование и отладку системы.
            Каркас проекта задаёт общую структуру создаваемой системы, что позволяет ещё
            на этапе проектирования системы оценивать её будущие качества.
\end{enumerate}

\begin{par}
Один из немногих существующих и поддерживаемых на сегоднешний день языков программирования
отвечающих всем этим требованиям является Erlang и платформа Open Telecom Platform (OTP).
\end{par}

\subsubsection{Обзор языка Erlang и платформы OTP}

\begin{par}
В середине 1980-х, Ericsson Computer Science Laboratory было дано задание --- исследовать языки программирования подходящие для разработки телекоммуникационных продуктов нового поколения. Джой Армстронг(Joe Armstrong), Роберт Вирдинг (Robert Virding) и Майк Вильямс (Mike Williams) под руководством Брайна Декера (Bjarne Dacker) потратили два года на прототипирование телекоммуникацинного приложения поочерёдно  используя все доступные на тот момент языки и системы программирования. В результате, несмотря на то, что многие языки программирования обладали интересными и подходящими свойствами, ни один из них не удовлетворял всех их требованиям. В результате они приняли решение создать их собственный язык программирования.
\end{par}

\begin{par}
Erlang был создан под влиянием функциональных языков таких как  ML и Miranda, параллельных языков ADA, Modula и Chill, и языка логического программирования Prolog. Erlang так же унаследовал некоторые черты таких язков как Smalltalk, проприетарных языков Ericsson EriPascal и PLEX \cite{erlang}.
\end{par}

\begin{par}
Используя построенную на Prolog виртуальную машину Erlang (VM), лаборатория потратила  четыре года на
прототипирование телекоммуникационного приложения с применением и постоянными доработками новго языка.
Именно из-за применения метода проб и ошибок язык Erlang стал таким, каким он является сейчас. В 1999
Майк Вильямс переписал на Си виртуальную машину, и годом позже, на этом языке был выпущен первый
коммерческий продукт. 
\end{par}

\begin{par}
История создания языка Erlang важна для понимания его философии, так как в отличии от других языков, которые находили свою нишу уже после разработки и распространения, Erlang изначально создавался для решения конкретных задач бизнеса. Он создавался под задачи построения распределённых, отказоустойчивых систем реального времени массового обслуживания.
\end{par}

\begin{par}
Так как такие области как системы поддержки продаж, банковские системы, системы компьютерной телефонии, системы интеграции уровня предприятий зачастую предъявляют к своему программному обеспечению аналогичные требования, то Erlang нашёл своё применение и в них.
\end{par}

Подтверждением применимости Erlang в этих областях могут служить факты использования этого языка в
проектах компаний:
\begin{itemize}
	\item{} Amazon --- использует Erlang для реализации SimpleDB, предоставления системы
        управления базами данных как части Amazon Elastic Compute Cloud (EC2).
	\item{} Yahoo! --- использует Erlang для реализации своего сервиса социальных закладок,
    который обслуживает более 5 милионов пользователей и 150 милионов URL.
	\item{} T-Mobile --- использует Erlang в их SMS и авторизирующей системах.
	\item{} Motorola --- использует Erlang в системе обработки звонков системы обслуживания клиентов.
	\item{} Ericsson --- использует Erlang для поддержки узлов используемых в GPRS и 3G
    мобильных сетях по всему миру.
	\item{} Facebook и Yandex --- используют написанный на Erlang сервер мгновенных
    сообщений Ejabberd.
\end{itemize}


%%\section[Детализация постановки задачи]{ДЕТАЛИЗАЦИЯ ПОСТНОВКИ ЗАДАЧИ}
\section{ДЕТАЛИЗАЦИЯ ПОСТАНОВКИ ЗАДАЧИ}
\begin{par}
Целью выполняемой работы является автоматизированное проектирование программно-аппаратного комплекса
на основе микроконтролера AVR семейства XMega --- atxmega32a4, позволяющего учитывать текущее время
и выдавать сигнал звукового оповещения при достижении времени заданного пользователем устройства.
\end{par}

\subsection{Функциональная структура устройства}

Функциональная структура устройства приведена на рисунке \ref{img:funcd}.

\begin{figure}[h]
	\center{\includegraphics[bb=0 0 453 437, clip, scale=0.8]{funcd.png}}
	\caption{Фнкциональная структура}
	\label{img:funcd}
\end{figure}

\begin{enumerate}
    \item{}ЦПУ --- основной блок отвечает за отсчёт врени и за организацию канала коммуникации
        между блоками воспроизведения мелодии, хранения данных мелодии, и модуля синхронизации времени.
    \item{}Модуль синхронизации времени отвечает за получение информации о текущем веремни
            от удалённого SNTP сервера.
    \item{}Модуль постоянной памяти отвечает за хранение данных, используемых для формирования
        мелодии будильника.
    \item{}Модуль индикации отвечает за отображение текущего времени и наступившего события
        будильника.
    \item{}Модуль формирования аналогово сигнала иcпользуется для формирования сигнала и проигрывания
            мелодии будильника.
    \item{}Модуль ввода данных используется для устновки текущего времени и задания
        времени срабатывания будильника.
\end{enumerate}


\subsection{Описание работы устройства}
\subsubsection{Центральный микроконтроллер}
\begin{par}
Центральное место в схеме занимает микроконтроллер atxmega32a4. Микропрограмма контроллера
должна быть составлена таким образом, что бы прерывание модуля RTC происходило с частотой 2 Гц.
При приходе прерывания микроконтроллер отсчитывает количество $ \frac{1}{2} $ с. прошедших с момента
включения схемы и выводит текущее время на устройство индикации.
\end{par}

\begin{par}
Программа микроконтроллера должна быть написана на языке C++ для компилятора входящего в поставку
AVR-GCC, с учётом всех допущений указанных в разделе <<Язык программирования микроконтроллра>>.
\end{par}

\subsubsection{Устройство индикации}
\begin{par}
В качестве устройства индикации должен использоваться TFT ЖК-дисплей DST2001PH\cite{display} со встроенным
драйвером ili9320 включённым в режиме system i80 16-бит\cite{ili9320}.
Так как количество портов\\*
ввода-вывода на используемом микроконтролере достаточно, введение дополнительных узлов
расширяющих возможности 
ввода-вывода микроконтроллера --- не предусматривается.
\begin{figure}[h]
	\center{\includegraphics[bb=0 0 150 250, clip, scale=1.0]{ili9320.png}}
	\caption{Внешний вид TFT ЖК-дисплея DST2001PH}
	\label{img:iili9320}
\end{figure}
\end{par}



\begin{par}
В начальный момент работы, устройство должно погасить изображение на ЖК-дисплее и подготовить внутреннюю
память модуля индикации.
\end{par}

\subsubsection{Ввод данных}
\begin{par}
Модуль ввода данных пользователя должен быть выполнен на сенсорном экране ЖК-дисплея,
выполненного по 4-проводной схеме включения резистивного сенсорного экрана. Для
оцифровки значений плучаемых с сенсорного экрана должны использоваться АЦП\cite{avradc} центрального
микроконтроллера.
\end{par}

\begin{par}
В начальный момент работы устройства, после инициализация внутренней памяти ЖК-дисплея, должна быть
произведена процедура калибровки сенсорного экрана, а полученные коэффициенты далее должны быть
использованы для корректировки показаний снятых с сенсорного экрана. Это позволит отказаться
от дополнительной схемы температурной компенсации и статических калибровочных показателей,
и при этом устройство будет выдавать давольно стабильный результат,
так как эксплуатироваться устройство будет в условиях постоянной комнатной температуры,
а собственная рассеиваемая мощьность устройства не должна быть велика на столько, что бы
вносить ощутимую погрешность в показания.
\end{par}


\subsubsection{Звуковое оповещение о наступившем событии}
\begin{par}
Звуковое оповещение о наступившем событии должно быть выполнено на широко распростонённой
микросхеме mc34119, включённой в стандартном режиме (рис. \ref{img:mc34119m}).
\begin{figure}[h]
	\center{\includegraphics[bb=0 0 200 190, clip, scale=0.8]{mc34119.png}}
	\caption{Схема включения mc34119}
	\label{img:mc34119m}
\end{figure}
\end{par}

\begin{par}
В начальном состоянии на вход Chip Disable микросхемы mc34119 должен подаваться сигнал высокого уровня.
Это переводит микросхему в состояние низкого энергопотребления \cite{mc34119}.
\end{par}

\begin{par}
В момент наступления события будильника на вход Chip Disable микросхемы необходимо подать сигнал 
низкого уровня, выполнить задержку не менее 40 мс., необходимую для перевода микросхемы в
нормальное состояние, и затем подавать на неё вход сигнал с ЦАП центрального микроконтроллера.
Принципиальная схема ЦАП микроконтроллеров AVR семейства XMega приведена на рисунке \ref{img:avrdacp}.
\begin{figure}[h]
	\center{\includegraphics[bb=0 0 300 100, clip, scale=0.8]{avrdac.png}}
	\caption{Принципиальная схема ЦАП микроконтроллеров AVR семейства XMega}
	\label{img:avrdacp}
\end{figure}
\end{par}

\begin{par}
Для воспроизведения музыкального отрывка необходимо использовать нижние 8 бит 12-битного
ЦАП микроконтроллера. Опорное напряжение ЦАП микроконтроллера должно быть равно либо внутреннему
напряжению микроконтроллера 1 В, либо стабилизированному напряжению питания ($AV_{cc}$). Выбор того или иного
опорного напряжения ЦАП обусловливается параметрами динамика и необходимым уровнем громкости.
Для переодического получения данных и их переноса в буфер воспроизведения необходимо использовать
обработик прерывания таймера Timer 1 микроконтроллера. Частота срабатывания таймера выбирается
в соответствии с частотой дискретизации музыкального отрывка.
Механизм программирования ЦАП микроконтроллеров AVR семейства XMega детально описан в документации\cite{avrdac}.
\end{par}


\subsubsection{Устройство хранеия музыкальных отрывков}
\begin{par}
В качестве устройства хранения звуковых отрывков необходимо использовать энергонезависимые карты
памяти MicroSD. Применение этого вида памяти позволит не пребегая к перепрограммированию устройства
вносить музыкальные фрагменты на карту с персонального компьютера, оснащённого модулем
чтения/записи карт MicroSD.
\end{par}

\begin{par}
Для воспроизведения звуковых отрывков центральным микроконтроллером должен быть
организован буфер воспроизведения, таким образом, что бы в случае отсутствия карты MicroSD или
сбоя доступа к ней, в него вносились записанные на этапе программирования данные EEPROM микроконтроллера.
\end{par}

\begin{par}
Для чтения данных карт памяти MicroSD должен использоваться SPI режим работы. Методика программирования
интерфейса SPI микроконтроллеров AVR семейства XMega дана в документации \cite{avrspi}.
\end{par}

\subsubsection{Синхронизация времени}
\begin{par}
В качестве контроллера сетевого интерфейса необходимо использовать микроконтроллер фирмы Micro Chip
enc28j60. Пример включения этого микроконтроллера дан на рисунке \ref{img:ienc28j60}.
\begin{figure}[h]
	\center{\includegraphics[bb=0 0 500 220, clip, scale=0.6]{enc28j60.png}}
	\caption{Схема включения микроконтроллера enc28j60}
	\label{img:ienc28j60}
\end{figure}

Этот микроконтроллер требует использования трансформатора с отношением 1:1 сертефицированного
для сетей 10base-T. По этому для упрощения изделия необходимо использовать готовые RJ45
коннекторы ''Magjack'', которые включают в своей конструкции необходимые трансформаторы и
свето диоды. В случае использования этого метода в схему устройства необходимо будет добавить
только одну индуктивность в 10 мкГн.
\end{par}

\begin{par}
Один раз в 30 минут устройство должно повышать рабочую частоту микроконтроллера до 32 МГц.
И производить попытку синхронизации времени с удалённым SNTP сервисов. По завершению процедуры
синхронизации времени, устройство должно вновь перевести микроконтроллер на рабочую частоту 2 МГц.
\end{par}

\begin{par}
Взаимодействие центрального микроконтроллера и микроконтроллера сетевого интерфейса осуществляется
по протоколу SPI \cite{enc28j60}, расширенному дополнительными сигнальными линиями.
\end{par}


\subsection{Проектирование принципиальной схемы и печатной платы}
\begin{par}
Необходимо разработать принципиальную схему и печатную плату устройства в системе автоматизированного
проектирования Eagle.
\end{par}

\begin{par}
Необходимо разработать схемотическое обозначение и посадочное мето для подключения ЖК-дисплея.
\end{par}

\begin{par}
Необходимо провести DRC проверку выполненой печатной платы на соотвествие параметрам:
\begin{itemize}
    \item{} Количество слоев: до 2;
    \item{} максимальный размер платы (размер рабочего поля): 380х320 мм;
    \item{} минимальная ширина проводника/минимальный зазор: 0,24/0,24 мм;
    \item{} минимальный отступ полигона: 0,24 мм;
    \item{} минимальный диаметр отверстия/минимальная площадка на переходном отверстии: 0,2/0,6 мм;
    \item{} минимальной диаметр монтажного отверстия — 0,6 мм;
    \item{} размер минимальной контактной площадки: для металлизированных отверстий 1,2—1,6 мм — +0,55 мм;
    \item{} минимальная толщина линии шелкографии (маски) — 0,15 мм;
    \item{} минимально допустимое отторжение маски от КП — 0,1 мм, для КП ИМС с шагом 0,5 мм — 50 мкм;
    \item{} минимальная высота шрифта шелкографии — 1,5 мм.
\end{itemize}
\end{par}

\subsection{Разработка SNTP сервиса}
\begin{par}
Необходимо разработать SNTP сервис на языке ErLang. Дополнительных требований к проектированию этого сервиса
не предъявляется, так как основное его назначение --- получение более удобного
инструмента отладки подпрограммы синхронизации времени.
При этом делается допущение, что успешный результат испытания по синхронизации свремени с разрабатываемым
сервисом, подтверждает правильность работы как самого сервиса, так и микропрограммы устройства.
\end{par}
\newpage{}

\section{ОРГАНИЗАЦИОННО-ЭКОНОМИЧЕСКАЯ ЧАСТЬ}

\subsection{Обоснование необходимости и актуальности разработки проекта}
\begin{par}
В существующем на данный момент варианте контроля и оперативного
вмешательства в ход соблюдения технологического процесса обработки выпускаемой продукции (колбас
МПП <<Волжский бекон>>) предусмотрен лишь контроль со стороны работника термического
участка-оператора термопечей. Данный вариант значительно повышает степень риска выпуска некачественной
продукции в следствии несоблюдения и выдержки требуемых параметров обработки. Ответственность, в
данном случае, полностью ложится на конкретного исполнителя и зависит от его внимательности,
добросовестности и личной заинтересованности в результатах своей работы.
\end{par}

\begin{par}
В современных условиях вариант совершенно не приемлемый, так как результат  работы всего
коллектива не может зависеть от желания или нежелания отдельного работника.
В этих целях, для устранения подобных проблем, широко применяется 
автоматизация технологических процессов. Автоматизация технологических процессов это
совокупность методови средств, предназначенных для реализации системы или систем, позволяющих
осуществлять управление самим технологическим процессом без непосредственного участия человека,
либо оставления за ним права принятия наиболее ответственного решения.
Как правило, в результате автоматизации технологического процесса создается АСУТП.
Основными целями автоматизации технологического процесса являются: повышение эффективности,
безопасности, экономичности и экологичности производственного процесса.
\end{par}

\subsection{Определение трудоемкости разработки системы контроля технологического процесса}
Для определения трудоемкости разработки программного продукта необходимо воспользоваться <<Укрупненными
нормами времени на изготовление и сопровождение программных средств вычислительной техник>>.
Стадиями разработки программных средств являются:
\begin{itemize}
    \item{} техническое задание;
    \item{} эскизный проект;
    \item{} технический проект;
    \item{} рабочий проект;
    \item{} внедрение.
\end{itemize}

Так как программный продукт является развитием определенного параметрического ряда ПП на прежнем типе ЭВМ/ОС,
то Кн=0.4 --- поправочный коэффициент, учитывающий степень новизны ПП и использования при разработке
программного продукта новых типов ЭВМ и ОС. Степень  охвата реализуемых функций стандартными ПП составляет 60\%, 
поэтому принимаем Кт=0.8 --- поправочный коэффициент, учитывающий степень 
использования в разработке типовых ПП.

Для разработки программного продукта используем языки JavaScript, Erlang, C++.
В качестве среды разработки используется персональная ЭВМ с установленной Apple Mac OS X 10.6.
Поправочный коэффициент, учитывающий характер среды разработки и средств разработки ПП, Кур=0.36.
Программный продукт имеет вторую группу сложности и выполняет следующие функции:

\begin{itemize}
    \item{} управление работой компонентов ПС (V1);
    \item{} обработка ошибочных данных (V2);
    \item{} вывод данных на экран (V3);
    \item{} система настройки ПС на условия применения (V4).
\end{itemize}

Общая трудоемкость разработки ПП расчитывается по формуле

\begin{equation}
    Т_{общ.} = \sum_{i=0}^{n}Т_i
\end{equation}
\begin{ESKDexplanation}
    \item[где ]{} $Т_i$ --- трудоемкость i-той стадии разработки ПП (чел.-дни);
    \item{} n --- количество стадий разработки.
\end{ESKDexplanation}

В соответствии с п.11 <<Укрупненных норм времени на изготовление и сопровождение программных
средств вычислительной техники» в том случае, когда на стадии технико-экономического
обоснования проекта невозможно расчитать точный объем ПП, то данный объем может быть 
получен на основании прогнозной оценки или путем применения нормативов по каталогу функций.
Общая трудоемкость разработки ПП определяется по формуле

\begin{equation}
    То = Тур \times{} Ксл
\end{equation}
\begin{ESKDexplanation}
    \item[где ]{} Тур --- трудоемкость разработки ПП с учетом конкретных условий (чел.-дн.)
    \item{} Ксл --- коэффициент сложности 
\end{ESKDexplanation}


\begin{equation}
    Т_{ур} = ТБ \times{} К_{ур}
\end{equation}

\begin{equation}
    К_{сл} = 1 + \sum{}_{i=0}{n}К_i
\end{equation}
\begin{ESKDexplanation}
    \item[где ]{} ТБ --- базова трудоемкость разработки ПП (чел.-дн);
    \item{} Кур --- поправочный коэффициент, учитывающий характер среды и средств разработки ПП;
    \item{} Кi --- коэффициент повышения сложности ПП.
\end{ESKDexplanation}

\begin{par}
С учетом вышесказанного получаем следующие значения расчетных величин:
$Кур = 0.34$ \\
$Кi  =  0.08$,  тогда  $Кур = 1.08$ \\
$Т_{баз} = 2291$ (чел.-дн.) \\
$V_0 = 6000$ (строк) \\
$Тур = ТБ \times{} Кур = 2291 \times{} 0.34 = 778.9$ \\
$То =  Тур \times{} Ксл = 778.94 \times{} 1.08 = 841.25 (чел.-дн.) $ \\
\end{par}

\begin{par}
Трудоемкость i-той стадии разработки программного проекта определяется
по формулам: 
\begin{equation}
	Ti = Li \times{} Kн \times{} To
\end{equation}
для i = 1, 2, 3 , 5.
\begin{equation}
	T_i = Li \times{}  KT \times{} Kн \times{} T0
\end{equation}
для i = 4.
\end{par}

\begin{ESKDexplanation}
	\item[где ]{} Кн --- поправочный коэффициент, учитывающий степень новизны ПП и
		использования при разработке новых типов ЭВМ и ОС;
	\item{} Кт --- поправочный коэффициент,учитывающий степень использования в разработках типовых ПП;
	\item{} Li --- удельный вес трудоемкости i-той стадии разработки ПП.
\end{ESKDexplanation}

\begin{par}
Определяем значения вышеуказанных коэффициентов: \\
$Кн = 0.4$ \\
$Кт = 0.8$ \\
$L_1 = 0.06$, тогда: $Т_1 = 0.06 \times{} 0.4 \times{} 841.25 = 20.19$ (чел.-дн.) \\
$L_2 = 0.2$; $Т_2 = 0.2 \times{} 0.4 \times{} 841.25 = 67.3$ (чел.-дн.) \\
$L_3 = 0.3$; $Т_3 = 0.3 \times{} 0.4 \times{} 841.25 = 100.95$ (чел.-дн.) \\
$L_4 = 0.34$; $Т_4 = 0.34 \times{} 0.4 \times{} 0.8 \times{} 841.25 = 91.53$ (чел.-дн.) \\
$L_5 = 0.1$; $Т_5 = 0.1 \times{} 0.4 \times{} 841.25 = 33.648$ (чел.-дн.)
\end{par}

\begin{par}
\begin{table}
\caption{Трудоемкость стадий разработки ПП}
\begin{tabular}{|l|p{7cm}|p{4cm}|r|}
\hline{}
№ & Стадия разработки & Удельный вес & Трудоемкость \\
\hline{}
1 & Техническое задание & 0.06 &  20.19 \\
\hline{}
2 & Эскизный проект & 0.2 & 67.30 \\
\hline{}
3 & Технический проект & 0.3 & 100.95 \\
\hline{}
4 & Рабочий проект & 0.34 & 91.53 \\
\hline{}
5 & Внедрение & 0.1 & 33.65 \\
\hline{}
& Итого: & & 314.00 (чел.- дн.) \\
\hline
\end{tabular}
\label{table:trdSoftDev}
\end{table}
Следовательно, общая трудоемкость разработки программного продукта
составляет: $To = \sum{}_{i=1}^{n=5} = T1 + T2 + T3 + T4 + T5 = 314.0$ (человеко-дней).
\end{par}


\subsection{Определение состава исполнителей}
Среднее число исполнителей, участвующих в разработке программного проекта расчитывается по формуле:
\begin{equation}
    Chi=\frac{To}{Fm\times{}D}
\end{equation}

\begin{ESKDexplanation}
    \item[где ]{} Fм --- фонд времени одного работающего в месяц (дней);
    \item{} Д --- директивный срок выполнения разработки (месяцев).
\end{ESKDexplanation}

Фонд рабочего времени одного работающего в месяц определяется по формуле:
\begin{equation}
    F = \frac{Dk - Db - Dn}{12}
\end{equation}

\begin{ESKDexplanation}
    \item[где ]{} Dk --- общее число дне в году, равно 365 дней;
    \item{} Dв --- число выходных дне в году;
    \item{} Dп --- число праздничных дне в году.
\end{ESKDexplanation}

В 2010 году при пятидневной рабочей неделе с двумя выходными, количество рабочих дней составило 249,
в том числе 5 сокращенных на один час рабочих дней. Количество выходных составило 116 с учетом 5
дополнительных дней отдыха. Следовательно фонд рабочего времени одного работающего в месяц всоответствии
с вышесказанным составит:

$F_m = \frac{249}{12} = 20.75 (дней)$


Среднее число исполнителей,участвующих в разработке проекта определяется по формуле:
\begin{equation}
	Чи = \frac{To} {Fм \times{} Д}
\end{equation}

\begin{ESKDexplanation}
	\item[где ] Д --- директивный срок исполнения проекта , принимаем 5 месяцев.
\end{ESKDexplanation}


Тогда расчетное число исполнителей будет равно : $Чи = \frac{314}{20.75 \times{} 5} = 3.06$ (человек).

\begin{par}
Исходя из полученного результата принимаем число исполнителей равным 3.
А именно: два инженера-программиста 3 квалификационного уровня с окладом 6500 рублей (повышающий коэффициент = 0.25);
один инженер-программист 4 квалификационного уровня (ведущий) исполняющий обязанности руководителя
проекта с окладом 7500р (повышающий коэффициент=0.44). Базовый оклад, без учета повышающего коэффициента,
составляет 5200 рублей.
\end{par}

\begin{par}
\begin{table}
\caption{Состав исполнителей исполнителей программного проекта}
\begin{tabular}{|l|c|c|c|c|r|}
\hline{}
№ & Профессия исполнителя & Количество человек & Оклад базовый & Повышающий  коэфф-т & Заработная плата с учетом повышающего коэф-та \\
\hline{}
1 & Инженер-программист & 2 & 5200 & 0.25 & 6500 \\
\hline{}
2 & Руководитель & 1 & 5200 & 0.44 & 7500 \\
\hline{}
& Всего &  & 3 & 20500(руб.) \\
\hline
\end{tabular}
\label{table:econSostavIspol}
\end{table}
\end{par}

\begin{par}
В таблице \ref{table:econSostavIspol} указана заработная плата исполнителей без учета премии, которая будет в
дальнейшем учтена, при расчете стоимости разработки программного продукта.
\end{par}

\subsection{Расчет  стоимости  разработки программного продукта}
\begin{par}
Цена на научно-техническую продукцию устанавливается на этапе технического задания
до начала проведения исследований.
При  этом она должна соответствовать ряду требований: возмещать  издержки разработчику,
регулировать спрос и предложение такого вида продукции,заинтересовывать
разработчика и заказчика в проведении эффективных разработок. В основе
договорной цены программного продута заложена  сметная  стоимость 
разработки , определяемая в калькуляционном разрезе.
\end{par}

Материалы и покупные изделия.

\begin{par}
\begin{table}
\caption{Расчет затрат на материалы и покупные изделия}
\begin{tabular}{|l|p{7cm}|p{4cm}|c|r|}
\hline{}
№ & Наименование & Цена за единицу& Норма расхоа(месяц) & Стоимость(руб) \\
\hline{}
1 & Бумага офисная & 135.0 & 1 & 135.0 \\
\hline{}
2 & Услуги интернета & 360.0 & --- & 350.0 \\
\hline{}
3 & Компакт диски & 20.0 & 1 & 20.0 \\
\hline{}
& Итого: & &  505.0 \\
Транспортно-заготовительные расходы & 10\% & 50.5 \\
\hline{}
& Всего: & &  555.5 \\
\hline
\end{tabular}
\label{table:reschZatrMat}
\end{table}
\end{par}

\begin{par}
Таким образом затраты на материалы, услуги и покупные изделия составляют:
\begin{itemize}
    \item{} в месяц --- 555.5 рублей
    \item{} в год --- 6666 рублей
    \item{} за время директивного сока --- 2777.5 рублей.
\end{itemize}
\end{par}

\subsubsection{Основная заработная плата исполнителей, занятых в разработке проекта}
Основная заработная плата определяется по формуле:
\begin{equation}
	Зп = З_{дол.ок.} + З_{дол.ок.} \times{} \sum{}_{i=1}^n
\end{equation}

\begin{ESKDexplanation}
	\item[где ]{} ЗП --- заработная плата;
	\item{} ЗДОЛ.ОК. --- должностной оклад работника.
\end{ESKDexplanation}

Должностной оклад работника определяется по формуле:
\begin{equation}
	Здол.окл. = Зокл \times{} КПОВ^{min}
\end{equation}

\begin{ESKDexplanation}
	\item[где ]{} Ki --- повышающий коэффициент (в данном случае это премия 15\%);
	\item{} Кпов --- повышающий коэффициент квалификационного уровня для конкретного работника (min);
	\item{} Зокл. --- базовый оклад по ПКГ (руб);
\end{ESKDexplanation}

\begin{par}
Подставляем исходные значения в формулы (1.9) и получаем:
\begin{itemize}
	\item{} Заработная плата инженера-программиста равна --- 7475 рублей;
	\item{} Заработная плата руководителя — 8625 рублей.
\end{itemize}
Таким образом общая основная заработная плата исполнителей, занятых в работе над проектом оставит:
\begin{itemize}
	\item{} за месяц --- 23575 рублей;
	\item{} за год --- 282900 рублей;
	\item{} за 5 месяцев --- 117875 рублей.
\end{itemize}
\end{par}
%
%
%\subsubsection{Дополнительная заработная плата}
%Дополнительная заработная плата определяется по формуле:
%\begin{equation}
%	Рдоп = \frac{Росн \times{} ННДоп} {100\%}
%\end{equation}
%
%\begin{ESKDexplanation}
%	\item[где ]{} Росн --- основная заработная плата;
%	\item{} Ндоп --- норматив дополнительной заработной платы (15-20\%);
%\end{ESKDexplanation}
%
%В наших расчетах принимаем процент премии равный 15.
%Производим вычисления и получаем следующие значения дополнительной
%заработной платы:
%\begin{itemize}
%	\item{} в месяц Рдоп = 3536.25 (руб.);
%	\item{} за год Рдоп = 42435 (руб.);
%	\item{} за 5 месяцев = 17681.25 (руб.).
%\end{par}
%
%
%\subsubsection{Единый социальный налог}
%\begin{par}
%Единый социальный налог определяется по формуле:
%\begin{equation}
%    Ротч = \frac{(Розп + Рдзп) \times{} Нотч} {100\%}
%\end{equation}
%
%Нотч. --- норматив отчислений на социальные  нужды, составляет 34 \%.
%Таким образом месячный размер отчислений составит-8015.5 рублей
%Отчисления в единый социальный налог за год составят -96186 рублей
%Отчисления за директивный период будут равны — 40077.5 рублей
%
%\subsubsection{Амортизационные  отчисления  с  оборудования}
%\begin{par}
%Определяются по формуле:
%\begin{equation}
%    А = ЦОБ \times{} НА
%\end{equation}
%\begin{ESKDexplanation}
%	\item[где ]{} ЦОБ --— балансовая стоимость одной ПЭВМ с периферией.
%\end{ESKDexplanation}
%В нашем случае используется три одинаковых электронно-вычислительные машины.
%Балансовая стоимость составляет 10000 рублей.
%\end{par}
%
%
%\begin{par}
%НА --- норма амортизации на компьютерную технику, определяется по формуле:
%\begin{equation}
%	НА = \frac{1}{Тп} \times{} 100\%
%\end{equation}
%\begin{ESKDexplanation}
%	\item[где ] ТП --- срок полезного использования объекта в месяцах.
%\end{ESKDexplanation}
%Для компьютерной техники срок полезного использования установлен 2--3 года.
%Тогда $НА= (\frac{1}{3} \times{} 12) \times{} 100 = 2.7$.
%\end{par}
%
%Подставляем полученное значение в формулу для вычисления амортизационных отчислений
%и получаем: $А = ЦОБ \times{} НА = 3 \times{} 10000 \times{} \frac{2.7}{100} = 810 (руб.)$.
%Таким образом ежемесячные отчисления на амортизацию оборудования составляют
%810 рублей. Отчисления за год составят: $810 \times{} 12 = 9720 руб.$. Отчисления за
%директивный период: $810 \times{} 5 = 4050 руб.$.
%
%\subsubsection{Затраты на силовую электроэнергию и освещение}
%Затраты на силовую электроэнергию определяются по формуле:
%\begin{equation}Зэн^{с} = Fэф \times{} Цэ \times{} Рэвм\end{equation}
%\begin{equation}Fэф = Fном \times{} Ксм \times{} (1 - \frac{a}{100})\end{equation}
%\begin{equation}Fном. = tсм \times{} Dр - tп \times{}Dп\end{equation}
%
%\begin{ESKDexplanation}
%	\item[где ]{} Fэф --- эффективный фонд рабочего времени оборудования;
%	\item{} Fном --- номинальный  фонд времени работы оборудования;
%	\item{} tсм --- длительность рабочей смены в часах;
%	\item{} Dр --- количество рабочих дней в плановом периоде;
%	\item{} tп = 1 --- продолжительность нерабочего времени в предпраздничные дни;
%	\item{} Dп --- количество предпраздничных дней;
%	\item{} α --- процент плановых потерь рабочего времени.
%\end{ESKDexplanation}
%
%\begin{par}
%Количество рабочих дней составляет 249 дней, в том числе 5 дней предпраздничных, сокращенных на один час.
%В этом случае номинальное количество рабочих часов составит: $Fном =  8 \times{} 249 - 1 \times{} 5 = 1987 (часов)$.
%\end{par}
%
%\begin{par}
%Эффективный фонд  рабочего времени определяется о формуле:
%\begin{equation}
%	Fэф = Fном \times{} Kсм \times{} (1 - \frac{a}{100})
%\end{equation}
%\end{par}
%
%\begin{ESKDexplanation}
%	\item[где ]{} Ксм --- количество смен;
%	\item{} α --- коэффициент, учитывающий процент плановых потерь рабочего времени.
%\end{ESKDexplanation}
%
%
%Подставляем  все значения в формулу и получаем величину эффективного рабочего времени:
%$Fэф = 1987 \times{} 1 \times{} (1 - \frac{20}{100}) = 1589.6 (ч.)$.
%
%\begin{par}
%Затраты на силовую электроэнергию вычисляются по следующей формуле:
%\begin{equation}
%	Зэн^с = Fэф \times{} Цэ \times{} Рэвм
%\end{equation}
%\begin{ESKDexplanation}
%	\item[где ]{} Цэ --- стоимость электроэнергии (в нашем случае = 4.94 руб/кВт);
%	\item{} Рэвм --- суммарная мощность ПЭВМ с  периферией (0.7 --- 1.2 кВт ).
%\end{ESKDexplanation}
%$Зэн^c = 1589.6 \times{} 4.94 \times{} 0.7 = 5496.84$ (руб.)
%\end{par}
%
%\subsubsection{Затраты электроэнергии на освещение}
%\begin{equation}
%	Зэн = Fэф \times{} Цэ \times{} Рос.
%\end{equation}
%\begin{ESKDexplanation}
%	\item[где ]{} Рос --- суммарная мощность, идущая на освещение (в нашем случае = 0.2кВт).
%\end{ESKDexplanation}
%
%Следовательно Зэн = 1589.6 * 4.94 * 0.2 = 1570.53 (руб.).
%Затраты за пять месяцев составят: 654.39 (руб.).
%Общие затраты на электроэнергию: $Зэн = Зэн^с + Зэн^{ос}$.
%
%
%Годовые  затраты на электроэнергию  составляют 7067.37  (руб.).
%Затраты за пять месяцев равны 2944,74 (руб.).
%
%
%\subsubsection{Расходы на профилактику оборудования}
%Расходы на профилактику оборудования вычисляются по формуле:
%\begin{equation}
%	Рпроф = \frac{ЦОБ \times{} Нпроф}{100\%}
%\end{equation}
%
%\begin{ESKDexplanation}
%	\item[где ]{} ЦОБ --- балансовая стоимость одной ПЭВМ с периферией;
%	\item{} Нпроф --- процент расходов на профилактику (2-4\%).
%\end{ESKDexplanation}
%
%$ Рпроф = (10000 * 2) / 100  = 200 (руб.) $
%
%Так как в работе над проектом задействованы три ПВЭМ, то суммарные расходы на профилактику составят 600 рублей.
%
%\subsubsection{Прочие производственные расходы}
%\begin{equation}
%Рпр.=\frac{Росн\times{}Нпр}{100\%}
%\end{equation}
%
%\begin{ESKDexplanation}
%	\item[где ]{} Росн --- основная заработная плата работников, обеспечивающих функционирование ПВЭМ;
%	\item{} Нпр --- процент прочих производственных расходов (30\%).
%\end{ESKDexplanation}
%
%\begin{par}
%$Рпр. = \frac{(23575 \times{} 30)}{100} = 7072.5$ (руб.). \\
%Производственные расходы за год =  84870 (руб.). \\
%Производственные расходы за 5 месяцев = 35362.5 (руб.).
%\end{par}
%
%\subsubsection{Накладные расходы}
%\begin{equation}
%	Рнак = \frac{Розп \times{} Ннакл}{100}
%\end{equation}
%\begin{ESKDexplanation}
%	\item[где ]{} Розп --- основная заработная плата;
%	\item{} Ннакл. --- норматив накладных расходов (120-150\%);
%\end{ESKDexplanation}
%
%$Рнак = \frac{23575 \times{} 120}{100} = 28290$ (руб.).
%
%Годовая величина накладных расходов равна = 28290 * 12 = 339480 (руб.).
%Накладные расходы за 5 месяцев равны = 28290 * 5 = 141450 (руб.).
%
%\subsubsection{Расчет расходов на содержание и эксплуатацию ПЭВМ}
%$Рсэ = Росн + Рдоп + Ротч + А + Зэн + Рпроф + Рпр$. \\
%$Рсэ = 282900 + 42435 + 96186 + 9720 + 7067.37 + 600 + 84870 = 517417.37$ (руб).
%
%Стоимость одного машино-часа работы ПВЭМ:
%\begin{equation}                                                   
%	Смч = \frac{Рсэ}{Fэф}
%\end{equation}
%\begin{ESKDexplanation}
%	\item{} Рсэ --- годовые расходы на содержание и эксплуатацию одной ПВЭМ. В нашем случае задействованы три ПВЭМ;
%	\item{} Fэф --- эффективный годовой фонд времени работы ПВЭМ.
%\end{ESKDexplanation}
%
%$Смч = \frac{172472.4}{1589.6} = 108.5$ (руб.)
%
%
%\begin{par}
%\begin{table}
%\caption{Расчет затрат на эксплуатацию оборудования (ПВЭМ)}
%\begin{tabular}{|l|p{7cm}|r|}
%\hline{}
%№ & Показатель & Значение \\
%\hline{}
%1 & Основная заработная плата & 282900.0 \\
%\hline{}
%2 & Дополнительная заработная плата & 42435.0 \\
%\hline{}
%3 & Отчисления единого социального налога & 96186.0 \\
%\hline{}
%4 & Амортизационные отчисления & 9720.0 \\
%\hline{}
%5 & Затраты на электроэнергию & 7067.37 \\
%\hline{}
%6 & Расходы на профилактику оборудования & 600.0 \\
%\hline{}
%7 & Прочие производственные расходы & 84870.0 \\
%\hline{}
%8 & Годовые расходы на содержание и эксплуатацию & 517417.37 \\
%\hline{}
%9 & Стоимость одного машино-часа работы ПВЭМ & 108.5 \\
%\hline{}
%10 & Расходы на содержание и эксплуатацию ПЭВМ, относящиеся к данному ПП & 225489.04 \\
%\hline
%\end{tabular}
%\label{table:reschZatrMat}
%\end{table}
%\end{par}
%
%
%
%\subsubsection{Расчет расходов на содержание и эксплуатацию ПЭВМ, относящихся к данному программному продукту}
%\begin{ESKDexplanation}
%Рсэ^{пс} = Cмч \times{} Тпп
%\end{ESKDexplanation}
%\begin{ESKDexplanation}
%	\item[где ]{} Смч --- стоимость одного машино-часа работы ПЭВМ;
%	\item{} Тпп --- суммарное время этапов , требующих использования ПЭВМ;
%\end{ESKDexplanation}
%
%$Тпп =  67.3 + 100.95 + 91.53 =259 .78$ (чел.-дн) (или 2078.24 час).
%$Рсэ =  108.5 * 2078.24 = 225489.04$ (руб.).
%
%
%\begin{par}
%\begin{table}
%\caption{Расчет сметной стоимости программного продукта}
%\begin{tabular}{|l|p{7cm}|r|}
%№ & Наименование затрат & Величина затрат \\
%\hline{}
%1 & Материалы и покупные изделия & 2777.5 \\
%\hline{}
%2 & Расходы на содержание эксплуатацию оборудования & 225302.0 \\
%\hline{}
%3 & Основная заработная плата исполнителей & 117875.0 \\
%\hline{}
%4 & Дополнительная заработная плата исполнителей & 17681.25 \\
%\hline{}
%5 & Единый социальный налог & 40077.5 \\
%\hline{}
%6 & Накладные расходы & 141450.0 \\
%\hline{}
%7 & Сметная стоимость разработки & 545163.25 \\
%\hline{}
%8 & Прибыль & 81774.49 \\
%\hline{}
%9 & Договорная цена & 5995.99 \\
%\hline{}
%10 & Затраты на эксплуатацию & 225489.04 \\
%\hline
%\end{tabular}
%\label{table:softwareCost}
%\end{table}
%\end{par}
%
%
%\subsection{Расчет трудоемкости сопровождения системы}
%Параметры влияющие на расчет трудоемкости сопровождения:
%
%\begin{par}
%\begin{table}
%\caption{Параметры влияющие на расчет трудоемкости сопровождения}
%\begin{tabular}{|l|p{7cm}|r|}
%№ & Название параметра & Принимаемые значения \\
%\hline{}
%1 & Характер поставки & Локальная поставка нестандартного комплекта ПП \\
%\hline{}
%2 & Характеристика средств разработки & Процедурные алгоритмические языки \\
%\hline{}
%3 & Степень участия службы сопровождения в разработке ПП & Не участвовала \\
%\hline{}
%4 & Характер внедрения & Локальное внедрение ПП \\
%\hline{}
%5 & Функции ПП & Управление работой компонентов ПС. Отработка ошибочных ситуаций. Вывод данных на экран система настройки ПС на условия применения \\
%\hline{}
%6 & Объем документации & 6000 строк \\
%\hline{}
%7 & Разработка дополнительных функций & Порядка 1000 условных машинных команд. \\
%\hline
%\end{tabular}
%\label{table:supportLaboriousness}
%\end{table}
%\end{par}
%
%
%Сопровождение подразумевает под собой поддержание технически исправного состояния.
%Обновление документов. Общая трудоемкость сопровождения ПП расчитывается по формуле:
%\begin{equation}
%	Тсопр. = Тос + Тпф + Тип + Топ
%\end{equation}
%\begin{ESKDexplanation}
%	\item[где ]{} Тос --- трудоемкость приемки и освоения ПП (чел.-дн.);
%	\item{} Тпф --- трудоемкость проверки функционирования поставленных ПП на контрольных задачах пользователя;
%	\item{} Тап --- трудоемкость изготовления и контроля комплекта поставки программной части на магнитном носителе;
%	\item{} Топ --- трудоемкость оказания технической помощи пользователю.
%\end{ESKDexplanation}
%
%\begin{par}
%\begin{equation}Тос=Ксл\times{}Кан\times{}Куч\times{}Нвр.ос.\end{equation}
%\begin{ESKDexplanation}
%	\item{} Ксл --- коэффициент, учитывающий сложность ПП;
%	\item{} Кан --- коэффициент, учитывающий наличие в фонде аналогов данного ПП;
%	\item{} Куч --- коэффициент, характеризующий степень участия службы сопровождения в разработке ПП;
%	\item{} Нвр.ос --- норма времени на приемку и освоение опытного образца ПП.
%\end{ESKDexplanation}
%\end{par}
%
%\begin{par}
%\begin{equation}Ксл =1 + sum{}_{i=1}^{n}Ксл_i\end{equation}
%\begin{ESKDexplanation}
%	\item[где ]{} Ксл._i --- коэффициент,  учитывающий уровень повышения сложности ПП.
%\end{ESKDexplanation}
%\end{par}
%
%\begin{par}
%Находим требуемые коэффициенты из таблиц: \\
%Ксл. = 0.21 \\
%Нвр.ос. = 23 (чел.-дн.) \\
%Куч. = 1.1 \\
%Кан. = 0.12
%\end{par}
%
%\begin{par}
%Определяем кэффициент сложности : \\
%Ксл. = 1 + 0.21 = 1.21. \\
%Тос. =  1.21 \times{} 0.12 \times{} 1.1 \times{} 23 = 3.67 (чел.-дн.) 
%\end{par}
%
%\begin{par}
%\begin{equation}
%	Тпф = Кхв \times{} Нвр.пф.
%\end{equation}
%\begin{ESKDexplanation}
%	\item[где ] Кхв. --- коэффициент, учитывающий характер внедрения;
%	\item{} Нвр.пф. --- норма времени на проверку функционирования поставляемых ПП на контрольных задачах пользователя.
%\end{ESKDexplanation}
%\end{par}
%
%\begin{par}
%Тогда :
%Тпф. = 1 * 4 = 4 ( чел.-дн.)
%Тип. = 1 (чел.-дн.)
%Топ. = Кхв * Ксл. * Нвр.оп.
%Нвр.оп. - норма времени на оказание технической помощи 
%Топ. = 1.0 * 1.21 * 2.5 =   3.025 (чел.-дн.)
%В итоге получаем суммарное время сопровождения :
%Тсопр. = Тос + Тпф +Тип + Топ    или
%Тсопр = 3.67 + 4 + 1 + 3.025 =  11.6 (чел.-дн.)
%\end{par}
%
%
%\subsection{Определение стоимости сопровождения программного продукта}
%\begin{equation}
%	Ссопр = Тсопр \times{} ЗПдн
%\end{equation}
%\begin{ESKDexplanation}
%	\item[где ]{} Зпдн --- дневная заработная плата исполнителя.
%\end{ESKDexplanation}
%
%Дневная заработная плата исполнителя, осуществляющего сопровождение программного продукта вычисляется по формуле:
%\begin{equation}
%	ЗПдн = \frac{ЗП}{Fм}
%\end{equation}
%\begin{ESKDexplanation}
%	\item[где ]{} ЗП --- месячная заработная плата (7475руб.);
%	\item{} Fм --- месячный фонд рабочего времени (20.75 дней).
%\end{ESKDexplanation}
%$Зпдн = \frac{7475}{20.75} = 360.24$ ( руб.).
%
%Стоимость сопровождения ПП: $Ссопр. = Тсопр. \times{} ЗПдн = 11.6 \times{} 369.24 = 4178.78$ (руб.).
%
%\subsection{Планирование цены и прогнозирование прибыли}
%\subsubsection{Сравнительный анализ существующих средств автоматизированного контроля технологических параметров и предлагаемого в данном проекте}
%Существующий рынок средств промышленной автоматики в основном представлен  продукцией фирмы <<ОВЕН>>.
%Предприятие выпускает широкую гамму контрольно-измерительных приборов.
%В том числе датчики измерения температуры (табл. \ref{table:temperatureSensors}), веса и других физических параметров.
%
%\begin{par}
%\begin{table}
%\caption{Датчики измерения температуры}
%\begin{tabular}{|l|p{7cm}|r|}
%№ & Наименование & Цена \\
%\hline{}
%1 & Универсальный измеритель-регулятор температуры ОВЕН ТРМ138 & 9440.0 \\
%\hline{}
%2 & Универсальный регулятор ОВЕН ТРМ148 & 10325.0 \\
%\hline{}
%3 & Устройство контроля температуры ОВЕН УКТ38-13 & 8791.0 \\
%\hline{}
%4 & Универсальный регулятор ОВЕН ТРМ138В & 14168.0 \\
%\hline
%\end{tabular}
%\label{table:temperatureSensors}
%\end{table}
%\end{par}
%
%\begin{par}                                                                 
%В отличии от представленной в данном проекте системы контроля и управления 
%технологическим процессом контрольные приборы фирмы <<ОВЕН>> не предусматривают
%возможность дистанционного управления и контроля за ходом выполнения установленных
%режимов обработки продукции. Все операции --- установка и контроль температуры, времени выдержки 
%производятся в ручном режиме оператором термопечей. На основе данных  о затратах на разработку и сопровождение, результатах прогнозирования объема продаж, определяем стоимость одного комплектапрограммного обеспечения. Исходя из потребности на рынке подобных продуктов
%можно предположить, что количество проданных копий составитоколо 300 штук. Стоимость выставляемого на рынок ПП определяется частью стоимости разработки ПП, затрат на сопровождение и прибыли
%организации-разработчика. Стоимость сопровождения остается постоянной для каждой установки
%ПП, а частичная стоимость разработки, приходящаяся на каждый комплект ПП,
%определяется исходя из данных о планируемом объеме установок.
%\end{par}
%
%
%Стоимость программного продукта можно рассчитать, используя соотношение:
%\begin{equation}
%	Цпп = (\delta{}С + Ссопр.) \times{} (1 + Dприб)
%\end{equation}
%
%\begin{equation}
%	\delta{}С = \frac{Спп}{N} \times{} (1 + Нст)
%\end{equation}
%
%\begin{ESKDexplanation}                       
%	\item[где ] $\delta{}С$ --- часть стоимости разработки приходящаяся на одну копию ПП;
%	\item{} Ссопр. --- стоимость сопровождения ПП;
%	\item{} Dприб --- процент прибыли, закладываемый в стоимость;
%	\item{} Нст -  Ставка банковского процента (при условии кредита).
%\end{ESKDexplanation}
%
%$ \delat{}C = \frac{545163.25}{300} = 1817.21$ (руб.).
%Стоимость одного комплекта ПП: $Цпп = (1817.21 +4178.78)*1.15  =  6895.39$ руб.
%
%
%Сумма прибыли составит:
%\begin{equation}
%	Сприб = Спп \times{} Dприб \times{} (1 - Нндс)
%\end{equation}
%подставим все значения и получим: $Сприб = 545163.25 \times{} 0.15 =  81774.49 (руб.)$.
%
%
%
%\subsection{Анализ конкурентноспособности и качества   разрабатываемой системы}
%В рыночной экономике решающим фактором коммерческого успеха товара является конкурентноспособность.
%Это многоаспектное понятие, означающее соответствие товара условиям рынка, конкретным
%требованиям потребителей не только по своим качественным, техническим.
%Экономическим, эстетическим характеристикам, но и по коммерческим условиям его реализации.
%Данная система контроля технологического процесса конкурентноспособна, как и другие аналоги
%на рынке программной продукции.
%
%\subsubsection{Анализ технической  прогрессивности , разрабатываемого ПП}
%Техническая прогрессивность  измеряемых показателей  характеризуется
%коэффициентом технической прогрессивности. Расчет этого коэффициента 
%осуществляется путем сравнения  технического уровня  товара-конкурента
%и разрабатываемого по отношению к эталонному уровню  ПП  данного направления.
%Расчет коэффициента технической прогрессивности производим по формуле:
%
%\begin{equation}
%	Ктп. = \frac{Ктн.}{КтБ}
%\end{equation}
%
%\begin{ESKDexplanation}
%	\item[где ]{} Ктн --- коэффициент технического уровня нового ПП;
%	\item{} КтБ --- коэффициент технического уровня базового ПП.
%\end{ESKDexplanation}
%
%
%\begin{equation}
%	Кт(БН) = \sum{}_{i=1}^{I}\beta{}\frac{П_i(Б.Н.)}{Пэi}
%\end{equation}
%
%(при прямой зависимости)
%
%
%\begin{equation}
%	Кт(бн) = sum{}_{i=1}{I}\beta{}\frac{Пэ_i}{П_i(бн)}
%\end{equation}
%(при обратной зависимости)
%
%\begin{ESKDexplanation}
%	\item{} \beta{}i --- коэффициент весомости. Устанавливается экспертным путем;
%	\item{} Пi(Б.Н) --- численное значение параметра соответственно базового и нового ПП.
%\end{ESKDexplanation}
%
%
%\begin{par}
%Анализируемый ПП технически прогрессивен, если Ктп больше 1. Результаты расчета сводим в таблицу (таб. \ref{table:calcKoefTechProgress}).
%\begin{table}
%\caption{Расчет коэффициента технической прогрессивности разрабатываемого программного продукта}
%\begin{tabular}{|l|c|c|c|c|c|c|c|r|}
%Наимение параметра & \beta{} & Пэ & Пб & Пн & $\frac{Пн}{Пэ}$ & $\frac{Пб}{Пэ}$ & $\beta{}\frac{Пб}{Пэ}$ & $\beta{}\frac{Пн}{Пэ}$ \\
%\hline{}
%Flash-память & 0.4 & 256 & 128 & 256 & 1 & 0.5 & 0.2 & 0.4 \\
%\hline{}
%Рабочее напряжение & 0.4 & 12В & 25В & 3.6В & 3& 0.5 & 0.2 & 132 \\
%\hline{}
%Количество линий ввода-вывода & 0.2 & 64 & 20 & 64 & 1 & 0.31 & 0.06 & 0.2 \\
%\hline{}
%Итого: & & & & & & 1 & 0.46 & 192 \\
%\hline
%\end{tabular}
%\label{table:calcKoefTechProgress}
%\end{table}               
%Находим коэффициент технической прогрессивности: $Ктп = \frac{1.92}{0.46} = 4.1$.
%Полученный коэффициент больше единицы, следовательно ПП технически прогрессивен.
%\end{par}
%
%
%\subsubsection{Анализ изменений функциональных возможностей разрабатываемого программного продукта}
%\begin{par}
%В этом разделе анализируются  эстетические, эргономические,
%экологические параметры , характеризующие функциональные возможности 
%разрабатываемого программного продукта, не имеющие количественного 
%выражения, трудно поддающиеся непосредственной количественной оценке.
%Коэффициент изменения функциональных возможностей рассчитывается
%по  формуле:
%\end{par}
%
%
%\begin{equation}
%	Кфв = \frac{Ан}{АБ}
%\end{equation}
%\begin{ESKDexplanation}
%	\item[где ]{} Ан и АБ --- суммарная бальная оценка не измеряемых параметров соответственно разрабатываемого и базового ПП.
%\end{ESKDexplanation}
%Функциональные возможности нового ПП лучше чем у базового, если КфВ  больше 1.
%
%
%Результаты расчетов сводим в таблицу.
%
%\begin{par}
%\begin{table}
%\caption{Расчет коэффициента изменения функциональных возможностей разрабатываемого ПП}
%\begin{tabular}{|l|p{7cm}|r|}
%Не измеряемые параметры & Характеристика параметра & Характеристика параметра & Бальная оценка & Бальная оценка \\
%базовый
%новый
%базовый
%новый
%\hline{}
%1-wire протокол & Да & Да & 2 & 2 \\
%\hline{}
%Защита доступа & Нет & Да & 1 & 2 \\
%\hline{}
%Учет статистики & Нет & Да & 1 & 2 \\
%\hline{}
%Отражение графических обьектов & Нет & Да & 1 & 2 \\
%\hline{}
%Итого & 5 & & & 8 \\
%\hline
%\end{tabular}
%\label{table:calcKoefTechProgress}
%\end{table}               
%
%Таким образом: $Кфв = 1.6$  больше единицы, функциональные возможности нового ПП лучше чем базового.
%
%
%
%\subsubsection{Анализ соответствия разрабатываемого ПП нормативным показателям}
%Нормативные или, так называемые, регламентируемые параметры характеризуют соответствие разрабатываемого
%ПП международным и национальным стандартам, нормативам, законодательным актам. Для оценки этого
%показателя применяется единичный или групповой показатель- Кнорм. Единичный показатель может принимать
%только одно значение: 1-соответствие и 0- несоответствие. Для разрабатываемого программного продукта Кнорм=1.
%
%
%\subsubsection{Анализ экономических параметров программного продукта}
%На данном этапе осуществляется анализ стоимостных параметров ПП, характеризующих его основные экономические свойства.
%В ходе анализа рассчитывается коэффициент цены потребления по  формуле:
%\begin{equation}
%	Кцп = \frac{Цп^{нов}}{Цп^{баз}}
%\end{equation}
%
%\begin{ESKDexplanation}
%	\item[где ] Цп^{нов}, Цп^{баз} --- цена потребления разрабатываемого и базового программного продукта.
%\end{ESKDexplanation}
%
%Цена потребления представляет собой затраты покупателя на приобретение, доработку.
%А также эксплуатацию анализируемого программного продукта на протяжении периода эксплуатации.
%
%\begin{equation}
%	Цп = Цпр. + Рдор. + Иэкс. \times{} Тн
%\end{equation}
%
%\begin{ESKDexplanation}
%	\item[где ]{} Цп --- цена потребления (руб.);
%	\item{} Цпр. --- цена приобретения копии ПП покупателем (руб.);
%	\item{} Иэкс. --- годовые эксплуатационные издержки потребителя (руб.);
%	\item{} Тн --- нормативный срок эксплуатации (лет);
%	\item{} Рдор. --- затраты па доработку;
%\end{ESKDexplanation}
%
%$Рдор.=Тдор.\times{}Здн$.
%
%\begin{equation}
%Тдор. = Крз \times{} Ксл. \times{} Куч. \times{} Нвр.др.
%\end{equation}
%
%\begin{ESKDexplanation}
%	\item[где ]{} Здн. --- дневная заработная плата исполнителя (Здн.баз = 800 руб.; Здн.нов. = 360.24 руб);
%	\item{} Тдор. --- время доработки. (Тдор.баз = 9.6 дн;   Тдор.нов. = 8 дн.);
%	\item{} Крз --- коэффициент, учитывающий язык программирования (Крз.баз. = 1.0;  Крз.нов. = 1.0);
%	\item{} Ксл --- коэффициент сложности (Ксл. Баз =1.1;  Ксл.нов. = 1.21);
%	\item{} Куч. --- коэффициент участия (Куч.баз =1.1 ;Куч.нов. =1.1 );
%	\item{} Нвр.дор. --- норма времени на доработку (Нвр.дор.баз. = 9.6 дней; Нвр.дор.нов. = 8 дней);
%\end{ESKDexplanation}
%
%Эксплуатационные издержки за весь период эксплуатации:
%\begin{equation}
%	Иэкс =  Рперс +  Рнакл +  Рпр
%\end{equation}
%
%\begin{equation}
%	Рперс = sum{} (n_i \times{} Огод \times{} (1 + \frac{Нотч}{100}) \times{} (1 + \frac{П}{100}))
%\end{equation}
%
%\begin{equation}
%	Рперс = \sum{}(n_i \times{} Огод \times (1 + \frac{Нотч}{100}) \times{} (1 + \frac{П}{100}))
%\end{equation}
%
%\begin{ESKDexplanation}
%	\item[где ]{} n --- численность персонала;
%	\item{} Огод --- среднегодовая заработная плата одного работника;
%	\item{} Нотч. --- процент отчисления на ЕСН;
%	\item{} П --- процент премии;
%	\item{} Ннакл. --- накладные расходы;
%	\item{} Рпр. --- прочие расходы ( 1\% - 3\%  от суммы всех эксплуатационных расходов);
%\end{ESKDexplanation}
%
%\begin{par}
%Все расчеты сводим в таблицы.
%\end{par}
%
%\begin{par}
%\begin{table}
%\caption{Расчет годовых эксплуатационных издержек потребителя ПП}
%\begin{tabular}{|l|p{7cm}|r|}
%Наименование расходов & Базовый & Новый \\
%Расходы на содержание персонала & 567241.0 & 435948.9\\
%Накладные расходы & 587241.0 & 523138.68\\
%Прочие расходы & 11744.82 & 9590.86 \\
%Итого & 1186196.82 & 968678.44 \\
%\end{tabular}
%\label{table:yearEkspIzd}
%\end{table}
%\end{par}
%
%
%\begin{par}
%\begin{table}
%\caption{Расчет цены потребления программного продукта}
%\begin{tabular}{|l|p{7cm}|r|}
%Наименование расходов &  Базовый ПП & Новый ПП \\
%\hline{}
%Продажная цена ПП &  8791.0 & 6895.39 руб. \\
%\hline{}
%Расходы на доработку & 9292.8 & 3835.84 \\
%\hline{}
%Эксплуатационные издержки потребителя за весь период эксплуатации & 9489574.56 & 7749427.52 \\
%\hline{}
%Цена потребления & 9597658.36 & 7759259.35\\
%\hline
%\end{tabular}
%\label{table:zhenaPotreblenija}
%\end{table}
%\end{par}
%
%Исходя из полученных результатов определяем коэффициент цены потребления: $Кцп.=0.816$ меньше единицы,
%следовательно экономические параметры разрабатываемого ПП лучше базовых.
%
%\subsection{Оценка конкурентноспособности программного продукта}
%В целом конкурентноспособность нового ПП по отношению к базовому можно оценить с помощью интегрального
%коэффициента конкурентноспособности, учитывающего все ранее рассчитанные параметры.
%
%\begin{equation}
%Ки = Ктп \times{} Кфв \times{} Кнорм \times{} \frac{1}{Кцп}
%\end{equation}
%
%Подставляем рассчитанные значения коэффициентов: $Ки = 4.1 \times{} 1.6 \times{} 1.0 \times{} \frac{1}{0.816} = 8.01$.
%Полученное значение больше единицы, значит ПП конкурентноспособен.
%
%\subsubsection{Анализ технико - экономических показателей разработки и эксплуатации ПП}
%Ранее рассчитанные показатели сводим в таблицу.
%
%\begin{par}
%\begin{table}
%\caption{Технико-экономические показатели разработки и эксплуатации  ПП}
%\begin{tabular}{|l|r|}
%Показатели & Значение (руб.) \\
%\hline{}
%Затраты на разработку & 545163.25 \\
%\hline{}
%Затраты на сопровождение & 4178.78 \\
%\hline{}
%Продажная цена & 6895.39 \\
%\hline{}
%Годовые эксплуатационные издержки потребителя & 968678.0 \\
%\hline{}
%Цена потребления & 775259.35 \\
%\hline{}
%Интегральный коэффициент   конкурентноспособности ПП & 8.01 \\
%\hline{}
%Коэффициент технической прогрессивности ПП & 4.1 \\
%\hline{}
%Коэффициент функциональных возможностей ПП & 1.6 \\
%\hline
%\end{tabular}
%\label{table:econSostavIspol}
%\end{table}
%\end{par}
%
%\subsection{Выводы}
%Произведенные расчеты показали, что рассматриваемый проект по сравнению с базовым имеет ряд преимуществ,
%которые могут позволить ему занять достойное место на рынке контрольно-измерительных систем.
%\newpage{}

%%\section{Разработка программно-аппаратного комплекса}
%%Расчёт органичивающего ток резистра светодиодной подсветки ЖКИ DST2001PH

\begin{par}
В данном ГЖКИ используются четыре параллельно соединённых светодиода белого свечения со следующими характеристиками:
\begin{itemize}
    \item{}Допустимый ток --- 15мА
    \item{}Прямое падение напряжения --- 3.2В
\end{itemize}

Таким образом ограничивающий ток резистор должен иметь сопротивление не менее: \\
$$R = (U_n - U_p) / I$$ \\
$$R = (3.3 - 3.2) / 0.015 = 6.6$$ \\
Ближайшее значение стандартного резистра будет 6.8Ом.

\end{par}


Управление яркостью светодиодов подсветки ЖКИ производиться методом широтно-импульсной модуляции.

%%\subsection{Управление яркостью светодиодной подсветки ЖКИ}

\begin{par}
Одним из основных параметров светодиодов является: яркость — величина,
равная отношению силы света к площади светящейся
поверхности, измеряемая в канделах на квадратный метр. \\

Спектральная характеристика светодиода выражает зависимость интенсивности
излучения от длины волны излучаемого света и дает представление о цвете
свечения светодиода. \\

Длина волны излучаемого света определяется разностью
энергий двух энергетических уровней, между которыми происходит переход
электронов на излучательном этапе процесса рекомбинации и определяется
исходным полупроводниковым материалом и легирующими примесями.
\end{par}

\subsubsection{Уменьшение яркости светодиода методом ШИМ}

\begin{par}
Наиболее простой способ уменьшения яркости светодиода - изменение прямого
тока или напряжения [http://www.e-neon.ru/user_img/pages/Dimming_InGaN_rus.pdf].
Но необходимо учитывать, что изменение этих параметров будет влиять на
длину волны излучаемого света, причём чем больше длинна волны, тем сильнее
выражен этот эффект.
Помимо тока, на длину волны оказывает влияние так же и температура.
Но это влияние не столь существенно и может игнорироваться.
\end{par}

\begin{par}
В схемах с использованием ШИМ через светодиод проходит последовательность импульсов.
Если частота следования импульсов более 200Гц, человеческий
глаз, обладающий инерционностью[TODO], будет ощущать непрерывное свечение
светодиода. Изменняя длительность и скважность импульса можно добиться того,
что зрение будет интегрировать и интерпретировать отдельные световые импульсы
как изменение силы света.
\end{par}

\begin{figure}[h]
	\center{\includegraphics[bb=0 0 507 396]{led_pwm.png}}
	\caption{Управление яркосьтю светодиода ШИМ}
	\label{img:led_pwm}
\end{figure}

\begin{par}
При неизменном токе яркость свечения зависит от скважности
следующим образом: \\
    $$ D_2 < D_1 < D_3 $$

При этом, визуальная сила света будет меняться линейно при соответствующем
линейгном изменении скважности [TODO: откуда я взял эту информацию].
\end{par}

[TODO: А так же какова будет зависимость в чиселках]



%% Библиография
\begin{thebibliography}{99}
\bibitem{lessing} Lawrence Lessig. ''Free culture: how big media uses technology and the law to lock down
culture and control creativity'', ''Rebuilding Free Culture: One Idea'', KF2979.L47 200, ст. 282
\bibitem{electric} http://www.staticfreesoft.com/electric.html
\bibitem{geda} http://en.wikipedia.org/wiki/Geda
\bibitem{avrref} Atmel corp., ''XMEGA A MANUAL'', ст. 129, ст. 190, ст. 317
\bibitem{avrxm} http://atxmega.narod.ru/, структурное описание
\bibitem{avrev} Евстифеев А.В. ''Микроконтроллеры AVR семейства Mega''., М.: Додэка-XXI, 2007
\bibitem{technob} Одинцов В., ''Профессиональное программирование. Системный подход. 2 изд.'', BHV-СПб 2006, ст. 405
\bibitem{erlang} Francesco Cesarini, Simon Thompson, ''Erlang Programming'', 2009
\bibitem{display} Displaysun, ''Specification for TFT LCD module DST2001PH'', DST2001PH REV B, 2008/02/23
\bibitem{ili9320} ILITek., ''a-Si TFT LCD Single Chip Driver 240RGBx320 Resolution and 16.7M color''
\bibitem{avradc} Atmel corp., ''AVR1300: Using the XMEGA ADC'', Rev. 8033B-AVR-04/08
\bibitem{mc34119} Freescale Semiconductor, ''Low power audio amplifier 34119. Technical Data.'', Rev. 3.0, 12/2006
\bibitem{avrdac} Atmel corp., ''AVR1301: Using the XMEGA DAC'', Rev. 8033B-AVR-04/08
\bibitem{avrspi} Atmel corp., ''AVR1309: Using the XMEGA SPI'', Rev. 8057A-AVR-02/08
\bibitem{enc28j60} Microchip inc., ''ENC28J60 Data Sheet. Stand-Alone Ethernet Controller with SPI Interface''
\end{thebibliography}
\end{document}

