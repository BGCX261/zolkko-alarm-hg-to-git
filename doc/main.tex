%%
%% Автоматизированная разработка програмно-аппаратного комплекса <<цифровые часы с будильником>>
%%

\documentclass[russian,utf8,pointsection]{eskdtext}
\usepackage[utf8]{inputenc}
\usepackage{eskdchngsheet}

\usepackage[ unicode ]{ hyperref }
\usepackage{amstext}
\usepackage{ amsmath }
\usepackage{ listings }
\usepackage{tikz}
\graphicspath{{/Users/zolkko/Projects/zolkko-alarm/doc/imgs/}}

\ESKDdepartment{Федеральное агентство по образованиюГосударственное образовательное учреждениеВысшего профессионального образования<<Воронежский государственный технический университет>>(гоувпо <<ВГТУ>>)}
\ESKDcompany{}
\ESKDclassCode{31 1398}
\ESKDdocName{Пояснительная записка}
\ESKDsignature{TODO: ШИФР}


%% Разработка програмно-аппаратного комплекса <<Будильник>>
\ESKDtitle{Отчт по преддипломной практике}

\ESKDauthor{Анисимов~А.~Н.}
\ESKDtitleApprovedBy{Руководитель}{Барабанов~В.~Ф.}
%%\ESKDtitleAgreedBy{XXX}{XXX~X.~X.}
\ESKDtitleDesignedBy{студент группы ВМ-072}{Анисимов А.Н.}
%%\ESKDtitleDesignedBy{XXX}{XXX~X.~X}




\begin{document}
\maketitle
%Задание на курсовой проект%Лист замечаний руководителя
\tableofcontents

\tikz{\draw (-1,-1) -- (1,1); \path[fill=green!80!blue,draw=red] (0,0) circle (7mm);}

\newpage
\section*{ВВЕДЕНИЕ}
\addcontentsline{toc}{section}{ВВЕДЕНИЕ}
\begin{par}
На сегодняшний день при проектировании систем промышленной автоматизации и устройств
бытового применения, перед проектировщиками и разработчиками встают вопросы не только технического
характера, но и вопросы экономической целесообразности применения тех или иных решений.
То есть при их решении необходимо учитывать не только системные характеристики применяемых для
реализации конечных устройств технологий, но и искать компромис с их стоимостью. При этом
на конечную стоимость изделия будут влиять цена применённых схемотехнических решений,
время затраченное на проектирование и реализацию устройства, цена применяемых средства автоматизации
и цена специалистов проектировщиков и разработчиков.
\end{par}

\begin{par}
Немаловажным при проектировании устройств является учёт стремления
современной европейской культуры не только к открытым, но и полностью свободным системам,
зачастую обладающим более качественными системными и
потребительскими характеристиками и способствующими общему\\*
научно-техническому прогрессу\cite{lessing}.
\end{par}
\newpage{}



\section{Обзор САПР}
\subsection{Общие сведения о САПР}
\begin{par}
САПР --- Система автоматизированного проектирования --- автоматизированная система, реализующая
информационную технологию выполнения функций проектирования, представляет собой
организационно-техническую систему, предназначенную для автоматизации процесса проектирования,
состоящую из персонала и комплекса технических, программных и других средств
автоматизации его деятельности.
\end{par}

\begin{par}
На некотором этапе своего развития системы проектирования претерпели качественное изменение.
Оно было связано с тем, что САПР из набора каким-то образом связанных между собой прикладных программ начали превращаться в мобильные и стройно организованные системы, способные к настройке на особенности предметной области и на требования конечного пользователя, допускающие расширение функциональных возможностей за счёт сравнительно несложного подключения новых прикладных программных модулей и обеспечивающих поддержку групповой разработки сложных схем коллективом проектировщиков.
\end{par}

\begin{par}
Особенное значение САПР приобрели в микроэлектронике, поскольку современная радиоэлектронная аппаратура базируется на применении сверхбольших интегральных схем, разработка которых без применения САПР невозможна или крайне затруднительна.
\end{par}

\begin{par}
Специализированные САПР для разработки электронных устройств и печатных плат получили название Electronic Design Automation - EDA, автоматизация проектирования электронных приборов. Комплексы такого типа зачастую позволяют создавать принципиальные электрические схемы схемы с помощью графического интерфейса, создавать и модифицировать  базу радиоэлектронных компонентов, проверять целостность сигналов на ней и проводить аналоговое и цифровое моделирование разрабатываемого устройства ещё на этапе проектирования.
\end{par}

\begin{par}
Ниже перечислены примеры EDA программ:
	\begin{itemize}
		\item{}P-CAD
		\item{}OrCAD
		\item{}Electric
		\item{}Proteus
		\item{}KiCad
		\item{}gEDA
		\item{}Eagle EDA
	\end{itemize}
\end{par}


\subsection{Обзор Eagle EDA}
\begin{par}
EAGLE это лёгкий в использовании, но достаточной мощный EDA пакет, разрабатываемый немецкой компанией CadSoft.
В состам системы входят:
\begin{enumerate}
	\item{}Layout Editor - пограмма для проектирования печатных плат. 
		\begin{itemize}
			\item{}Максимальная рабочая поверхность 1.6 x 1.6м;
			\item{}разрешение - 1/10,000мм (0.1 микрон);
			\item{}до шестнадцати сигнальных слоёв;
			\item{}большая библиотека компонентов;
			\item{}copper pouring - области на печатной плате заполненные  медью, что часто используется для создания областей <<земли>> и уменьшения необходимого травильного вещества при производстве;
			\item{}встроенная ДРЦ проверка.
		\end{itemize}
	
	\item{}Schematic Editor - программа для создания принципиальных электрических схем.
		\begin{itemize}
			\item{}до 999 листов на одну схему;
			\item{}проверка эллектичских правил;
			\item{}обмен вентелей и пинов;
			\item{}возможность создания печатной платы из схемы одной командой.
		\end{itemize}
	\item{}Autorouter - программа автоматической трассировки.
		\begin{itemize}
			\item{}ripup-and-retry трассировщик - на первом проходе выполняется соединение абсолютно всех проводников без обращения внимания на возможные конфликты, заключающиеся в пересечении проводников на одном слое и нарушении зазоров. На каждом последующем проходе автотрассировщик пытается уменьшить количество конфликтов, разрывая и вновь прокладывая связи;
			\item{}до шестнадцати сигнальных слоёв;
			\item{}стратегия трассировки может быть подстроена заданием весовых коэфициентов.
		\end{itemize}
\end{enumerate}
Все эти программы встроены в один пользовательский интерфейс, таким образом нет необходимости в предварительном конвертировании нет-листов.
\end{par}

%!TEX root = /Users/zolkko/Projects/zolkko-alarm/doc/main.tex
\section{Выбор инструментальных средств разработки}

\subsection{Выбор микроконтроллера}
Основными требованиями предъявляемыми мной к центральному вычислительному устройству
создаваемого устройства является:
\begin{itemize}
	\item{} невысокая стоимость;
	\item{} низкое энергопотребление;
	\item{} в число поддерживаемой переферии должен присутствовать ЦАП;
	\item{} аппаратная поддержка SPI;
	\item{} поддержка производителем этого рода устройств;
	\item{} доступность устройства;
	\item{} <<сильное>> сообщество разработчиков под эту архитектуру;
	\item{} наличие литературы и справочных материалов.
\end{itemize}


По всем этм пунктам идеально подходит микроконтрллер семейства XMega компании ATmel --- atxmega32a4.
Этот микроконтроллер полностью отвечает минимальным требованиям. В целях достижения максимальной
производительности и параллелизма у микроконтроллеров AVR используется
Гарвардская\\*
архитектура (рис. \ref{img:avr_arch}) с отдельными памятью и шинами программ и данных. Инструкции,
хранящиеся в памяти программ, выполняются на одноуровневом конвейере. Это означает, что
во время выполнения одной инструкции выполняется предварительная выборка из памяти программ
следующей инструкции. Данная концепция делает возможным выполнение по одной инструкции за
каждый цикл синхронизации.

\begin{figure}[ht]
	\center{\includegraphics[bb=0 0 590 460, clip, scale=0.6]{avr_arch.png}}
	\caption{Архитектура AVR}
	\label{img:avr_arch}
\end{figure}


Так же в случае необходимости может быть заменён на более мощьный микроконтроллер того же семейства.
Для обеспечения этого функционала компания Atmel предприняла ряд шагов:
\begin{itemize}
    \item{} Была реорганизована область памяти, таким образом, чтобы все связанные между
            собой регистры располагались в памяти строго последовательно;
    \item{} была переписана стандартная библиотека C/C++, так чтобы обеспечивалась
            максимальная переносимость кода написанного для одного микроконтроллера
            на другой микроконтроллер того же семейства.
\end{itemize}

Микроконтроллеры серии XMega обладают следующими характеристиками:
\begin{itemize}
	\item{} 8/16-битное высокопроизводительное RISC ЦПУ AVR;
	\item{} 138 инструкций;
	\item{} аппаратное умножающее устройство;
	\item{} 32 8-битных регистра, напрямую подключенные к АЛУ
	\item{} прямая адресация до 16 Мбайт памяти программ и 16 Мбайт памяти данных;
	\item{} полная поддержка 16/24-битного доступа к 16/24-битным регистрам ввода-вывода;
	\item{} эффективная поддержка 8-, 16- и 32-битных арифметических инструкций;
	\item{} защита от изменения настроек критических функций системы.
\end{itemize}

Помимо этого значительными изменениями нового семейства микроконтроллеров XMega по сравнению
с предыдущим семейством Mega стало\cite{avrref}:
\begin{itemize}
    \item{} Свехмалое энергопотребление обеспечиваемое технологией picoPower второго поколения.
    picoPower позволяет еще больше улучшить эффективность использования батарейного источника.
    То, что микроконтроллеры гарантируют нормальное функционирование при напряжении 1.6 В означает,
    что, например, в составе мобильных телефонов, они могут быть запитаны от стабилизированного
    источника напряжением 1.8В 10\%, тем самым, позволяя снизить себестоимость системы и увеличить
    длительность работы от батарейного источника. 
    \item{} Event System ---  позволяет организовать передачу данных между встроенными периферийными
    устройствами без вмешательства ЦПУ или использования ПДП.
    Этим гарантируется 100\% предсказуемость и малое время реагирования.
    До 8 одновременных событий или условий прерывания в периферийных устройствах могут автоматически
    инициировать действия в других периферийных устройствах \cite{avrxm}.
    \item{} 12 битные АЦП и ЦАП. Для обеспечения высокоточной обработки аналоговых сигналов в состав 
    микроконтроллеров XMEGA интегрированы 12-битные преобразователи аналоговых сигналов. АЦП
    микроконтроллеров XMEGA могут достигать частоты преобразования до 2 МГц, что делает их самыми
        быстродействующими и точными на фоне АЦП обычных микроконтроллеров.
        Поскольку микроконтроллеры XMEGA также интегрируют два 12-битных ЦАП на частоту преобразования
        до 1 МГц и четыре усовершенствованных аналоговых компаратора, это делает их лидерами по
        степени интеграции компонентов для аналоговой обработки.
    \item{} Расширена функциональноть портов ввода-вывода общего назначения. Каждый из потров
    ввода-вывода может быть сконфигурирован как источник внешнего прерывания, при этом событие внешнего прерывания
    может минуюя ЦПУ запускать обработку данных используя модуль ПДП. \\*
    Новыми доступными состояниями портов ввода-вывода стали totem-pull, bus-keeper, wired-or, wired-and.
\end{itemize}


\subsection{Язык программирования микроконтроллера}
\begin{par}
Микропрограммы для микроконтроллеров XMega можно писать на нескольких языках программирования.
Сама компания Atmel предоставляет для программирвания своих устройств два средства разработки:
	\begin{enumerate}
		\item{}AVR assembler --- язык низкого уровня, транслирует исходный код пользовательской
                программы в объектны и исполняемый код. Применение AVR assembler позволяет
                добиться от пограммы наибольшей производительности.
		\item{}AVR-GCC --- кодогенератор и набор дополнительных утилит для Gnu C Compiller от
                компании Atmel, активно поддерживаемый сообществом разработчиков, а так же
                входящий в комплект среды разработки AVR Studio, AVR32 Stdio и
                распростроняемый на правах свободного программного обеспечения.\\*
                В состав AVR-GCC входят компиляторы языка C и C++.
	\end{enumerate}
\end{par}

\begin{par}
В начале 2011 года компания предоставила новую версию своей интегрированной системы
разработки - AVR Studio 5. В отличии от предшественника, новая система позволяет разрабатывать
программное обеспечение для всех семейств микроконтроллеров AVR используюя единую среду разработки.
\end{par}

\begin{par}

\begin{figure}[ht]
	\center{\includegraphics[bb=0 0 800 700, clip, scale=0.5]{avrstudio5.png}}
	\caption{AVR Studio 5}
	\label{img:avr_studio}
\end{figure}


Новая среда разработки AVR Studio 5 (рис. \ref{img:avr_studio}) основана на базе лидирующего продука --- Microsoft Visual Studio.
И так же как и предшественники, новая версия AVR Studio 5  распространяется бесплатно.
В качестве компилятора и кодогенератора в AVR Studio 5 по прежнему используется AVR-GCC,
что позволяет продолжать использовать уже существующие наработки.
\end{par}

\begin{par}
Считается\cite{avrev}, что для разработки эффективного встраиваемого приложения для 8-битных микроконтрллеров
AVR наиболее целесообразно применять язык ассемблера или в случае, когда необходимо добиться
большей читаемости и поддерживаемости приложения --- язык С.
Однако, лидирующая группа японских производителей ЦПУ, возглавляемая NEC, Hitachi, Fujitsu и Toshiba,
разработала специализированных диалект языка C++ --- Embedded C++ (EC++), позволяющий применять
C++ для встраиваемых систем. Основная цель разработки --- перенос существующих методологий и шаблонов
программирования C++ в облать применения встраивамых систем.
При этом программы код генерируемых компиляторами EC++ может даже дать приемущества по сравлению
с языком C. Так как сама структура языка C++ позволяет минимизировать объём кода и одновременно
повышая его эффективность.
\end{par}

\begin{par}
Однако, при выборе в качестве языка программирования C++, из текущей реализации поставляемой с AVR-GCC,
необходимо учитывать некоторые ограничения:
\begin{itemize}
    \item{} Отсутствует стандартная библиотека C++.
    \item{} EC++ --- это только подмножество C++, по этому некоторые особенности языка были
        убраны из стандарта:
        \begin{itemize}
            \item{} Множественное наследование;
            \item{} базовые виртульаные классы;
            \item{} информация времени исполнения;
            \item{} приведение типов (static\_cast, dynamic\_cast, reinterpret\_cast и const\_cast);
            \item{} квалификаторы типов;
            \item{} пространства имён;
            \item{} исключения;
            \item{} шаблоны.
        \end{itemize}
\end{itemize}
\end{par}



\subsection{Выбор языка программирования сетевого сервиса}

При проектировании сетевого сервиса от инструментального средства ожидается, чтобы оно
отвечало следующим требованиям\cite{technob}:

\begin{enumerate}
	\item{} Инструментальное средство должно быть высокого уровня --- язык высокого уровня,
            освобождает разработчика от рутинной работы вроде ручного выделения и освобождения
            памяти, и позволяет ему сфокусироваться на оперировании абстракциями предметной области.
	\item{} Язык должен минимизировать количество ошибок которые может допустить программист в
             процессе разработки системы.
	\item{} Высокая степень параллелизма --- необходима возможность обслуживать тысячи клиентов
            одновременно.
	\item{} Отказоустойчивость --- телеком-системы слишком масштабны, чтобы самому разработчику
             имело смысл даже пытаться предусмотреть все возможные ошибки.
	\item{} Возможность обновления кода сервиса без останова выполнения программы.
	\item{} Наличие обширной системной библиотеки,  а так же предопределённых каркасов
            проектов --- это избавляет разработчика от необходимости реализовывать типовые
            решения самостоятельно, при этом тратя время на тестирование и отладку системы.
            Каркас проекта задаёт общую структуру создаваемой системы, что позволяет ещё
            на этапе проектирования системы оценивать её будущие качества.
\end{enumerate}

\begin{par}
Один из немногих существующих и поддерживаемых на сегоднешний день языков программирования
отвечающих всем этим требованиям является Erlang и платформа Open Telecom Platform (OTP).
\end{par}

\subsubsection{Обзор языка Erlang и платформы OTP}

\begin{par}
В середине 1980-х, Ericsson Computer Science Laboratory было дано задание --- исследовать языки программирования подходящие для разработки телекоммуникационных продуктов нового поколения. Джой Армстронг(Joe Armstrong), Роберт Вирдинг (Robert Virding) и Майк Вильямс (Mike Williams) под руководством Брайна Декера (Bjarne Dacker) потратили два года на прототипирование телекоммуникацинного приложения поочерёдно  используя все доступные на тот момент языки и системы программирования. В результате, несмотря на то, что многие языки программирования обладали интересными и подходящими свойствами, ни один из них не удовлетворял всех их требованиям. В результате они приняли решение создать их собственный язык программирования.
\end{par}

\begin{par}
Erlang был создан под влиянием функциональных языков таких как  ML и Miranda, параллельных языков ADA, Modula и Chill, и языка логического программирования Prolog. Erlang так же унаследовал некоторые черты таких язков как Smalltalk, проприетарных языков Ericsson EriPascal и PLEX \cite{erlang}.
\end{par}

\begin{par}
Используя построенную на Prolog виртуальную машину Erlang (VM), лаборатория потратила  четыре года на
прототипирование телекоммуникационного приложения с применением и постоянными доработками новго языка.
Именно из-за применения метода проб и ошибок язык Erlang стал таким, каким он является сейчас. В 1999
Майк Вильямс переписал на Си виртуальную машину, и годом позже, на этом языке был выпущен первый
коммерческий продукт. 
\end{par}

\begin{par}
История создания языка Erlang важна для понимания его философии, так как в отличии от других языков, которые находили свою нишу уже после разработки и распространения, Erlang изначально создавался для решения конкретных задач бизнеса. Он создавался под задачи построения распределённых, отказоустойчивых систем реального времени массового обслуживания.
\end{par}

\begin{par}
Так как такие области как системы поддержки продаж, банковские системы, системы компьютерной телефонии, системы интеграции уровня предприятий зачастую предъявляют к своему программному обеспечению аналогичные требования, то Erlang нашёл своё применение и в них.
\end{par}

Подтверждением применимости Erlang в этих областях могут служить факты использования этого языка в
проектах компаний:
\begin{itemize}
	\item{} Amazon --- использует Erlang для реализации SimpleDB, предоставления системы
        управления базами данных как части Amazon Elastic Compute Cloud (EC2).
	\item{} Yahoo! --- использует Erlang для реализации своего сервиса социальных закладок,
    который обслуживает более 5 милионов пользователей и 150 милионов URL.
	\item{} T-Mobile --- использует Erlang в их SMS и авторизирующей системах.
	\item{} Motorola --- использует Erlang в системе обработки звонков системы обслуживания клиентов.
	\item{} Ericsson --- использует Erlang для поддержки узлов используемых в GPRS и 3G
    мобильных сетях по всему миру.
	\item{} Facebook и Yandex --- используют написанный на Erlang сервер мгновенных
    сообщений Ejabberd.
\end{itemize}


\include{development}

\newpage
\section{Разработка программно-аппаратного комплекса}
Выбор ЖКИ

В проектируемом изделии используется графический жидкокристаллический индикатор
DST2001PH компании Shenzhen Display Optech Technology Co.

В качестве устройства отображения информации в проекте используется ЖКИ DST2001PH.
Выбора именно этого жидкокристаллического индикатора определяется наилучшим сочетанием,
с одной стороны, эксплуатационных характеристих для пользователя системы, и с другой - 
удобством его использования в изделии.

Основные характеристики устройства:
\begin{itemize}
	\item{} Тип --- TFT 
	\item{} Количество точек --- 240x320
	\item{} Размер точки --- 0.18x0.18 мм
	\item{} Размер активной области --- 57.6x43.2 мм
	\item{} Управляющая ИМС --- ILITEK9320 вкючённая в режиме 16bit RGB
	\item{} Типовое питающее напряжение --- 3.2В
\end{itemize}

Дополнительными характеристиками повлиявшими на выбор именно этого устройства
стали:
\begin{itemize}
	\item{} наличие в устройстве четырёхпроводного резистивного датчика
		прикосновения ---  это позволяет выполняемому изделию 
		исключить из своей конструкции необходимость использовать
		большое количество механических кнопок
	\item{} многофункциональность устройства ЖКИ
\end{itemize}



Расчёт органичивающего ток резистра светодиодной подсветки ЖКИ DST2001PH

\begin{par}
В данном ГЖКИ используются четыре параллельно соединённых светодиода белого свечения со следующими характеристиками:
\begin{itemize}
    \item{}Допустимый ток --- 15мА
    \item{}Прямое падение напряжения --- 3.2В
\end{itemize}

Таким образом ограничивающий ток резистор должен иметь сопротивление не менее: \\
$$R = (U_n - U_p) / I$$ \\
$$R = (3.3 - 3.2) / 0.015 = 6.6$$ \\
Ближайшее значение стандартного резистра будет 6.8Ом.

\end{par}


Управление яркостью светодиодов подсветки ЖКИ производиться методом широтно-импульсной модуляции.

\subsection{Управление яркостью светодиодной подсветки ЖКИ}

\begin{par}
Одним из основных параметров светодиодов является: яркость — величина,
равная отношению силы света к площади светящейся
поверхности, измеряемая в канделах на квадратный метр. \\

Спектральная характеристика светодиода выражает зависимость интенсивности
излучения от длины волны излучаемого света и дает представление о цвете
свечения светодиода. \\

Длина волны излучаемого света определяется разностью
энергий двух энергетических уровней, между которыми происходит переход
электронов на излучательном этапе процесса рекомбинации и определяется
исходным полупроводниковым материалом и легирующими примесями.
\end{par}

\subsubsection{Уменьшение яркости светодиода методом ШИМ}

\begin{par}
Наиболее простой способ уменьшения яркости светодиода - изменение прямого
тока или напряжения [http://www.e-neon.ru/user_img/pages/Dimming_InGaN_rus.pdf].
Но необходимо учитывать, что изменение этих параметров будет влиять на
длину волны излучаемого света, причём чем больше длинна волны, тем сильнее
выражен этот эффект.
Помимо тока, на длину волны оказывает влияние так же и температура.
Но это влияние не столь существенно и может игнорироваться.
\end{par}

\begin{par}
В схемах с использованием ШИМ через светодиод проходит последовательность импульсов.
Если частота следования импульсов более 200Гц, человеческий
глаз, обладающий инерционностью[TODO], будет ощущать непрерывное свечение
светодиода. Изменняя длительность и скважность импульса можно добиться того,
что зрение будет интегрировать и интерпретировать отдельные световые импульсы
как изменение силы света.
\end{par}

\begin{figure}[h]
	\center{\includegraphics[bb=0 0 507 396]{led_pwm.png}}
	\caption{Управление яркосьтю светодиода ШИМ}
	\label{img:led_pwm}
\end{figure}

\begin{par}
При неизменном токе яркость свечения зависит от скважности
следующим образом: \\
    $$ D_2 < D_1 < D_3 $$

При этом, визуальная сила света будет меняться линейно при соответствующем
линейгном изменении скважности [TODO: откуда я взял эту информацию].
\end{par}

[TODO: А так же какова будет зависимость в чиселках]



%\newpage
%\begin{ thebibliography }{99}
%\bibitem{1} Автор1~И.~О. , Автор2~И.~О. Название первой книги. "−−− М.: Название	первого	издательства ,	1999.	"−−−	543~с .
%\bibitem{2} Автор3~И.~О. , Автор4~И.~О. Название второй книги. "−−− К.: Название	второго	издательства ,	1999.	"−−−	543~с .
%\end{ thebibliography }

%\ESKDappendix{рекомендуемое}{Исходный код этого документа} \ lstset {columns=fixed , language=[LaTeX]TeX,
%% basicstyle=\small , breaklines=true } \ l s t i n p u t l i s t i n g { general . tex }

\end{document}
