\subsection{Разработка микропрограммы управляющего устройства}
Микропрограмма управляющего устройства должна выполнять следующие функции:
\begin{itemize}
    \item{} Вывод текущих показаний датчиков
    \item{} адаптивный алгоритм регулирования
    \item{} TODO:
\end{itemize}

TODO: class diagram, short description

\subsubsection{Методика понижения потребления тока устройством управления}
TODO: describe picoPower technology. \\
TODO: describe sleep walk ADC / DAC solution.

\subsubsection{Разработка адаптивного алгоритма регулятора температуры}
В качестве нагревательного элменета коптильных камер используются тепловые тенты,
включаемые в эллектрические сети общего назначения. При одна из главных проблем
которую необходимо решить при управлении таким нагревательным прибором связана
с инерционым характером роста температуры после отключения управляющего воздействия.
Из-за чего возникает длительный переходный процесс приводящий к перерегулированию.

\subsubsection{Алгоритм управления силовыми механизмами подачи влаги и дыма в коптильную камеру}

\subsubsection{Фиксация статуса выполняемой операции на удалённом конроллирующем сервере}
TODO: point that mac layer has been implemented already
TODO: describe UDP \\

\subsubsection{Модуль взаимодействия с внешней памятью}

\subsubsection{Модуль ввода-вывода управляющего устройства}

\subsubsection{Модуль сетевго взаимодействия}
Для осущетвления сетевого взаимодйствия управляющего устройства и
контроллирующего сервиса используется сетевой интерфейс построенны
на основе микроконтроллера компании MicroChip enc28J60.

Контроллер ENC28J60 это IEEE 802.3 совместимый контроллер с интегрированными
MAC и 10BASE-T PHY модулями. Этот контроллер сдержит 8 килобайт буферной памяти
с аппаратной поддержкой FIFO и CRC денераци.
Микроконтроллер поддерживает Unicast, Multicast и Broadcast пакеты, и программируемый
фильтр пакетов.
Аппаратный модуль вычисления CRC позволяет производить вычисление контролных сумм
используемых в большинстве сетевых протоколов.
Контроллер аппаратно управляет двумя программируемыми светодиодами для
отображения статуса подключения и сетевой активности.
По умолчанию зелёный светодиод обозначение присутствие сети, а
жёлтый --- сетевую активность.

Взаимодействие с конртоллером осуществляется по стандартному
протоколу SPI.


\subsubsection{Программирование микроконтрллера}

