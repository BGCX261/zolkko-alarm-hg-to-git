\section{Обзор САПР}
\subsection{Общие сведения о САПР}
\begin{par}
САПР --- Система автоматизированного проектирования --- автоматизированная система, реализующая
информационную технологию выполнения функций проектирования, представляет собой
организационно-техническую систему, предназначенную для автоматизации процесса проектирования,
состоящую из персонала и комплекса технических, программных и других средств
автоматизации его деятельности.
\end{par}

\begin{par}
На некотором этапе своего развития системы проектирования претерпели качественное изменение.
Оно было связано с тем, что САПР из набора каким-то образом связанных между собой прикладных программ начали превращаться в мобильные и стройно организованные системы, способные к настройке на особенности предметной области и на требования конечного пользователя, допускающие расширение функциональных возможностей за счёт сравнительно несложного подключения новых прикладных программных модулей и обеспечивающих поддержку групповой разработки сложных схем коллективом проектировщиков.
\end{par}

\begin{par}
Особенное значение САПР приобрели в микроэлектронике, поскольку современная радиоэлектронная аппаратура базируется на применении сверхбольших интегральных схем, разработка которых без применения САПР невозможна или крайне затруднительна.
\end{par}

\begin{par}
Специализированные САПР для разработки электронных устройств и печатных плат получили название Electronic Design Automation - EDA, автоматизация проектирования электронных приборов. Комплексы такого типа зачастую позволяют создавать принципиальные электрические схемы схемы с помощью графического интерфейса, создавать и модифицировать  базу радиоэлектронных компонентов, проверять целостность сигналов на ней и проводить аналоговое и цифровое моделирование разрабатываемого устройства ещё на этапе проектирования.
\end{par}

\begin{par}
Ниже перечислены примеры EDA программ:
	\begin{itemize}
		\item{}P-CAD
		\item{}OrCAD
		\item{}Electric
		\item{}Proteus
		\item{}KiCad
		\item{}gEDA
		\item{}Eagle EDA
	\end{itemize}
\end{par}


\subsection{Обзор Eagle EDA}
\begin{par}
EAGLE это лёгкий в использовании, но достаточной мощный EDA пакет, разрабатываемый немецкой компанией CadSoft.
В состам системы входят:
\begin{enumerate}
	\item{}Layout Editor - пограмма для проектирования печатных плат. 
		\begin{itemize}
			\item{}Максимальная рабочая поверхность 1.6 x 1.6м;
			\item{}разрешение - 1/10,000мм (0.1 микрон);
			\item{}до шестнадцати сигнальных слоёв;
			\item{}большая библиотека компонентов;
			\item{}copper pouring - области на печатной плате заполненные  медью, что часто используется для создания областей <<земли>> и уменьшения необходимого травильного вещества при производстве;
			\item{}встроенная ДРЦ проверка.
		\end{itemize}
	
	\item{}Schematic Editor - программа для создания принципиальных электрических схем.
		\begin{itemize}
			\item{}до 999 листов на одну схему;
			\item{}проверка эллектичских правил;
			\item{}обмен вентелей и пинов;
			\item{}возможность создания печатной платы из схемы одной командой.
		\end{itemize}
	\item{}Autorouter - программа автоматической трассировки.
		\begin{itemize}
			\item{}ripup-and-retry трассировщик - на первом проходе выполняется соединение абсолютно всех проводников без обращения внимания на возможные конфликты, заключающиеся в пересечении проводников на одном слое и нарушении зазоров. На каждом последующем проходе автотрассировщик пытается уменьшить количество конфликтов, разрывая и вновь прокладывая связи;
			\item{}до шестнадцати сигнальных слоёв;
			\item{}стратегия трассировки может быть подстроена заданием весовых коэфициентов.
		\end{itemize}
\end{enumerate}
Все эти программы встроены в один пользовательский интерфейс, таким образом нет необходимости в предварительном конвертировании нет-листов.
\end{par}
