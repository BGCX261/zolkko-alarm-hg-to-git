%!TEX root = /Users/zolkko/Projects/zolkko-alarm/doc/main.tex
\section{Безопасность и экологичность в дипломном проекте}

\subsection{Вредные производственные факторы на рабочем месте оператора ПЭВМ}
Под рабочим местом в контексте данной работы понимается участок рабочего
помещения, оборудованный комплексом средств вычислительной техники, в
пределах которого постоянно или временно пребывает оператор ПЭВМ в прцессе трудовой деятельности.


В настоящем проекте рабочее место оператора ПЭВМ расположено в 
специально отведенном помещении .Размеры помещения : длина — 4 метра;
ширина- 3 метра ; высота потолков — 3 метра. Рабочее место включает в себя:
системный блок XTEND-11417 Intel Celeron 430;  монитор LG Flatron F700B
(электронно-лучевой);  принтер  Panasonic KX P6500 ; клавиатура A4 Terh 
G9300 Black. Кроме этого в помещении установлен оконный кондиционер
LW 05 LC.

На пользователя ПЭВМ одновременно воздействуют несколько вредных факторов.
Их источниками являются не только монитор,но и другие модули вычислительного устройства
и факторы окружающей среды. 

Ниже приведены основные вредные факторы.
\begin{par}
		1. Электромагнитные поля и излучения.
		Дисплей по типу электронно-лучевой трубки является источником мягкого рентгеновского,
		видимого инфракрасного, ультрафиолетового, низкого и сверх
		частотного электромагнитного излучений;
\end{par}
\begin{par}
		2. Пренапряжение зрительного анализатора.
		В обыденной жизни человек имеет дело с низкой фоновой яркостью при высокой
		контрастности предметов, и к этому в процессе эволюции приспособился наш глаз.
		При работе за дисплеем глаз считывает информацию  с излучателя, имеющего
		высокую фоновую яркость при низкой контрастности обьектов различения.
		При уменьшении яркости экрана контрастность падает, поэтому для обеспечения
		оптимального контраста необходимо повышать яркость, что увеличивает интенсивность
		вредных излучений и утомляет глаза.
\end{par}
\begin{par}
	3.	Избыточность энергетических потоков на орган зрения в оптическом диапазоне.
		Результаты исследования НИИ глазных болезней им. Гельмгольца показывают,
		что наиболее вредное влияние на пользователей в оптическом диапазоне излучений экрана
		оказывает избыточный сине-фиолетовый и синий свет этого диапазона.
\end{par}

\begin{par}
	4. Нерациональное освещение рабочего места , повышенная блесткость и яркость на столе,
		пульсации светового потока.
\end{par}

\begin{par}
	5. Некачественный состав воздуха рабочей зоны. Наличие пыли. Недостаток
		легких отрицательных и избыток положительных ионов.
\end{par}
\begin{par}
 	6. Гиподинамия и длительные статические нагрузки на кисти рук.
\end{par}

\begin{par}
	7. Вибрация и шум.
\end{par}

\begin{par}
	8. Повышенные нервные, умственные и эмоциональные нагрузки.
\end{par}

\begin{par}
	9. Монотонность труда в сочетании с повышенным напряжением внимания и зрения.
\end{par}
\begin{par}
	10. Опасность поражения электрическим током.
\end{par}
\begin{par}
 	11. Опасность возникновения пожара.
\end{par}

\subsection{Организация рабочего места оператора ПЭВМ}
\subsubsection{Помещение}
Помещение с ПЭВМ расположено в отдельно стоящем административном здании без соприкосновения
с внешними стенами и посторонними комнатами. Трассы обычного и пожарного водоснабжения,
отопления и канализации вынесены за пределы помещения  и не находятся непосредственно над ним
на верхних этажах.


Помещение,где располагается рабочее место, оборудовано защитным заземлением в соответствии
с требованиями по технической эксплуатации. Отсутствуют линии, рядом проходящих, силовых
кабелей, создающих помехи в работе с ПЭВМ.


Площадь на одно рабочее место  составляет  12.0 кв.м. ,а обьем 31.2 кубических метра.
Звукоизоляция ограждающих конструкций отвечает гигиеническим требованиям и обеспечивает
нормируемые параметры шума (не более 50 дБА). Помещение оборудовано системой отопления и
кондиционирования воздуха . Для внутренней отделки использовались диффузноотражающие
материалы с коэффициентом отражения для потолка-0.7-0.8; для стен-0.5-0.6; для пола-0.3-0.5.


Для окраски стен  применялись краски холодных тонов: светло-зеленого и светло-серого цветов.
Все указанные параметры соответствуют  требованиям  санитарно-технических норм.


\subsubsection{Освещение}
В помещении с ПЭВМ  применяется комбинированное освещение: естественное и искусственное.
Естественное освещение осуществляется через светопроемы, ориентированные на  северо-восток
и обеспечивают коэффициент естественной освещенности (КЕО)не ниже 1.2 процента.
Искусственное освещение  осуществляется системой общего равномерного освещения.
Освещенность на поверхности стола в зоне размещения рабочего документа составляет 300-500лк 
Освещение не  создает бликов на поверхности экрана.


Освещенность поверхности экрана не превышает 300лк.  В качестве источников света при искусственном
освещении применяются  люминисцентные лампы с экранирующими решотками.


Общее освещение, при использовании люминисцентных светильников, 
выполнено в виде сплошной линий светильников, расположенных сбоку от рабочего места.
Рабочий стол размещен таким образом , чтобы дисплей был ориентирован боковой стороной к
световым проемам, чтобы свет падал преимущественно слева, что соответствует санитарно-техническим
нормам.



\subsubsection{Характеристика микроклимата}
Необходимым условием жизнедеятельности человека является поддержание постоянной
температуры тела благодаря терморегуляции, т. е. способности 
организма регулировать отдачу тепла в окружающую среду. Принцип нормирования
микроклимата -- создание оптимальных условий для теплообмена тела человека с окружающей
средой \cite{bjd}. Вычислительная техника является источником существенных тепловыделений, что может
привести к  повышению температуры и снижении относительной влажности.
В соответствии с требованиями СанПиН 2.2.2.542-96  (п.5.2) в помещениях с ПЭВМ  должны
соблюдаться оптимальные параметры микроклимата. Для повышения влажности воздуха следует
применять увлажнители воздуха, заправленные дистилированной водой  или свежей кипяченой
питьевой водой.


Содержание вредных химических веществ в помещении с ПЭВМ не должно превышать предельно
допустимых концентраций загрязняющих веществ в атмосферном воздухе населенных пунктов.


\subsubsection{Эргономические требования рабочему месту оператора ПЭВМ}
Проектирование рабочих мест, снабженных видеотерминалами , относится к числу важных
проблем эргономического проектирования в области вычислительной техники. Рабочее место и
взаимное расположение всех его элементов должно соответствовать антропометрическим,
физическим и психологическим требованиям. Большое значение имеет также характер работы.
В частности, при организации рабочего места технолога-оператора ПЭВМ, должны быть соблюдены
следующие основные условия: оптимальное расположение оборудования, входящего в состав
рабочего места и достаточное рабочее пространство позволяющее осуществлять все необходимые
движения. Эргономическими аспектами проектирования видеотерминальных рабочих мест являются:
\begin{itemize}
	\item{} высота рабочей поверхности;
	\item{} размеры пространства для ног;
	\item{} требования к расположению документов
на рабочем месте;
	\item{} характеристики рабочего кресла;
	\item{} требования к поверхности рабочего стола;
	\item{} регулируемость элементов рабочего места.
\end{itemize}

Необходимо выполнять ниже перечисленные требования.


1. Высота рабочей поверхности нерегулируемого стола должна составлять  725мм;
регулируемого от 680 до 800 мм. Размеры рабочей поверхности стола: глубина
не менее  600 мм.; ширина не менее 1200мм. Рабочий стол должен иметь пространство для ног
высотой не  менее 600мм. , шириной не менее 500мм,  глубиной на уровне колен --- не менее 450мм
и на уровне вытянутых ног --- не менее 650мм. Рабочая поверхность стола не должна иметь острых
углов и краев. Покрытие рабочей поверхности стола должно быть выполнено из диффузноотражающего
материала с коэффициентом отражения 0.45--0.50.


2. Рабочий стул (кресло) должен обеспечивать поддержание физиологически 
рациональной рабочей позы оператора ПЭВМ в процессе трудовой 
деятельности , создавать условия для изменения позы с целью снижения 
статического напряжения мышц шейно-плечевой области и спины, а также
для исключения нарушения циркуляции крови в нижних конечностях.
Рабочий стул (кресло) должен быть подъемно-поворотным и регулируемым
по высоте и углам наклона сиденья и спинки, а также расстоянию спинки от 
переднего края сиденья. Поверхность сиденья должна иметь ширину и глубину не менее 400мм.
Должна быть предусмотрена возможность изменения угла наклона поверхности сиденья от 15
градусов вперед  до 5 градусов назад. Высота стула должна регулироваться в пределах от 400 до
550 мм. Опорная поверхность спинки стула должна иметь высоту  300 мм (плюс--минус 20мм),
ширину не менее 380мм. Радиус кривизны в горизонтальной плоскости 400мм.
Угол наклона спинки в вертикальной плоскости должен регулироваться
 в пределах 0 плюс-минус 30 градусов от вертикального положения..Расстояние
спинки от переднего края сиденья должно регулироваться в пределах от 260
до 400мм. Рабочий стул должен иметь подлокотники длиной не менее 250мм, 
и шириной 50-70мм.


3. Дисплей на рабочем месте оператора  должен располагаться так, чтобы 
изображение в любой его части было различимо без необходимости поднять
или опустить голову. Дисплей должен быть установлен ниже уровня глаз оператора.
Угол наблюдения экрана оператором относительно горизонтальной линии взгляда не должен
превышать 60 градусов (ГОСТ Р 50923-96).



4. Клавиатура не должна препятствовать оптимальной видимости экрана. Она 
должна иметь возможность свободного перемещения .Клавиатуру следует
располагать на поверхности стола на расстоянии  от 100 до 300мм  от переднего
края, обращенного к оператору.


5. При работе с ПЭВМ очень важную роль играет соблюдение правильного 
режима труда и отдыха . В противном случае у персонала отмечается значительное напряжение
зрительного аппарата, появляются жалобы на головную боль, усталость, нарушение сна, боли в
области пояснице, шеи и рук. Продолжительность непрерывной работы с ПЭВМ без регламентированного
перерыва не должна превышать 2 часов. Во время перерывов , с целью снижения нервно--эмоционального
напряжения, утомления зрительного анализатора, устранения влияния гиподинамии и гипокинезии,
предотвращения развития познотонического утомления целесообразно выполнять комплексы упражнений,
направленных на устранение на устранение отрицательного воздействия производственных факторов.


6. Необходимо соблюдение гигиенических требований по обеспечению защиты от неблагоприятного влияния электромагнитных полей.
Обеспечение защиты работающих от неблагоприятного воздействия 
электромагнитных полей осуществляется путем проведения организационных,
инженерно-технических и лечебно-профилактических мероприятий.
Организационные мероприятия включают в себя: выбор рациональных режимов работы оборудования;
выделение зон воздействия; расположение рабочих мест и маршрутов передвижения обслуживающего
персонала на расстоянии от источников электромагнитных полей, обеспечивающих предельно допустимые
уровни электромагнитных полей; соблюдение правил безопасной эксплуатации оборудования;
периодический контроль уровней электромагнитных полей.


Инженерно--технические мероприятия должны обеспечивать снижение уровней 
электромагнитных полей на рабочих местах путем внедрения новых технологий
и применения средств коллективной и индивидуальной защиты. Коллективные
и индивидуальные средства защиты должны обеспечивать снижение 
неблагоприятного влияния электромагнитных полей и не должны оказывать 
вредного воздействия. Они должны изготавливаться с использованием технологий,
основанных на экранировании, отражении и поглощении электромагнитных излучений.
Размещение оборудования и выполнение разводки токоведущих цепей, должно обеспечивать
минимизацию их  вредного влияния. Для обеспечения достаточной контратсности и исключения
бликов необходимо применять приэкранные фильтры, которые к тому же уменьшают
заметность мельканий. Фильтр должен иметь антибликовое покрытие, желательно с двух сторон.
Кроме этого приэкранный фильтр с проводящим слоем, соединенный с заземляющей шиной,
защищает от статического электричества. Для уменьшения влияния рентгеновского излучения и
электромагнитного  поля необходимо находиться не ближе 1.22м от задних стенок дисплея и от
экрана дисплея не ближе 0.5 метра. Лечебно-профилактические мероприятия проводятся в целях
предупреждения и  раннего обнаружения изменений здоровья. Все лица, профессионально связанные с
обслуживанием и эксплуатацией источников электромагнитных полей, должны проходить периодические
профилактические медосмотры в соответствии с действующим законодательством.

В данном проекте режим работы оператора ПЭВМ не связан с необходимостью 
постоянного пребывания  перед экраном монитора. Такая потребность возникает только в случае
изменения режимов обработки продукции и в случае возникновения внештатных ситуаций.
Суммарная продолжительность его нахождения перед экраном ПЭВМ  не превышает двух часов в смену.
Следовательно вредное воздействие излучений для него будет минимальным, однако, не смотря
на это, выполнение вышеуказанных требований является обязательным условием безопасной работы.


\subsection{Расчет освещенности}
Расчет освещенности рабочего места оператора ПЭВМ сводится к выбору 
системы освещения, определению необходимого светильников  их типа и
размещения.В данном проекте используются люминисцентные лампы ЛБ40-1,
светильники типа ОД. Расчет ведется методом светового потока. \\*
Исходные данные:
\begin{itemize}
	\item{} длина комнаты-4 метра;
	\item{} ширина комнаты-3 метра;
	\item{} площадь комнаты-12 метров кв.
\end{itemize}
Световой поток, падающий на поверхность определяется по формуле:
\begin{equation}
F = \frac{ E \times K \times S \times Z }{n}
\end{equation}
\begin{ESKDexplanation}
	\item[где ] $F$ --- рассчитываемый световой поток (Лм);
	\item{} $E$ --- нормируемая минимальная освещенность (300Лк);
	\item{} $S$ --- площадь освещаемого помещения;
	\item{} $Z$ --- отношение средней освещенности к минимальной, принимается равной 1.1;
	\item{} $K$ --- коэффициент запаса, учитывающий уменьшение светового потока лампы
	в результате загрязнения светильников в процессе эксплуатации=1.5
	\item{} $n$ --- коэффициент использования, выражается отношением светового потока,
	падающего на расчетную поверхность, к суммарному потоку всех ламп и    и
	исчисляется в долях единицы. Зависит от характеристик светильника, размеров помещения,
	окраски стен и потолка, характеризуемых коэффициентами отражения (Pc , Pп ).
\end{ESKDexplanation}
В нашем случае:
$Р_{с} = 40\%$ \\*
$Р_{п} = 70\%$ \\ *
Данные коэффициенты определяются по таблице в зависимости от индекса помещения I.

\begin{equation}
I = \frac{S}{h(A + B)}
\end{equation}
\begin{ESKDexplanation}
	\item[где ]{} $I$ --- индекс помещения;
	\item{} $S$ --- площадь помещения;
	\item{} $h$ --- расчетная высота подвеса;
	\item{} $A$ --- ширина помещения;
	\item{} $B$ --- длина помещения.
\end{ESKDexplanation}

Подставляем все значения в формулу и получаем: $I = \frac{12}{2.45 \times (3 + 4)} = 0.7$. \\*
Зная индекс помещения по таблице находим коэффициент использования $n = 0.34$ \\
определяем световой поток: $F = \frac{300 \ times 1.5 \times 12 \times 1.1}{0.34} = 17470.6$ (Лм).
В нашем случае используются лампы типа ЛБ40-1 световой поток которой равен 4320 Лм. \\*
Расчитываем необходимое количество ламп:
\begin{equation}
N = \frac{F}{F_{л}}
\end{equation}
\begin{ESKDexplanation}
	\item[где ]{} $F_{л}$ — световой поток лампы.
\end{ESKDexplanation}
Подставляем известные значения и получаем: $N = \frac{17470.6}{4320.0} = 4$ (шт.).

В данном проекте используем светильники типа ОД. Каждый светильник комплектуется двумя лампами.
Располагаем светильники в один ряд ,вдоль большей стороны помещения, в соответствии с требованиями
санитарно--гигиенических норм.


\subsection{Расчет уровня шума}
В помещениях с ПЭВМ одним из неблагоприятных факторов вредного воздействия производственной среды
на организм человека является шум. В совокупности с повышенным зрительным напряжением,
нервно--эмоциональными нагрузками он оказывает отрицательное влияние, кроме всего прочего,
и на результаты производственной деятельности оператора ПЭВМ.
Это может привести к принятию неверного решения и , как результат, к выпуску бракованной
продукции. Что в современных условиях является весьма нежелательным фактом.


В данном проекте рассматривается вариант при котором в помещении оператора
находится следующее оборудование:
\begin{itemize}
	\item{} принтер, уровень звукового давления равен 45 дБ;
	\item{} кондиционер, уровень звукового давления равен 42 дБ;
	\item{} монитор, уровень звукового давления равен 17 дБ;
	\item{} клавиатура, уровень звукового давления равен 10 дБ;
	\item{} вентилятор, уровень звукового давления равен 45 дБ;
	\item{} жесткий диск, уровень звукового давления равен 40 дБ.
\end{itemize}


Уровень шума, возникающий от нескольких некогерентных источников, работающих одновременно,
подсчитывается на основании принципа энергетического суммирования излучений отдельных источников.
\begin{equation}
L = 10 Lg\sum_{i=1}{n} 10^{0.1L_i}
\end{equation}
\begin{ESKDexplanation}
	\item[где ]{} $L_i$ --- уровень звукового давления i-того источника шума;
	\item{} $n$ --- количество источников шума.
\end{ESKDexplanation}
Подставляем все значения и получаем: $L = 10Lg (10^{4.5} + 10^{4.3} + 10^{4.0} + 10^{4.5} + 10^{1.7} + 10^{1.0}) = 49.7$ (дБ).

Полученное значение не превышает допустимый уровень шума для рабочего места оператора ПЭВМ.
Кроме этого снизить уровень шума в помещении  можно использованием звукопоглощающих материалов
с максимальными коэффициентами звукопоглощения в области частот 63--8000 Гц
для отделки помещений. (СанПиН 2.2.2.542-96).

\subsection{Электробезопасность}
\subsubsection{Общие правила электробезопасности}
К работе с электроустановками допускаются лица, не моложе 18 лет, прошедшие обучение по
профессии, медицинский осмотр и имеющие соответствующую группу по электробезопасности.
Неэлектрическому персоналу присваивается первая группа электробезопасности.
Освидетельствование данной группы работников проводится с периодичностью не реже одного раза в год.
Проверка знаний работников подразделяется на первичную и периодическую, а периодическая,
в сою очередь, подразделяется на очередную и внеочередную. Первичная
проверка знаний  проводится у работников, впервые поступивших на работу,
связанную с обслуживанием электроустановок, или при перерыве в проверке
знаний более трех лет.


Очередная проверка знаний проводится у работников один раз в год.
Внеочередная проверка знаний  проводится не зависимо от срока предидущей
проверки:
\begin{itemize}
	\item{} при введении в действие новых или переработанных норм и правил;
	\item{} при нарушении работником требований нормативных актов по охране труда;
	\item{} при перерыве в работе в данной должности более шести месяцев.
\end{itemize}

\subsubsection{Действие электрического тока на организм человека}
Первая группа по электробезопасности  присваивается неэлектрическому персоналу, связанному с работой,
при выполнении которой может  возникнуть опасность поражения электрическим током.
Электрический ток оказывает на человека биологическое, электролитическое и термическое воздействие.


Биологическое воздействие выражается в раздражении и возбуждении нервных
клеток организма, что приводит к непроизвольным судорожным сокращениям мышц, нарушению
работы нервной системы, органов дыхания и кровообращения.


Электрическое воздействие проявляется в разложении плазмы крови и других
органических жидкостей, что приводит к нарушению их физико-химического
состава.


Термическое воздействие сопровождается ожогами отдельных участков
тела и перегревом внутренних органов.



\subsubsection{Средства обеспечения электробезопасности}
Для защиты людей от вредного и опасного воздействия электрического тока,
электромагнитного поля и статического электричества разработана система 
организационных и технических мероприятий и средств, называемая
электробезопасностью. Из всего комплекса мер в данной работе предусмотрены
следующие:
\begin{itemize}
	\item{} назначение ответственного лиц;
	\item{} защитное заземление;
	\item{} проведение плановых ремонтов и проверок электрооборудования.
\end{itemize}


\subsubsection{Требование электробезопасности при работе на ПЭВМ}

\begin{par}
	1.	Использовать для электропитания только типовые и исправные кабели,
   сетевые провода, не допускать их перетирания.
\end{par}

\begin{par}
 	2. Перед включением питания необходимо убедиться в наличии и исправности
   защитного заземления, в исправности кабельных соединений,проводов,
   вилок, розеток.
\end{par}

\begin{par}
 	3. При работе на ПЭВМ допускается подключать и отключать разъемы 
    кабелей устройства только на обесточенном оборудовании.
\end{par}

\begin{par}
	4. Не вскрывать корпус дисплейного устройства, не устранять самому
    неисправности.
\end{par}

\begin{par}
	6. При внезапном отключении напряжения в сети , при отсутствии блока 
   бесперебойного питания, немедленно выключить ПЭВМ и периферийные
   устройства.
\end{par}

\begin{par}
	7. Не допускать попадания воды и другой жидкости.
\end{par}

\begin{par}
 	8. Не укладывать бумагу, папки и пр. на дисплей.
\end{par}


В данном проекте  в качестве помещения для размещения ПЭВМ используется 
специально оборудованное помещение с температурой до 30 градусов и 
влажностью не более 60\%, без токопроводящей пыли и химически
 активной среды. Следовательно его можно отнести к сухим отапливаемым помещениям без
повышенной опасности.


\subsection{Пожарная безопасность}
В электроустановках потребителей электроэнергии причины пожаров могут
быть электрического и неэлектрического характера.


\subsubsection{Причины электрического характера}
Причинами электрического характера могут быть: искрение в электрических
машинах и аппаратах; токи коротких замыканий и перегрузок проводников,
вызывающих их перегрев; плохие контакты в местах соединения проводников;
перегрузки и неисправность обмоток электрических машин и трансформаторов.


\subsubsection{Причины пожара неэлектрического характера}
Таковыми  являются: неосторожное обращение с открытым огнем; 
самовоспламенение и самовозгорание некоторых материалов; неисправность
оборудования, нарушения технологического процесса и др.


\subsubsection{Мероприятия по профилактике  и устранению пожаров}
Технические мероприятия: соблюдение противопожарных норм при сооружении
здания, устройстве отопления и вентиляции, выборе и монтаже электрооборудования и др.
Эксплуатационные мероприятия : предусматривают правильную техническую
эксплуатацию оборудования, правильное содержание зданий и территории предприятия
Организационные мероприятия: предусматривают обучение персонала противопожарным правилам,
издание необходимых инструкций и плакатов. Режимные мероприятия: ограничение или запрещение
применения открытого огня, курения и т. д.



\subsubsection{Средства пожаротушения}
Для эффективного и безопасного тушения различных веществ и материалов
необходим правильный выбор огнегасительных веществ и средств их подачи
в очаг возгорания. В зависимости от характеристик горючей среды пожары
делятся на 5 классов, для тушения каждого из которых рекомендуется использовать
соответствующие огнегасительные вещества. Пожары на электроустановках относятся к классу Е,
для тушения которых используются углекислотные, порошковые и углекислотно--бромэтиловые огнетушители.
Наиболее эффективным способом ликвидации пожара является его тушение на самом начальном этапе
загорания с помощью первичных средств пожаротушения.  В связи с этим к первичным средствам
пожаротушения  и их эксплуатации предъявляются следующие требования: 
\begin{itemize}
	\item{} они должны располагаться в непосредственной близости от помещения;
места их расположения должны быть известны всему персоналу; средства
пожаротушения должны находиться в постоянной готовности; количество и типы огнетушителей на каждом объекте должны соответствовать возможному
классу пожара и размерам помещения.
\end{itemize}

В соответствии с установленными нормами  принимаем для оснащения помещения АСУТП , при условии:
\begin{itemize}
	\item{} категория помещения  В;
	\item{} площадь помещения --- до 200 м.кв.;
	\item{} класс пожара --- Е,
\end{itemize}
один порошковый огнетушитель на 10 литров и два углекислотных  по 5 литр.


\subsection{Экологический анализ проекта}
ПЭВМ является малошумящим и поэтому вне помещения, в котором расположена, не создает
шумового загрязнения. Излучения, генерируемые ПЭВМ, задерживается экранами, и рассеивается в
пространстве (обратно пропорционально квадрата расстояния). ПЭВМ не оказывают вредного
воздействия на окружающую среду вне помещения, в связи с этим не требуется применение
дополнительных мероприятий по изоляции вредных факторов.
Отсутствие вредных выбросов в окружающую среду, не использование природных ресурсов,
позволяют определить работу ЭВМ как экологически безопасную. В ходе дипломного проектирования
разработан программный продукт, при реализации которого применялась только ЭВМ, вследствие
чего на окружающую среду не оказывалось никакого вредного воздействия. Практическое использование
программно-аппаратного комплекса на предприятии ''Волжский бекон'' так же не приведет к усилению
отрицательной нагрузки на окружающую среду.
