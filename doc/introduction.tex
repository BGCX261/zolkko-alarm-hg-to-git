%!TEX root = /Users/zolkko/Projects/zolkko-alarm/doc/main.tex
\section*{Введение}
\addcontentsline{toc}{section}{Введение}
\begin{par}
На сегодняшний день при проектировании систем промышленной автоматизации и устройств
бытового применения, перед проектировщиками и разработчиками встают вопросы не только технического
характера, но и вопросы экономической целесообразности применения тех или иных решений.
То есть при их решении необходимо учитывать не только системные характеристики применяемых для
реализации конечных устройств технологий, но и искать компромис с их стоимостью. При этом
на конечную стоимость изделия будут влиять цена применённых схемотехнических решений,
время затраченное на проектирование и реализацию устройства, цена применяемых средства автоматизации
и цена специалистов проектировщиков и разработчиков.
\end{par}

\begin{par}
Немаловажным при проектировании устройств является учёт стремления
современной европейской культуры не только к открытым, но и полностью свободным системам,
зачастую обладающим более качественными системными и
потребительскими характеристиками и способствующими общему\\*
научно-техническому прогрессу \cite{lessing}.
\end{par}


В дипломном проекте рассмотрены современные методы и средства проектирования и разработки
программно-аппаратным комплексов на базе микроконтроллеров AVR компании Ateml. При этом
в процессе проектирования и разработки отдавалось предпочтение именно свободному или
доступному по цене программному и аппаратному обеспечению. В качестве примера разрабатывался
программно-аппаратный комплекс <<Универсальная система терморегулирования на базе
микроконтроллера AVR cемейства XMega>>. Таким образом в процессе выполнения дипломного проекта
стало возможным проведение анализа пригодности для практического применения свободного,
бесплатного и доступного по цене программного и аппаратного обеспечения для целей промышленного
производства.

Программный код написанный в процессе выполнения дипломного проекта использует большинство
внешней периферии микроконтроллера. Таким образом, сформированный программный код и электрическая
принципиальная схема, позволяют оценить преимущества и недостатки использования
микроконтроллеров AVR семейства XMega.

\newpage{}

