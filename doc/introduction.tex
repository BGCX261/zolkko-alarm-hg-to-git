\section*{Введение}
\addcontentsline{toc}{section}{Введение}
\begin{par}
На сегодняшний день при проектировании систем промышленной автоматизации и устройств
бытового применения, перед проектировщиками и разработчиками встают вопросы не только технического
характера, но и вопросы экономической целесообразности применения тех или иных решений.
То есть при их решении необходимо учитывать не только системные характеристики применяемых для
реализации конечных устройств технологий, но и искать компромис с их стоимостью. При этом
на конечную стоимость изделия будут влиять цена применённых схемотехнических решений,
время затраченное на проектирование и реализацию устройства, цена применяемых средства автоматизации
и цена специалистов проектировщиков и разработчиков.
\end{par}

\begin{par}
Не маловажным, при проектировании устройств является учёт стремления
современной европейской культуры не только к открытым, но и полностью свободным системам,
зачастую обладающим более качественными системными и
потребительскими характеристиками и способствующими общему\\*
научно-техническому прогрессу\cite{lessing}.
\end{par}
\newpage{}

